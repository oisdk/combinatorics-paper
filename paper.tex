\PassOptionsToPackage{dvipsnames}{xcolor}
\documentclass{article}

\usepackage[mono=false]{libertine}
\usepackage{catchfilebetweentags}
\usepackage{mathtools}
\usepackage{mathrsfs}
\usepackage{turnstile}
\usepackage{bbm}
\usepackage[greek, english]{babel}
\usepackage{stmaryrd}
\usepackage{latexsym}
\newcommand\doubleplus{+\kern-1.3ex+\kern0.8ex}
\newcommand\mdoubleplus{\ensuremath{\mathbin{+\mkern-8mu+}}}
% \usepackage[dvipsnames]{xcolor}
\usepackage{relsize}
\newcommand\doublelbrace{%
  {\{\mkern-4mu\lvert}%
}
\newcommand\doublerbrace{%
  {\rvert\mkern-4mu\}}%
}

\makeatletter
\newcommand\incircbin
{%
  \mathpalette\@incircbin
}
\newcommand\@incircbin[2]
{%
  \mathbin%
  {%
    \ooalign{\hidewidth$#1#2$\hidewidth\crcr$#1\bigcirc$}%
  }%
}
\newcommand{\oeq}{\ensuremath{\incircbin{=}}}
\makeatother
\makeatletter
\newcommand\insquarebin
{%
  \mathpalette\@insquarebin
}
\newcommand\@insquarebin[2]
{%
  \mathbin%
  {%
    \ooalign{\hidewidth$#1#2$\hidewidth\crcr$#1\bigbox$}%
  }%
}
\newcommand{\sqtri}{\ensuremath{\insquarebin{\triangle}}}
\makeatother
\newcommand{\circeps}{\ensuremath{\insquarebin{\epsilon}}}
\makeatother
\newcommand{\qeq}{\mathrel{\mathpalette\qeqf\relax}}
\newcommand{\qeqf}[2]{%
  \ooalign{$#1=$\cr
    \hidewidth\raisebox{1ex}{$#1\mkern0mu{\mathsmaller{\mathsmaller{?}}}$}\hidewidth\cr}}
\usepackage{ucs}
\DeclareUnicodeCharacter{8605}{\ensuremath{\leadsto}}
\DeclareUnicodeCharacter{951}{\textgreek{\texteta}}
\DeclareUnicodeCharacter{957}{\textgreek{\textnu}}
\DeclareUnicodeCharacter{961}{\textgreek{\textrho}}
\DeclareUnicodeCharacter{929}{\textgreek{\textRho}}
\DeclareUnicodeCharacter{954}{\textgreek{\textkappa}}
\DeclareUnicodeCharacter{928}{\textgreek{\textPi}}
\DeclareUnicodeCharacter{922}{\textgreek{\textKappa}}
\DeclareUnicodeCharacter{931}{\textgreek{\textSigma}}
\DeclareUnicodeCharacter{916}{\textgreek{\textDelta}}
\DeclareUnicodeCharacter{921}{\textgreek{\textIota}}
\DeclareUnicodeCharacter{8759}{\ensuremath{\dblcolon}}
\DeclareUnicodeCharacter{737}{\ensuremath{^\text{l}}}
\DeclareUnicodeCharacter{691}{\ensuremath{^\text{r}}}
\DeclareUnicodeCharacter{7523}{\ensuremath{_\text{r}}}
\DeclareUnicodeCharacter{8343}{\ensuremath{_\text{l}}}
\DeclareUnicodeCharacter{8718}{\ensuremath{\blacksquare}}
\DeclareUnicodeCharacter{10214}{\ensuremath{\llbracket}}
\DeclareUnicodeCharacter{10215}{\ensuremath{\rrbracket}}
\DeclareUnicodeCharacter{8857}{\ensuremath{\odot}}
\DeclareUnicodeCharacter{8860}{\oeq}
\DeclareUnicodeCharacter{9043}{\ensuremath{\sqtri}}
\DeclareUnicodeCharacter{3870}{\ensuremath{\cong}}
\DeclareUnicodeCharacter{8799}{\ensuremath{\qeq}}
\DeclareUnicodeCharacter{10181}{\ensuremath{\lbag}}
\DeclareUnicodeCharacter{10182}{\ensuremath{\rbag}}
\DeclareUnicodeCharacter{8760}{\ensuremath{-}}
\DeclareUnicodeCharacter{9428}{\ensuremath{\textcircled{ε}}}
\DeclareUnicodeCharacter{8623}{\ensuremath{\lightning}}
\DeclareUnicodeCharacter{10627}{\ensuremath{\doublelbrace}}
\DeclareUnicodeCharacter{10628}{\ensuremath{\doublerbrace}}
\DeclareUnicodeCharacter{10631}{\ensuremath{\llparenthesis}}
\DeclareUnicodeCharacter{10632}{\ensuremath{\rrparenthesis}}
\DeclareUnicodeCharacter{9666}{\ensuremath{\scriptscriptstyle\blacktriangleleft}}
\DeclareUnicodeCharacter{8855}{\ensuremath{\otimes}}
\DeclareUnicodeCharacter{894}{\ensuremath{\fatsemi}}
\DeclareUnicodeCharacter{65373}{\ensuremath{\}}}
\DeclareUnicodeCharacter{65371}{\ensuremath{\{}}
\DeclareUnicodeCharacter{120021}{\ensuremath{\mathcal{F}}}
\DeclareUnicodeCharacter{120017}{\ensuremath{\mathcal{B}}}
\DeclareUnicodeCharacter{8761}{\ensuremath{\coloneq}}
\DeclareUnicodeCharacter{8788}{\ensuremath{\mathrel{\mathop:}=}}
\DeclareUnicodeCharacter{119974}{\ensuremath{\mathcal{K}}}
\DeclareUnicodeCharacter{8492}{\ensuremath{\mathcal{B}}}
\DeclareUnicodeCharacter{8495}{\ensuremath{\mathcal{E}}}
\DeclareUnicodeCharacter{8466}{\ensuremath{\mathcal{L}}}
\DeclareUnicodeCharacter{8779}{\ensuremath{\equiv}}
\DeclareUnicodeCharacter{10626}{\ensuremath{:}}
\DeclareUnicodeCharacter{10629}{\ensuremath{\llparenthesis}}
\DeclareUnicodeCharacter{10630}{\ensuremath{\rrparenthesis}}
\DeclareUnicodeCharacter{9671}{\ensuremath{\Diamond}}
\DeclareUnicodeCharacter{9723}{\ensuremath{\Box}}
\DeclareUnicodeCharacter{9655}{\ensuremath{\triangleright}}
\DeclareUnicodeCharacter{7584}{\ensuremath{^\text{f}}}
\DeclareUnicodeCharacter{8346}{\ensuremath{_\text{p}}}
\DeclareUnicodeCharacter{8621}{\ensuremath{\leftrightsquigarrow}}
\DeclareUnicodeCharacter{12310}{\ensuremath{\lbparen}}
\DeclareUnicodeCharacter{12311}{\ensuremath{\rbparen}}
\DeclareUnicodeCharacter{7470}{\ensuremath{^\text{B}}}
\DeclareUnicodeCharacter{7468}{\ensuremath{^\text{A}}}
\DeclareUnicodeCharacter{7580}{\ensuremath{^\text{c}}}
\usepackage[references]{agda}

\newcommand{\Nat}{\AgdaDatatype{ℕ}}
\newcommand{\Int}{\AgdaDatatype{ℤ}}
\colorlet{AgdaKeyword}{BlueViolet}
\colorlet{AgdaFunction}{OliveGreen}
\colorlet{AgdaDatatype}{OliveGreen}
\colorlet{AgdaPostulate}{OliveGreen}
\colorlet{AgdaPrimitive}{OliveGreen}
\colorlet{AgdaRecord}{OliveGreen}
\colorlet{AgdaRecordtype}{OliveGreen}
\colorlet{AgdaPrimitiveType}{OliveGreen}
\colorlet{AgdaModule}{OliveGreen}
\colorlet{AgdaInductiveConstructor}{Sepia}
\colorlet{AgdaField}{Sepia}
\renewcommand{\AgdaKeywordFontStyle}[1]{\textbf{#1}}
\usepackage{csquotes}


\usepackage{tikz-cd}
\tikzcdset{
  arrow style=tikz,
  diagrams={>={Straight Barb[scale=0.8]}},
}
\usepackage{amsmath}
\usepackage{amsthm}
\renewcommand{\qed}{\ensuremath{\hfill\blacksquare}}

% Boxed theorem envs without italics

\usepackage[framemethod=tikz]{mdframed}

\newtheoremstyle{roman}
{0} % Space above
{0} % Space below
{} % Body font
{} % Indent amount
{\bfseries} % Theorem head font
{.} % Punctuation after theorem head
{.5em} % Space after theorem head
{} % Theorem head spec (can be left empty, meaning `normal')

\theoremstyle{roman}
\newmdtheoremenv[default,nobreak=true]{definition}{Definition}
\newmdtheoremenv[default,nobreak=true]{lemma}{Lemma}
\newmdtheoremenv[default,nobreak=true]{theorem}{Theorem}
\newmdtheoremenv[
  default,
  nobreak=true,
  linecolor=white,
  tikzsetting = {draw=black, line width = 1pt,dotted},
  ]{agdalisting}{Listing}

% Haskell bind operator >>=
\newcommand\hbind{%
  \ensuremath{\gg\mkern-5.5mu=}%
}
% Haskell apply operator <*>
\newcommand\hap{%
  \ensuremath{\mathbin{<\mkern-9mu*\mkern-9mu>}}%
}
% Haskell alternative operator <|>
\newcommand\halt{%
  \ensuremath{\mathbin{<\mkern-5mu\vert\mkern-5mu>}}%
}
% Haskell fmap operator <$>$
\newcommand\hfmap{%
  \ensuremath{\mathbin{<\mkern-6mu\$\mkern-6mu>}}%
}
% Haskell mappend operator <>
\newcommand\hcmb{%
  \ensuremath{\mathbin{\diamond}}%
}
% Haskell fish operator <><
\newcommand\hcmbin{%
  \ensuremath{\mathbin{\diamond\mkern-7.4mu\rtimes}}%
}
% Haskell Kleisli composition <=<
\newcommand\hkcomp{%
  \ensuremath{\mathbin{<\mkern-10mu=\mkern-7mu<}}
}
% Haskell forall
\newcommand\hforall{%
  \ensuremath{\forall}%
}
% Unique membership \in!
\newcommand\inunique{%
  \mathrel{\in\mkern-6mu\raisebox{-0.5pt}{!}}
}
% Decide membership \in?
\newcommand\decin{%
  \mathrel{\in\mkern-6mu\raisebox{-0.5pt}{?}}
}
% Dot as wildcard character
\newcommand\wc{{\mkern 2mu\cdot\mkern 2mu}}
% Binary operator !
\newcommand\ind{\mathbin{!}}
\usepackage{subcaption}
\usepackage{locallhs2TeX}
\renewcommand{\Varid}[1]{\mathit{#1}}
\renewcommand{\Conid}[1]{\textcolor{OliveGreen}{\textsf{#1}}}
\usepackage{multicol}
\usepackage{todonotes}
\usepackage{tabularx}
\usepackage{pict2e}

% Proper double left paren [( (also ``plano-concave lens'')
\newcommand{\lbparen}{%
  \mathopen{%
    \sbox0{$()$}%
    \setlength{\unitlength}{\dimexpr\ht0+\dp0}%
    \raisebox{-\dp0}{%
      \begin{picture}(.32,1)
        \linethickness{\fontdimen8\textfont3}
        \roundcap
        \put(0,0){\raisebox{\depth}{$($}}
        \polyline(0.32,0)(0,0)(0,1)(0.32,1)
      \end{picture}%
    }%
  }%
}

% Proper double right paren [( (also ``plano-concave lens'')
\newcommand{\rbparen}{%
  \mathclose{%
    \sbox0{$()$}%
    \setlength{\unitlength}{\dimexpr\ht0+\dp0}%
    \raisebox{-\dp0}{%
      \begin{picture}(.32,1)
        \linethickness{\fontdimen8\textfont3}
        \roundcap
        \put(-0.08,0){\raisebox{\depth}{$)$}}
        \polyline(0,0)(0.32,0)(0.32,1)(0,1)
      \end{picture}%
    }%
  }%
}

\usepackage[square,numbers]{natbib}

\PrerenderUnicode{í}

\usepackage{hyperref}

% Nice typesetting for the definition of an inductive type.
% For the following Agda-like thing:
%
% data List a where
%   []   : List a
%   (::) : a -> List a -> List a    
%
% write:
% \begin{inductivetype}{\mathbf{List}(A)}
%   \inductivetypeclause{\left[ \right]}{\mathbf{List}(A)}
%   \inductivetypeclause{\wc \dblcolon \wc}{A \rightarrow \mathbf{List}(A) \rightarrow \mathbf{List}(A)}
% \end{inductivetype}

\newif\iffirstclauseofinductivetype
\newenvironment{inductivetype}[1]{%
  \firstclauseofinductivetypetrue
  \begin{equation}
    \begin{alignedat}{3}
      #1
    }{%
    \end{alignedat}
  \end{equation}%
}
\newcommand{\inductivetypeclause}[2]{%
  \iffirstclauseofinductivetype
    \coloneqq
  \else
    \\ \vert
  \fi
  & \; #1 &:&& \; #2 ;
  \firstclauseofinductivetypefalse
}
\newmdenv[
leftmargin = 0pt,
innerleftmargin = 1em,
innertopmargin = 0pt,
innerbottommargin = 0pt,
innerrightmargin = 0pt,
rightmargin = 0pt,
linewidth = 3pt,
topline = false,
rightline = false,
bottomline = false
]{leftbar}

\makeatletter
\renewenvironment{proof}[1][\proofname]{%
  \begin{leftbar}%
  \par
  \pushQED{\qed}%
  \normalfont \topsep6\p@\@plus6\p@\relax
  \list{}{%
    \leftmargin=.5em
    % \rightmargin=\leftmargin
    \settowidth{\itemindent}{\itshape#1}%
    \labelwidth=\itemindent
    % the following line is not needed with amsart, but might be with other classes
    \parsep=0pt \listparindent=\parindent 
  }
\item[\hskip\labelsep
  \itshape
  #1\@addpunct{.}]\ignorespaces
}{%
  \popQED\endlist\@endpefalse
  \end{leftbar}
}
\makeatother

%%% Local Variables:
%%% mode: latex
%%% TeX-master: "paper"
%%% End:

\author{
  Donnacha Oisín Kidney \and
  Gregory Provan \and
  Nicolas Wu
}
% \affiliation{University College Cork}
% \email{o.kidney@cs.ucc.ie}
% \author{Gregory Provan}
% \affiliation{University College Cork}
% \email{g.provan@cs.ucc.ie}
% \author{Nicolas Wu}
% \affiliation{Imperial College London}
% \email{n.wu@imperial.ac.uk}

\title{Finiteness in Cubical Type Theory}

% \keywords{Agda, Homotopy Type Theory, Cubical Type Theory,
%   Dependent Types, Finiteness, Topos, Kuratowski finite}


\begin{document}
% \setlength{\abovedisplayshortskip}{-\baselineskip}%

\maketitle

% \begin{abstract}
%   We study five different notions of finiteness in Cubical Type Theory and prove
%   the relationship between them.
%   In particular we show that any totally ordered Kuratowski finite type is
%   manifestly Bishop finite.

%   We also prove closure properties for each finite type, and classify them
%   topos-theoretically.
%   This includes a proof that the category of decidable Kuratowski finite sets
%   (also called the category of cardinal finite sets) form a \(\Pi\)-pretopos.

%   We then develop a parallel classification for the countably infinite types, as
%   well as a proof of the countability of \(A^\star\) for a countable type \(A\).

%   We formalise our work in Cubical Agda, where we implement a library for proof
%   search (including combinators for level-polymorphic fully generic currying).
%   Through this library we demonstrate a number of uses for the computational
%   content of the univalence axiom, including searching for and synthesising
%   functions.
% \end{abstract}
\section{Introduction}
\subsection{Foreword}
Proofs in constructive mathematics are often more substantial than their
classical counterparts.
``Substantial'' here doesn't mean the literal length of the proofs (although the
double meaning is not lost on anyone who has had to struggle with a modern
constructive proof assistant), but rather refers to their \emph{content}: a
constructive proof contains more \emph{stuff} than a classical one.

If we were to prove that a set was finite in classical mathematics, we might
show that it is in bijection with another finite set, say a finite prefix of the
natural numbers.
In semi-formal notation, to say that a set \(A\) is finite we would show the
following:
\begin{equation} \label{example-finite}
  \mathbf{Finite}(A) \coloneqq \exists (n : \mathbb{N}) . \; A \Leftrightarrow \left\{ m \mid m < n\right\}
\end{equation}
Where \(A~\Leftrightarrow~B\) means ``a bijection between \(A\) and \(B\)''.

Classically, a proof of equation~\ref{example-finite} is nothing more than
\emph{evidence} for the proposition of equation~\ref{example-finite}.
A constructive proof of \ref{example-finite}, however, is a mathematical object
in its own right, containing:
\begin{enumerate}
  \item A number \(n\) representing the cardinality of \(A\).
  \item A function from \(A\) to \(\left\{ m \mid m < n \right\}\).
  \item A function from \(\left\{ m \mid m < n \right\}\) to \(A\).
\end{enumerate}
In other words, it makes sense (in constructive mathematics) to talk about a
function of type \(\mathbf{Finite}(A)~\rightarrow~\mathbb{N}\), which takes a
\emph{proof} of the finiteness of \(A\) and returns \(A\)'s cardinality.

Such functions are perhaps even more concrete than this description suggests:
many constructive proofs are in fact computer programs in disguise (and
vice-versa \cite{wadlerPropositionsTypes2015}).
All ``proofs'' in this paper correspond to honest-to-goodness proofs in a
programming language (Cubical Agda \cite{vezzosiCubicalAgdaDependently2019} in
our case) which can be run on a computer to produce real output.

We have now hopefully established, albeit informally, what it means for
constructive proofs to be ``substantial''.
Clearly this extra information carried by proofs can add richness to their
study: we will see, for instance, no fewer than five finiteness predicates in
this paper which would collapse to one in a classical setting.
The extra information also presents a challenge, however: occasionally we will
see that we have revealed \emph{too much} information while proving some
property.

Consider the category of finite sets.
We will explore this category in much more detail later, but for now we just
need to know that the objects of this category are---shockingly---finite sets.
Constructively, the objects of this category can be thought of as \emph{pairs},
where the first element is a set (like \(\left\{ \text{True}, \text{False}
\right\}\), the two-element set of the booleans) and the second is a proof that
that set is finite.
If we take those proofs to be the kind given in equation~\ref{example-finite},
we should see that they contain a bijection from the booleans to the set
\(\left\{ 0, 1 \right\}\): however \emph{two} such bijections exist!
\begin{figure}[h]
  \centering
  \begin{subfigure}[b]{.45\textwidth}
    \centering
    \begin{tikzcd}
      \text{True}  \ar[rr, leftrightarrow] & & 0 \\
      \text{False} \ar[rr, leftrightarrow] & & 1
    \end{tikzcd}
  \end{subfigure}
  \begin{subfigure}[b]{.45\textwidth}
    \centering
    \begin{tikzcd}
      \text{True}  \ar[rrd, leftrightarrow, start anchor=south east, end anchor=north west] & & 0 \\
      \text{False} \ar[rru, leftrightarrow, start anchor=north east, end anchor=south west, crossing over] & & 1
    \end{tikzcd}
  \end{subfigure}
  \caption{The Two Bijections Between the Booleans and \(\left\{ 0, 1 \right\}\)}
\end{figure}
This means that where we wanted one object to correspond to the finite set
\(\left\{ \text{True}, \text{False} \right\}\), we instead have two.
In fact, for any finite set of cardinality \(n\) we will have \(n!\) finiteness
proofs, and therefore \(n!\) objects.
We have constructed, in other words, entirely the wrong category.

Being able to hide the extra information in a constructive proof when we need to
is clearly important.
HoTT (and CuTT by extension) facilitate this hiding: how exactly it is
accomplished will be explored in the subsequent sections.

Finally, for all of its benefits, constructive mathematics (especially
dependently-typed constructive mathematics on a computer) still suffers from
many painful drawbacks.
As already hinted at, proofs in a constructive setting can tend to be verbose,
especially when the arbiter of correctness is as strict as a computer, which
neither understands nor trusts the time-honoured technique of ``the other cases
are obviously true''.
While the goal of using a computer to automate the boring parts of a proof still
seems attainable, too often getting a computer involved in a proof means adding
thousands of lines of tedium to satisfy its lack of intuition.

Simple proof search over a finite domain is a simple, if effective, way to
relieve some of the tedium.
Many libraries in Agda, Coq, and other languages make use of this fact.
While the idea for finite proof search is simple, we found that implementing one
revealed a number of interesting theoretical considerations:
\begin{description}
  \item[How should we construct search spaces?]
    We found that a \(Pi\)-pretopos provided the appropriate ``generator
    language'' for constructing search spaces for finite types.
  \item[How do we deal with functions?]
    We of course would like to be able to search over functions: \todo{?}
  \item[How do we talk about ``searchability''?]
    \todo{Omniscience}
\end{description}
Finiteness is a prime example of a phenomenon which happens quite frequently in
dependently-typed constructive mathematics: through building a practical program
we unearth theoretically interesting results.
\todo{rephrase}
\subsection{Contributions}
We are interested in constructive notions of finiteness.
In contrast to classical finiteness, in a constructive setting there is a wide
variety of predicates which all have some claim to being the formal
interpretation of ``finiteness''.
Broadly, they vary along two axes: informativeness (i.e. how many properties
of the underlying type we can derive from a proof that it is finite), and
restrictiveness (i.e. how many types conform to its particular variety of
finiteness).
We will explore five of these predicates, proving formally their relations and
implications.

We will then focus on one of the predicates: decidable Kuratowski finiteness.
We will construct a category of decidable Kuratowski-finite types, and show that
it forms a \(\Pi\)-pretopos.

We will then take a brief detour into predicates for countably infinite types,
and compare its landscape to that of the finite predicates.

Finally, we will present a case-study in a practical application of the
predicates for finiteness we have constructed: a library for total proof search.
We will show how our earlier work gives a good theoretical foundation for this
library, and allows for a simple and principled API, and we will demonstrate how
some of the more complex theoretical results allow for a more powerful library
with capabilities not seen in comparable proof search libraries.

\begin{figure*}
  \centering
  \begin{tikzcd}[cramped, row sep=small, column sep=2.3em]
    {} &
    {} \ar[ddddd, dash, start anchor={[yshift=4ex]}, end anchor={[yshift=-4ex]}] &
    \text{Non Discrete} &
    \ar[ddddd, dash, dashed, start anchor={[yshift=4ex]}, end anchor={[yshift=-4ex]}] &
    \text{Discrete} &
    {}
    \\ %%%%%%%%%%%%%%%%%%%%%%%%%%%%%%%%%%%%%%%%%%%%%%%%%%%%%%%%%%%%%%%%%%%%%%%%%
    \ar[rrrr, dash, start anchor={[xshift=-6ex]}, end anchor={[xshift=10ex]}] &
    &
    &
    &
    {}
    &
    {}
    \\ %%%%%%%%%%%%%%%%%%%%%%%%%%%%%%%%%%%%%%%%%%%%%%%%%%%%%%%%%%%%%%%%%%%%%%%%%
    \text{Ordered} &
    &
    \text{\parbox{1.8cm}{\centering Manifest Enumerable}}
    \ar[
      from=rr,
      bend left=15,
      crossing over,
      start anchor=south west,
      end anchor={[yshift=-2ex]east}
      ]
    \ar[
      rr,
      bend left=15,
      "\text{Discrete}" description,
      start anchor={[yshift=2ex]east},
      end anchor=north west
      ]
    \ar[ddd, crossing over]
    &
    &
    \text{Split Enumerable}
      \ar[d, xshift=-1ex, bend right=30]
    &
    \\ %%%%%%%%%%%%%%%%%%%%%%%%%%%%%%%%%%%%%%%%%%%%%%%%%%%%%%%%%%%%%%%%%%%%%%%%%
    &
    &
    &
    &
    \text{Manifest Bishop}
      \ar[u, xshift=1ex, bend right=30]
    &
    \\ %%%%%%%%%%%%%%%%%%%%%%%%%%%%%%%%%%%%%%%%%%%%%%%%%%%%%%%%%%%%%%%%%%%%%%%%%
    \ar[rrrr, dash, dashed, start anchor={[xshift=-6ex]}, end anchor={[xshift=10ex]}] &
    &
    &
    &
    {}
    &
    \\ %%%%%%%%%%%%%%%%%%%%%%%%%%%%%%%%%%%%%%%%%%%%%%%%%%%%%%%%%%%%%%%%%%%%%%%%%
    \text{Unordered} &
    {} &
    \text{Kuratowski}
    \ar[rr, crossing over, bend left=15, "\text{Discrete}" description, start anchor=north east, end anchor=north west]
    \ar[from=rr, crossing over, bend left=15, start anchor=south west, end anchor=south east]
    &
    {}
    &
    \text{Cardinal}
      \ar[from=uu, crossing over]
      \ar[uuu, crossing over, "\text{Ord}" description, end anchor=east, start
      anchor=east, in=-15, out=20]
    &
    \\
    & & & & &
    \\
    & \ar[rrrr, "\text{More Restrictive}" description, end anchor={[xshift=-6ex]}]
    &
    &
    &
    &
    \ar[uuuuuu, "\text{More Informative}" {description, near start}]
    {}
  \end{tikzcd}
  \caption{Classification of finiteness predicates according to whether they are
    discrete (imply decidable equality) and whether they imply a total order.}
  \label{finite-classification}
\end{figure*}%

%%% Local Variables:
%%% mode: latex
%%% TeX-master: "../paper"
%%% End:

\subsubsection{Finiteness Predicates}
Our first contribution is a cataloguing and characterisation of five finiteness
predicates, shown in figure~\ref{finite-classification}.

While our focus is decidable Kuratowski finiteness, in order to prove the
closure properties that we are interested in we will need to work with the
intermediate predicates as well (though they are also interesting in their own
right).

Since we have a proof-search library in mind, we will have to consider how to
present a good API for constructing proofs of finiteness.
In this vein we will show that some simple base types are finite
(\(\mathbf{Bool}\), \(\top\), \(\bot\), etc), and we will provide functions to
combine finiteness proofs together into more complex types (i.e. \(\times\),
\(\uplus\), \(\Sigma\), \(\Pi\), etc).
As we will see this choice of toolbox for building finiteness proofs actually
has a deep categorical justification, but at this point it makes sense to look
at it as just a suitably general and powerful API.



Our focus in this work is decidable Kuratowski finiteness.
We will characterise it, its strength, and its uses, and in particular we will
show that it forms a \(\Pi\)-pretopos.
Our proofs follow the structure of \cite[Chapters 9, 10]{hottbook} and
\cite{rijkeSetsHomotopyType2015}.
We will explore this concept in more detail later, but crucially we will need to
provide proofs of closure under \(\times\), \(\rightarrow\), etc.

In order to prove these closure properties, we will define and use some
intermediate finiteness predicates, classified in
figure~\ref{finite-classification}.
These predicates are also interesting in their own right, though!

The arrows in figure~\ref{finite-classification} are implication functions
between each predicate.
In section~\ref{relations} we will provide a function which inhabits each arrow.
Each unlabelled arrow is an unconditional implication: every manifest Bishop
finite set is cardinal finite (lemma~\ref{manifest-bishop-to-cardinal}), for
instance.
The labelled arrows are strengthening proofs: every manifest enumerable set
\emph{with decidable equality} is split enumerable
(lemma~\ref{manifest-enum-to-split-enum}).
Our most significant result here is the proof that a cardinal finite set with a
decidable total order is manifestly Bishop finite.

\subsubsection{Countability Predicates}
After the finite predicates, we will briefly look at the infinite countable
types, and classify them in a parallel way to the finite predicates
(section~\ref{infinite}).
\subsubsection{Search}
All of our work is formalised in Cubical Agda
\cite{vezzosiCubicalAgdaDependently2019}.
We will make mention of the few occasions where the formalisation of some proof
is of interest, but the main place where we will discuss the code is in
section~\ref{search}, where we implement a library for proof search, based on
omniscience and exhaustibility.
This library relies directly on the computational content of univalence, which
allows us to (for instance) have \(\Pi\) types in the domain of the search
space.
The language of a \(\Pi\)-pretopos turns out to be exactly the right language
for constructing ``generator''-like expressions.

The full code of the formalisation is available at
\url{https://github.com/oisdk/finiteness-in-cubical-type-theory}.
\subsection{Notation and Background}
\subsubsection{Notation}
We work in Cubical Type Theory \cite{cohenCubicalTypeTheory2016}.
For the various type formers we use the following notation:
\begin{description}
  \item[Type] We use \(\mathbf{Type}\) to denote the universe of (small) types.
    ``Type families'' are functions into \(\mathbf{Type}\).
  \item[0 , 1 , 2] We call the \(\mathbf{0}\), \(\mathbf{1}\), and
    \(\mathbf{2}\) types \(\bot\), \(\top\), and \(\mathbf{Bool}\) respectively.
    The single inhabitant of \(\top\) is tt, and the two inhabitants of
    \(\mathbf{Bool}\) are false and true.
    The ``negation'' of a type, written \(\neg A\), means \(A \rightarrow
    \bot\).
  \item[Dependent Sum and Product] We use \(\Sigma\) and \(\Pi\) for the
    dependent sum and product, respectively.
    The two projections from \(\Sigma\) are called fst and snd.
    In the non-dependent case, \(\Sigma\) can be written as \(\times\), and
    \(\Pi\) as \(\rightarrow\).
  \item[Disjoint Union] We define disjoint union as an inductive type.
    \begin{equation}
      \begin{alignedat}{3}
        A \uplus B \coloneqq & \;
        \text{inl} &: A \rightarrow A \uplus B ; \\
        | & \;  \text{inr} &: B \rightarrow A \uplus B ;
      \end{alignedat}
    \end{equation}
    It is also expressible with only \(\Sigma\):
    \(A \uplus B \simeq \Sigma(x : \mathbf{Bool}) , \text{if } x \text{ then
    } A \text{ else } B \).
  \item[Equalities, equivalences, and paths] We use the symbol \(\coloneqq\)
    for definitions.
    \(\simeq\) will be used for equivalences, and \(\equiv\) for equalities.
    Of course, we know that \((A \simeq B) \simeq (A \equiv B)\) by univalence,
    so the distinction isn't terribly important in usage: we will only use one
    or the other as a suggestion of how we constructed it or how it is to be
    used.
  \item[Lambdas] \todo{Figure out how to describe lambdas}
\end{description}
\subsubsection{Cubical Type Theory}
Cubical Type Theory \cite{cohenCubicalTypeTheory2016} is a constructive type
theory with an implementation in Cubical Agda
\cite{vezzosiCubicalAgdaDependently2019}.
It allows us to do much of the same theory as in HoTT, but crucially the
univalence ``axiom'' is a \emph{theorem}, giving it computational content.
\begin{definition}[Path Types] \label{path-types}
  The equality type (which we denote with \(\equiv\)) in CuTT is the type of
  Paths\footnotemark.
  The internal structure of paths is largely irrelevant to us here, as we will
  generally treat \(\equiv\) as a black-box equivalence relation with
  substitution and congruence.
\end{definition}

\footnotetext{
  Actually, CuTT does have an identity type with similar semantics to the
  identity type in MLTT.
  We do not use this type anywhere in our work, however, so we will not consider
  it here.
}
\begin{definition}[Homotopy Levels] \label{homotopy-types}
  Types in HoTT and CuTT are not necessarily sets, as they are in MLTT.
  Some have higher homotopies (paths which aren't unique).
  We actually have a hierarchy of complexity of structure of path spaces in
  types, starting with the contractions \cite[definition 3.11.1]{hottbook}, then
  the mere propositions \cite[definition 3.3.1]{hottbook}, and the sets
  \cite[definition 3.1.1]{hottbook}.
  \begin{alignat}{2}
    &\text{isContr}(A)    &&\coloneqq \Sigma(x : A) , \Pi(y : A) , (x \equiv y) \\
    &\text{isProp}(A)     &&\coloneqq \Pi(x, y : A) , (x \equiv y) \\
    &\text{isSet}(A)      &&\coloneqq \Pi(x, y : A) , \text{isProp}(x \equiv y)
  \end{alignat}
\end{definition}
\begin{definition}[Fibres] \label{fibres}
  A fibre \cite[definition 4.2.4]{hottbook} is defined over some function \(f :
  A \rightarrow B\).
  \begin{equation}
    \text{fib}_f(y) \coloneqq \Sigma(x : A) , (f (x) \equiv y)
  \end{equation}
\end{definition}
\begin{definition}[Equivalences] \label{equivalences}
  We will take contractible maps \cite[definition 4.4.1]{hottbook} as our
  ``default'' definition of equivalences.
  \begin{alignat}{2}
    &\text{isEquiv}(f) &&\coloneqq \Pi(y : B) , \text{isContr}(\text{fib}_f(y)) \label{is-equiv-def} \\
    &A \simeq B        &&\coloneqq \Sigma(f : A \rightarrow B) , \text{isEquiv}(f) \label{equiv-def}
  \end{alignat}
\end{definition}
\begin{definition}[Decidable Types]
  \begin{equation}
    \mathbf{Dec}(A) \coloneqq A \uplus \neg A
  \end{equation}
\end{definition}
\begin{definition}[Discrete Types]
  A discrete type is one with decidable equality.
  \begin{equation}
    \text{Discrete}(A) \coloneqq \Pi(x, y : A) , \mathbf{Dec}(x \equiv y)
  \end{equation}
  By Hedberg's theorem \cite{hedbergCoherenceTheoremMartinLof1998} any discrete
  type is a set.
\end{definition}
\begin{definition}[Higher Inductive Types] \label{HITs}
  Normal inductive types have \emph{point} constructors: constructors which
  construct values of the type.
  Higher Inductive Types (HITs) also have \emph{path} constructors: ways to
  construct paths in the type.
\end{definition}
\begin{definition}[Propositional Truncation] \label{prop-trunc}
  The type \(\lVert A \rVert\) on some type \(A\) is a propositionally truncated
  proof of \(A\) \cite[3.7]{hottbook}.
  In other words, it is a proof that some \(A\) exists, but it does not tell you
  \emph{which} \(A\).

  It is defined as a Higher Inductive Type:
  \begin{equation} {
    \begin{alignedat}{3}
      \lVert A \rVert \coloneqq & \; \lvert \wc \rvert &:& \; A \rightarrow \lVert A \rVert ; \\
                              | & \; \text{squash}     &:& \; \Pi {(x, y : \lVert A \rVert)} , x \equiv y  ; 
    \end{alignedat} }
  \end{equation}
  We will use two eliminators from \(\lVert A \rVert\) in this paper.
  \begin{enumerate}
  \item \label{elim-prop-prop} For any function \(A \rightarrow B\), where
    \(\text{isProp}(B)\), we have a function \(\lVert A \rVert \rightarrow B\).
  \item \label{elim-prop-coh} We can eliminate from \(\lVert A \rVert\) with a
    function \(f : A \rightarrow B\) iff \(f\) ``doesn't care'' about the
    choice of \(A\):
    \[\Pi {(x , y : A)} , f(x) \equiv f(y) \]
    Formally speaking, \(f\) needs to be ``coherently constant''
    \cite{krausGeneralUniversalProperty2015}, and \(B\) needs to be an
    \(n\)-type for some finite \(n\).
  \end{enumerate}
\end{definition}

%%% Local Variables:
%%% mode: latex
%%% TeX-master: "../paper"
%%% End:
\chapter{Finiteness Predicates} \label{finiteness-predicates}
In this section, we will define and briefly describe each of the five predicates
in Figure~\ref{finite-classification}.
As we will see, each of these predicates has subtle differences from the others:
we will outline how some predicates are too informative, and how others aren't
powerful enough, for our needs, before settling on decidable Kuratowski
finiteness as our focus.

As we make our way through each predicate, we will be interested in two aspects:
how can we build proofs of this predicate (i.e. is the product of two finite
types finite?) and what do we \emph{get} once we do (i.e. does this predicate
tell us the number of elements in the finite set?).
\section{Split Enumerability} \label{split-enumerability}
We will start with a simple notion of finiteness, called \emph{split}
enumerability.

\begin{definition}[Split Enumerable Set] \label{split-enum-def}
  To say that some type \(A\) is split enumerable is to say that there is a list
  \(\mathit{support} : \mathbf{List}(A)\) such that any value \(x : A\) is in
  \(\mathit{support}\).
  \begin{agdalisting} \label{split-enum-def-eqn}
    \ExecuteMetaData[agda/Cardinality/Finite/SplitEnumerable/Container.tex]{split-enum-def}
  \end{agdalisting}
  We call the first component of this pair the ``support'' list, and the second
  component the ``cover'' proof.
  An equivalent version of this predicate was called \verb+Listable+ in
  \cite{firsovDependentlyTypedProgramming2015}.
\end{definition}

We have used two types there which we have not yet defined:
\(\AgdaDatatype{List}\) and \(\AgdaDatatype{\ensuremath{\in}}\).
We will define them here.

\begin{definition}[\(\mathbf{List}\)] \label{List}
  In this paper we will work with two equivalent definitions of lists.
  The first is the standard definition as an inductive type:
  \begin{agdalisting}
    \ExecuteMetaData[agda/Data/List/Base.tex]{list-def}
  \end{agdalisting}

  The second way to define lists is to define them as a \emph{container}:
  \begin{agdalisting}
    \ExecuteMetaData[agda/Container/List.tex]{list-def}
  \end{agdalisting}
  The reason we use this second strange definition is that it turns out to be
  quite useful in some later proofs.
  We have proven the two types equivalent in our formalisation, however, so we
  can switch between them freely without loss of generality.
\end{definition}

In defining lists we have introduced another concept which needs defining:
\(\AgdaDatatype{Fin}\).
\begin{definition}[\(\mathbf{Fin}\)] \label{Fin}
  \(\mathbf{Fin}(n)\) is the type of natural numbers smaller than \(n\). We
  define it the standard way:
  \begin{agdalisting}
    \ExecuteMetaData[agda/Data/Fin/Base.tex]{fin-def}
  \end{agdalisting}
\end{definition}
Here \(\uplus\) refers to the disjoint union of two types.
\begin{definition}[Disjoint Union]
  We define disjoint union as an inductive type.

  \begin{agdalisting}
    \ExecuteMetaData[agda/Snippets/Introduction.tex]{disj-union}
  \end{agdalisting}

  It is also expressible with only \(\Sigma\):

  \begin{agdalisting}
    \ExecuteMetaData[agda/Snippets/Introduction.tex]{sigma-disj-union}
  \end{agdalisting}

  Although the inductive type definition is slightly more ergonomic.
\end{definition}

After that interlude, we can get back to defining containers.
\begin{definition}[Containers] \label{container-def}
  A container \cite{abbottContainersConstructingStrictly2005} is a pair
  \(S , P\) where \(S\) is a type, the elements of which are called
  the \emph{shapes} of the container, and \(P\) is a type family on \(S\), where
  the elements of \(P(s)\) are called the \emph{positions} of a container.
  We ``interpret'' a container into a functor defined like so:
  \begin{agdalisting} \label{container-interp}
    \ExecuteMetaData[agda/Container.tex]{container-interp}
  \end{agdalisting}
  Membership of a container can be defined like so:
  \begin{agdalisting} \label{container-membership}
    \ExecuteMetaData[agda/Container/Membership.tex]{membership-def}
  \end{agdalisting}
\end{definition}

\begin{definition}[Fibers] \label{fibers}
  A fiber \cite[definition 4.2.4]{hottbook} is defined over some function \(f :
  A \rightarrow B\).
  \begin{agdalisting}
    \ExecuteMetaData[agda/Snippets/Introduction.tex]{fiber}
  \end{agdalisting}
\end{definition}

\subsection{Instances}
Now that we have a suitable definition of finiteness, we will next prove that
some things are finite.
\begin{lemma}
  \(\bot\), \(\top\), and \(\AgdaDatatype{Bool}\) are split enumerable.
\end{lemma}
\begin{proof}
  These three types are quite obviously finite: we will show only the proof of
  finiteness for \(\AgdaDatatype{Bool}\) here for brevity's sake.
  \begin{agdalisting}
    \ExecuteMetaData[agda/Cardinality/Finite/SplitEnumerable.tex]{bool-inst}
  \end{agdalisting}
\end{proof}
With the most basic simple types out of the way, the obvious next choice is the
(non-dependent) sums and products: \(\uplus\) and \(\times\).
Both of these types can be constructed from the \emph{dependent} sum, however,
so that is the type we will prove finite.
From that we can derive a much wider array of finiteness proofs.

\begin{lemma} \label{split-enum-sigma}
  Split enumerability is closed under \(\Sigma\).
  \begin{equation}
    \frac{
      \AgdaDatatype{\ensuremath{\mathcal{E}!}}(A) \; \; \; \Pi(x : A) , \AgdaDatatype{\ensuremath{\mathcal{E}!}}(U(x))
    }{
      \AgdaDatatype{\ensuremath{\mathcal{E}!}}(\Sigma(x : A) , U(x))
    }
  \end{equation}
\end{lemma}
\begin{proof}
  Let \(A\) be a type which is split enumerable, and \(U\) be a type family over
  \(A\) which is split enumerable at every point.
  Formally, we have the following proofs:
  \begin{align}
    \AgdaDatatype{\ensuremath{\mathcal{E}!}}_A &: \AgdaDatatype{\ensuremath{\mathcal{E}!}}(A) \\
    \AgdaDatatype{\ensuremath{\mathcal{E}!}}_U &: \Pi(x : A) , \AgdaDatatype{\ensuremath{\mathcal{E}!}}(U(x))
  \end{align}

  Our task is to construct a proof of type:
  \begin{equation}
    \AgdaDatatype{\ensuremath{\mathcal{E}!}}(\Sigma(x : A) , U(x))
  \end{equation}
  This proof itself is composed of two components:
  \begin{align}
    \mathit{support} &: \mathbf{List}(\Sigma(x : A) , U(x)) \\
    \mathit{cover} &: \Pi(x : \Sigma(y : A) , U(y)) , x \in \mathit{support}
  \end{align}
  To construct the support list, we apply the function \(\AgdaDatatype{\ensuremath{\mathcal{E}!}}_U\) to
  every element in the support list of \(\AgdaDatatype{\ensuremath{\mathcal{E}!}}_A\), extract the support
  lists from the resulting finiteness proofs, and concatenate them.


  To prove that this support list does in fact cover the entirety of the type
  \(\Sigma \; A \; U\), we note that any element of type \(\Sigma \; A \; U\)
  must have a first component in the support list of \(\AgdaDatatype{\ensuremath{\mathcal{E}!}}_A\), and its
  second component must be in the result of applying \(\AgdaDatatype{\ensuremath{\mathcal{E}!}}_U\) to that
  first element (since that support list contains every element of type
  \(U(x)\)).
  Therefore, the pair itself must be in our constructed support list.
\end{proof}

\begin{agdalisting}
  This pattern of applying a function to each element in a list and
  concatenating the result is of course well-known in functional programming,
  and is in fact the pattern that makes lists a monad.
  While this insight isn't strictly relevant to our work here, it does mean
  the implementation of this function can use Agda's do notation, resulting
  in the following extremely clean implementation:
  \ExecuteMetaData[agda/Cardinality/Finite/SplitEnumerable.tex]{sup-sigma}
\end{agdalisting}

\subsection{Derivations}
We have a way to construct finiteness proofs, and a semiring-like toolbox to
combine them.
What we're now interested in is what we can \emph{derive} from them.

First, we will look at how this predicate relates to more traditional, classical
notions of finiteness.
in a classical setting we likely wouldn't mention ``lists'' or the like, and
would instead define finiteness based on the existence of some injection or
surjection.
As it turns out, our definition of finiteness here is precisely the same as the
surjection-based one, in quite a deep way!

First, we will need to define our terms: in HoTT, surjections are a little more
complex than what you'd find in either MLTT or classical mathematics.
\begin{definition}[Split Surjections] \label{split-surjections}
  We define \emph{split} surjections here \cite[definition
  4.6.1]{hottbook}.
  \begin{alignat}{3}
    &\text{sp-surj}(f)          &&\coloneqq \Pi(y : B) , \text{fib}_f(y) \label{sp-surj-eqn} \\
    &A \twoheadrightarrow! \; B &&\coloneqq \Sigma (f : A \rightarrow B) , \text{sp-surj}(f) \label{sp-surj-arrow-eqn}
  \end{alignat}
\end{definition}
Over sets, the surjections and split surjections are the same thing, but there
is a difference one we involve non-set types like the circle.

We will now see that split enumerability is in fact a split surjection in
another form:
\begin{lemma} \label{split-enum-is-split-surj}%
  A proof of split enumerability is equivalent to a split surjection from a
  finite prefix of the natural numbers.
  \begin{equation}%
    \AgdaDatatype{\ensuremath{\mathcal{E}!}}(A) \simeq \Sigma (n : \mathbb{N}) , \left( \mathbf{Fin}(n) \twoheadrightarrow ! \; A \right)
  \end{equation}
\end{lemma}
\begin{proof} \phantom{x}

  \begin{minipage}[t]{.8\textwidth}\vspace{-\baselineskip}
    \begin{agdalisting*}
      \ExecuteMetaData[agda/Cardinality/Finite/SplitEnumerable.tex]{is-split-inj}
    \end{agdalisting*}
  \end{minipage}
  \begin{minipage}[t]{.19\textwidth} \setstretch{1.08}

      def.~\ref{split-enum-def} (\AgdaDatatype{\ensuremath{\mathcal{E}!}}) \\
      eqn.~\ref{container-membership} (\AgdaDatatype{\ensuremath{\in}}) \\
      eqn.~\ref{sp-surj-eqn}  \\
      def.~\ref{List} (\(\AgdaDatatype{List}\)) \\
      eqn.~\ref{container-interp}  \\
      Reassociation \\
      eqn.~\ref{sp-surj-arrow-eqn} \qedhere
  \end{minipage}
\end{proof}
In the above proof syntax the
\(\AgdaDatatype{\ensuremath{\equiv \langle{} \rangle{} }}\) connects lines which
are definitionally equal, i.e. they are ``obviously'' equal from the type
checker's perspective.
Clearly, only one line isn't a definitional equality: 
\begin{agdalisting}
  \ExecuteMetaData[agda/Data/Sigma/Properties.tex]{reassoc}
\end{agdalisting}
(The simplicity of this proof, by the way, is why we preferred the
container-based definition of lists over the traditional one.)

Split enumerability implies decidable equality on the underlying type.
To prove this, we will make use of the following lemma, proven in the
formalisation:
\begin{definition}[Injections]
  Injective functions are more straightforward to define constructively than
  surjective ones:
  \begin{alignat}{3}
    & \text{injective}(f) &&\coloneqq \Pi (x, y : A) , f \; x \equiv f \; y \rightarrow x \equiv y \\
    & A \rightarrowtail B &&\coloneqq \Sigma(f : A \rightarrow B) , \text{injective}(f)
  \end{alignat}
\end{definition}
\begin{lemma} \label{split-surj-to-inj}
  A split-surjection from \(A\) to \(B\) implies an injection from \(B\) to
  \(A\).
  \begin{equation}
    (A \twoheadrightarrow! \; B) \rightarrow (B \rightarrowtail A)
  \end{equation}
\end{lemma}
\begin{lemma} \label{inj-discrete}
  For any type \(A\) which injects into a discrete type \(B\), \(A\) is
  discrete.
  \begin{equation}
    \frac{
      A \rightarrowtail B \; \; \; \text{Discrete}(B)
    }{
      \text{Discrete}(A)
    }
  \end{equation}
\end{lemma}

\begin{definition}[Decidable Types]
  \begin{equation}
    \mathbf{Dec}(A) \coloneqq A \uplus \neg A
  \end{equation}
\end{definition}
\begin{definition}[Discrete Types]
  A discrete type is one with decidable equality.
  \begin{equation}
    \text{Discrete}(A) \coloneqq \Pi(x, y : A) , \mathbf{Dec}(x \equiv y)
  \end{equation}
  By Hedberg's theorem \cite{hedbergCoherenceTheoremMartinLof1998} any discrete
  type is a set.
\end{definition}

\begin{lemma} \label{discrete-surj}
  \begin{equation}
    \frac{
      A \twoheadrightarrow! \; B \; \; \; \text{Discrete}(A)
    }{
      \text{Discrete}(B) 
    }
  \end{equation}
\end{lemma}
\begin{proof}
  This proof is can be straightforwardly derived from lemmas
  \ref{split-surj-to-inj} and \ref{inj-discrete}.
\end{proof}

\begin{lemma} \label{split-enum-discrete}
  Every split enumerable type is discrete.
\end{lemma}
\begin{proof}
  Let \(A\) be a split enumerable type.
  By lemma~\ref{split-enum-is-split-surj}, there is a surjection from
  \(\mathbf{Fin}(n)\) for some \(n\).
  Also, we know that \(\mathbf{Fin}(n)\) is discrete (proven in our
  formalisation).
  Therefore, by lemma~\ref{discrete-surj}, \(A\) is discrete.
\end{proof}

\section{Manifest Bishop Finiteness} \label{manifest-bishop-finiteness}
We mentioned in the introduction that occasionally in constructive mathematics
proofs will contain ``too much'' information.
With split enumerability we can see an instance of this.
Consider the following proof of the finiteness of Bool:
\begin{agdalisting} \label{bool-slop}
  \ExecuteMetaData[agda/Cardinality/Finite/SplitEnumerable.tex]{bool-slop}
\end{agdalisting}
While it represents the ``same'' information as our previous proof, it is
clearly not the same \emph{object}.

There is ``slop'' in the type of split enumerability: there are more distinct
values than there are \emph{usefully} distinct values.
As we can see in the example above, for instance, split enumerability allows
duplicate values in the support list.
To reconcile this, we will disallow duplicates in the support list.

How exactly we should do this is the next question.
One approach might be to change the definition of \(\mathbf{List}\), or
introduce a new type \(\mathbf{NoDupeList}\), and use it in the predicate
instead.
However, this would mean we lose access to the functions we have defined on
lists, and we have to change the definition of \(\in\) as well.

There is a much simpler and more elegant solution: we insist that every
\emph{membership proof} must be unique.
This would disallow a definition of \(\AgdaDatatype{\ensuremath{\mathcal{E}!}}\;
\AgdaDatatype{Bool}\) with
duplicates, as there are multiple values which inhabit the type \(\text{false}
\in \left[ \text{false}, \text{true}, \text{false} \right]\).
It also allows us to keep most of the split enumerability definition unchanged,
just adding a condition to the returned membership proof in the cover proof.

To specify that a value must exist uniquely in HoTT we can use the concept of a
\emph{contraction}.

\begin{definition}[Homotopy Levels] \label{homotopy-types}
  Types in HoTT and CuTT are not necessarily sets, as they are in MLTT.
  Some have higher homotopies (paths which aren't unique).
  We actually have a hierarchy of complexity of structure of path spaces in
  types, starting with the contractions \cite[definition 3.11.1]{hottbook}, then
  the mere propositions \cite[definition 3.3.1]{hottbook}, and the sets
  \cite[definition 3.1.1]{hottbook}.
  \begin{agdalisting}
    \ExecuteMetaData[agda/Snippets/Introduction.tex]{hlevels}
  \end{agdalisting}
\end{definition}
\begin{definition}[Unique Membership] \label{uniq-memb-def}
  Unique list membership is defined in terms of list membership: it is a
  contraction of it.
  \begin{agdalisting}
    \ExecuteMetaData[agda/Container/Membership.tex]{uniq-memb-def}
  \end{agdalisting}
\end{definition}

With this we can define manifest Bishop finiteness:
\begin{definition}[Manifest Bishop Finiteness]  \label{bish-def}
  A type is manifest Bishop finite if there exists a list which contains each
  value in the type once.
  \begin{agdalisting}
    \ExecuteMetaData[agda/Cardinality/Finite/ManifestBishop/Container.tex]{bish-def}
  \end{agdalisting}
\end{definition}
The only difference between manifest Bishop finiteness and split enumerability
is the membership term: here we require unique membership (\(\inunique\)),
rather than simple membership (\(\in\)).

We use the word ``manifest'' here to distinguish from another common
interpretation of Bishop finiteness, which we have called cardinal finiteness in
this paper.
The ``manifest'' refers to the fact that we have a concrete, non-truncated list
of the elements in the proof.

\subsection{The Relationship Between Manifest Bishop Finiteness and Split
  Enumerability}
While manifest Bishop finiteness might seem stronger than split enumerability,
it turns out this is not the case.
Both predicates imply the other.
\begin{lemma} \label{manifest-bishop-to-split-enum}
  Any manifest Bishop finite type is split enumerable.
\end{lemma}
\begin{proof}
  To construct a proof of split enumerability from one of manifest Bishop
  finiteness, it suffices to convert a proof of \(x \inunique \mathit{xs}\) to
  one of \(x \in \mathit{xs}\), for all \(x\) and \(\mathit{xs}\).
  Since \(\inunique\) is defined as a contraction of \(\in\), such a conversion
  is simply the \(\text{fst}\) function.
\end{proof}

\begin{lemma} \label{split-enum-to-manifest-bishop}
  Any split enumerable set is manifest Bishop finite.
\end{lemma}
This proof takes significantly more work.
The ``unique membership'' condition in
\(\AgdaDatatype{\ensuremath{\mathcal{B}}}\) means that we are not permitted
duplicates in the support list.
The first step in the proof, then, is to filter those duplicates out from the
support list of the \(\AgdaDatatype{\ensuremath{\mathcal{E}!}}\) proof: we can do this using the decidable
equality provided by \(\AgdaDatatype{\ensuremath{\mathcal{E}!}}\) (lemma~\ref{split-enum-discrete}).
From there, we need to show that the membership proof carries over
appropriately.
\todo{Provide more info on this proof?}

We have now proved that every manifestly Bishop finite type is split enumerable,
and vice versa.
While the types are not \emph{equivalent} (there are more split enumerable
proofs than there are manifest Bishop finite proofs), they are of equal power,
so any closure proof we have on one can be transferred to the other.
In particular, it means that manifest Bishop finiteness is closed under
\(\Sigma\).
\subsection{From Manifest Bishop Finiteness to Equivalence}
We have seen that split enumerability was in fact a split-surjection in
disguise.
We will now see that manifest Bishop finiteness is in fact an \emph{equivalence}
in disguise.
\emph{equivalence} with \(\mathbf{Fin}\).

\begin{definition}[Equivalences] \label{equivalences}
  We will take contractible maps \cite[definition 4.4.1]{hottbook} as our
  ``default'' definition of equivalences.
  \begin{agdalisting} \label{is-equiv-def}
    \ExecuteMetaData[agda/Snippets/Equivalence.tex]{is-equiv-def}
  \end{agdalisting} \vspace{-.5\baselineskip}
  \begin{agdalisting} \label{equiv-def}
    \ExecuteMetaData[agda/Snippets/Equivalence.tex]{equiv-def}
  \end{agdalisting}
\end{definition}
\begin{lemma} \label{bishop-equiv}
  Manifest bishop finiteness is equivalent to an equivalence to a finite prefix
  of the natural numbers.
  \begin{equation}
  \end{equation}
\end{lemma}
\begin{proof}
  \begin{align*}
     \AgdaDatatype{\ensuremath{\mathcal{B}}}(A) &
    \simeq \Sigma (\mathit{xs} : \textbf{List}(A)) , \Pi {(x : A)} , x \inunique \mathit{xs}
    && \text{def.~\ref{bish-def} }(\AgdaDatatype{\ensuremath{\mathcal{B}}})
    \\
    & \simeq \Sigma (\mathit{xs} : \textbf{List}(A)) , \Pi {(x : A)} , \text{isContr}(x \in \mathit{xs})
    && \text{eqn.~\ref{uniq-memb-def} } (\inunique)
    \\
    & \simeq \Sigma (\mathit{xs} : \textbf{List}(A)) , \Pi {(x : A)} , \text{isContr}(\text{fib}_{\text{snd}(\mathit{xs})}(x))
    && \text{eqn.~\ref{container-membership} } (\in)
    \\
    & \simeq \Sigma (\mathit{xs} : \textbf{List}(A)) , \text{isEquiv}(\text{snd}(\mathit{xs}))
    && \text{eqn.~\ref{is-equiv-def} (isEquiv)}
    \\
    & \simeq \Sigma (\mathit{xs} : \llbracket \mathbb{N} , \mathbf{Fin} \rrbracket (A)) , \text{isEquiv}(\text{snd}(\mathit{xs}))
    && \text{def.~\ref{List} } (\mathbf{List})
    \\
    & \simeq \Sigma (\mathit{xs} : \Sigma (n : \mathbb{N}) , \Pi (i : \mathbf{Fin}(n)) , A) , \text{isEquiv}(\text{snd}(\mathit{xs}))
    && \text{eqn.~\ref{container-interp} } (\llbracket \wc \rrbracket)
    \\
    & \simeq \Sigma (n : \mathbb{N}) , \Sigma (f : \mathbf{Fin}(n) \rightarrow A) , \text{isEquiv}(f)
    && \text{Reassociation of } \Sigma
    \\
    & \simeq \Sigma (n : \mathbb{N}) , ( \mathbf{Fin}(n) \simeq A )
    && \text{eqn.~\ref{equiv-def} } (\simeq) \; \qedhere
  \end{align*}
\end{proof}
This proof is almost identical\footnotemark to the proof for
lemma~\ref{split-enum-is-split-surj}: it reveals that
enumeration-based finiteness predicates are simply another perspective on
relation-based ones.
\footnotetext{
  Unfortunately in our formalisation this proof cannot be a single line: for
  performance reasons \(\simeq\) is defined as a record type with eta-equality
  disabled, instead of the definition here which uses \(\Sigma\).
}

As we are working in CuTT, a proof of equivalence between two types gives us the
ability to \emph{transport} proofs from one type to the other.
This is extremely powerful, as we will see.
\subsection{Closure Under \(\Pi\)}
The glaring omission from our closure proofs under type formers so far has been
the \(\Pi\) type: we have not proved closure under functions, dependent or
otherwise.
In MLTT, this is of course not provable: since all of the finiteness predicates
we have seen so far imply decidable equality, and since we don't have any kind
of decidable equality on functions in MLTT, we know that we won't be able to
show that any kind of function is finite; even one like \(\AgdaDatatype{Bool}
\rightarrow \AgdaDatatype{Bool}\).

CuTT is not so restricted.
Since we have things like function extensionality and transport, we can indeed
prove the finiteness of function types.
Our proof here makes use directly of the univalence axiom, and makes use
furthermore of all the previous closure proofs.
We will prove this closure on split enumerability, rather than on manifest
Bishop finiteness, as it requires slightly less legwork in the proof itself, but
of course we can derive the proof on manifest Bishop finiteness in a few lines.
\begin{theorem} \label{split-enum-pi}
  Split enumerability is closed under dependent functions.
  (\(\Pi\)-types).
  \begin{equation}
    \frac{
      \AgdaDatatype{\ensuremath{\mathcal{E}!}}(A) \; \; \; \Pi {(x : A)} , \AgdaDatatype{\ensuremath{\mathcal{E}!}}\left( U(x) \right)
    }{
      \AgdaDatatype{\ensuremath{\mathcal{E}!}}\left(\Pi {(x : A)} , U(x)\right)
    }
  \end{equation}
\end{theorem}
\begin{proof}
  Let \(A\) be a split enumerable type, and \(U\) be a type family from \(A\),
  which is split enumerable over all points of \(A\).

  As \(A\) is split enumerable, we know that it is also manifestly Bishop finite
  (lemma~\ref{split-enum-to-manifest-bishop}), and consequently we know \(A
  \simeq \mathbf{Fin}(n)\), for some \(n\) (lemma~\ref{bishop-equiv}).
  We can therefore replace all occurrences of \(A\) with \(\mathbf{Fin}(n)\),
  changing our goal to:
  \begin{equation}
    \frac{
      \AgdaDatatype{\ensuremath{\mathcal{E}!}}(\mathbf{Fin}(n)) \; \; \; \Pi (x : \mathbf{Fin}(n)) , \AgdaDatatype{\ensuremath{\mathcal{E}!}}\left( U(x) \right)
    }{
      \AgdaDatatype{\ensuremath{\mathcal{E}!}}\left(\Pi (x : \mathbf{Fin}(n)) , U(x)\right)
    }
  \end{equation}
  
  We then define the type of \(n\)-tuples over some type family \(T :
  \mathbf{Fin}(n) \rightarrow \mathbf{Type}\).
  \begin{equation}
    \begin{alignedat}{3}
      & \mathbf{Tuple}(0, T)   &&\coloneqq \top \\
      & \mathbf{Tuple}(n+1, T) &&\coloneqq T(0) \times \mathbf{Tuple}(n, T \circ \text{suc})
    \end{alignedat}
  \end{equation}
  We can show that this type is equivalent to functions (proven in our formalisation):
  \begin{equation}
    \Pi(x : \mathbf{Fin}(n)) , U(x) \simeq \mathbf{Tuple}(n, U)
  \end{equation}
  And therefore we can simplify again our goal to the following:
  \begin{equation}
    \frac{
      \AgdaDatatype{\ensuremath{\mathcal{E}!}}(\mathbf{Fin}(n)) \; \; \; \Pi (x : \mathbf{Fin}(n)) , \AgdaDatatype{\ensuremath{\mathcal{E}!}}\left( U(x) \right)
    }{
      \AgdaDatatype{\ensuremath{\mathcal{E}!}}\left(\mathbf{Tuple}(n, U)\right)
    }
  \end{equation}
  
  We can prove this goal by showing that \(\mathbf{Tuple}(n, U)\) is split
  enumerable: it is made up of finitely many products of points of \(U\), which
  are themselves split enumerable, and \(\top\), which is also split enumerable.
  Lemma~\ref{split-enum-sigma} shows us that the product of finitely many split
  enumerable types is itself split enumerable, proving our goal.
\end{proof}
\section{Cardinal Finiteness} \label{cardinal-finiteness}
While we have removed some of the unnecessary information from our finiteness
predicates, one piece still remains.
\begin{agdalisting}
  The two following proofs are both valid proofs of the finiteness of
  \(\AgdaDatatype{Bool}\), and both do not include any duplicates.
  However they still differ:
  \ExecuteMetaData[agda/Cardinality/Finite/SplitEnumerable.tex]{bool-inst} \smallskip
  \ExecuteMetaData[agda/Cardinality/Finite/SplitEnumerable.tex]{bool-rev}
\end{agdalisting}
Each finiteness predicate so far has contained an \emph{ordering} of the
underlying type.
For our purposes, this is too much information: it means that when constructing
the ``category of finite sets'' later on, instead of each type having one
canonical representative, it will have \(n!\), where \(n\) is the cardinality of
the type\footnotemark.

\footnotetext{
  We actually do get a category (a groupoid, even) from manifest Bishop
  finiteness \cite{yorgeyCombinatorialSpeciesLabelled2014}: it's the groupoid of
  finite sets equipped with a linear order, whose morphisms are order-preserving
  bijections.
  We do not explore this particular construction in any detail.
}

To remedy the problem, we will use propositional truncation
(def.~\ref{prop-trunc}).

\begin{definition}[Higher Inductive Types] \label{HITs}
  Normal inductive types have \emph{point} constructors: constructors which
  construct values of the type.
  Higher Inductive Types (HITs) also have \emph{path} constructors: ways to
  construct paths in the type.
\end{definition}
\begin{definition}[Propositional Truncation] \label{prop-trunc}
  The type \(\lVert A \rVert\) on some type \(A\) is a propositionally truncated
  proof of \(A\) \cite[3.7]{hottbook}.
  In other words, it is a proof that some \(A\) exists, but it does not tell you
  \emph{which} \(A\).

  It is defined as a Higher Inductive Type:
  \begin{equation} {
    \begin{alignedat}{3}
      \lVert A \rVert \coloneqq & \; \lvert \wc \rvert &:& \; A \rightarrow \lVert A \rVert ; \\
                              | & \; \text{squash}     &:& \; \Pi {(x, y : \lVert A \rVert)} , x \equiv y  ; 
    \end{alignedat} }
  \end{equation}
  We will use two eliminators from \(\lVert A \rVert\) in this paper.
  \begin{enumerate}
  \item \label{elim-prop-prop} For any function \(A \rightarrow B\), where
    \(\text{isProp}(B)\), we have a function \(\lVert A \rVert \rightarrow B\).
  \item \label{elim-prop-coh} We can eliminate from \(\lVert A \rVert\) with a
    function \(f : A \rightarrow B\) iff \(f\) ``doesn't care'' about the
    choice of \(A\):
    \[\Pi {(x , y : A)} , f(x) \equiv f(y) \]
    Formally speaking, \(f\) needs to be ``coherently constant''
    \cite{krausGeneralUniversalProperty2015}, and \(B\) needs to be an
    \(n\)-type for some finite \(n\).
  \end{enumerate}
\end{definition}
\begin{definition}[Cardinal Finiteness]
  A type \(A\) is cardinally finite if there exists a propositionally truncated
  proof that \(A\) is manifest Bishop finite or equivalent to a finite prefix of
  the natural numbers.
  \begin{equation}
    \mathcal{C}(A) \coloneqq \lVert \AgdaDatatype{\ensuremath{\mathcal{B}}}(A) \rVert \simeq \lVert \Sigma(n : \mathbb{N}) , (\mathbf{Fin}(n) \simeq A) \rVert
  \end{equation}
\end{definition}
At first glance, it might seem that we lose any useful properties we could
derive from \(\AgdaDatatype{\ensuremath{\mathcal{B}}}\).
Luckily, this is not the case: by eliminator \ref{elim-prop-coh} of
def.~\ref{prop-trunc}, we need only show that the output is uniquely determined.
\subsection{Deriving Uniquely-Determined Quantities}
The following two lemmas are proven in
\cite{yorgeyCombinatorialSpeciesLabelled2014} (Proposition 2.4.9 and 2.4.10,
respectively), in much the same way as we have done here.
Our contribution for this section is simply the formalisation.
\begin{lemma}
  Given a cardinally finite type, we can derive the type's cardinality, as well
  as a propositionally truncated proof of equivalence with \(\textbf{Fin}\)s of
  the same cardinality.
  \begin{equation}
    \mathcal{C}(A) \rightarrow \Sigma {(n : \mathbb{N})} , \lVert \textbf{Fin}(n) \simeq A \rVert
  \end{equation}
\end{lemma}
\begin{proof}
  Let \(A\) be a cardinally-finite type, with proof \(F : \mathcal{C}(A)\).
  Our task is to extract a natural number \(n : \mathbb{N}\) representing the
  cardinality of \(A\), and a propositionally-truncated proof that \(A\) is
  equivalent to \(\mathbf{Fin}(n)\).

  Extracting the second component of the pair is trivial, as it itself is
  truncated.
  We will now focus on extracting the cardinality.

  Without the propositional truncation, \(\text{fst}\) would suffice for this
  task.
  Given that the pair is hidden under the truncation, then, we need a way to
  convert a function \(f : A \rightarrow B\) to \(g : \lVert A \rVert
  \rightarrow B\).
  This is precisely what eliminator \ref{elim-prop-coh} gives us.
  For our case, we need to show the following:
  \begin{equation}
    \frac{(n : \mathbb{N}) \; \; \; (p : \mathbf{Fin}(n) \simeq A) \; \; \;
          (m : \mathbb{N}) \; \; \; (q : \mathbf{Fin}(m) \simeq A)
        }{
          n \equiv m
        }
  \end{equation}
  Immediately we can construct the following term:
  \begin{equation}
    \begin{alignedat}{3}
      \mathbf{Fin}(n) & \simeq A && (p) \\
                      & \simeq \mathbf{Fin}(m) && (q)
    \end{alignedat}
  \end{equation}
  Given univalence we have \(\mathbf{Fin}(n) \equiv \mathbf{Fin}(m)\),
  and the rest of our task is to prove:
  \begin{equation}
    \frac{\mathbf{Fin}(n) \equiv \mathbf{Fin}(m)}{n \equiv m}
  \end{equation}

  This is a well-known chestnut in dependently-typed programming, and one that
  has a surprisingly tricky and complex proof.
  We do not include it here, since it has already been explored elsewhere, but
  it is present in our formalisation.
\end{proof}

In order to prove that cardinal finiteness implies decidable equality, we will
need to show that decidable equality itself is a proposition.
In doing that we will use the following lemma:
\begin{lemma} \label{prop-refute}
  We can ``refute'' a propositionally-truncated proof of some proposition with a
  proof that the non-truncated proposition is false.
  \begin{equation}
    \frac{\neg A \; \; \; \lVert A \rVert}{\bot}
  \end{equation}
\end{lemma}
\begin{proof}
  We know we can eliminate from any value of type \(\lVert A \rVert\) into some
  \(B\) with a function \(A \to B\) if \(B\) is a proposition.
  That's precisely what we do in this case: \(\neg A\) is a function of type \(A
  \to \bot\), and we know that \(\bot\) is a proposition.
\end{proof}

\begin{lemma} \label{cardinal-finite-discrete}
  Any cardinal-finite set has decidable equality.
\end{lemma}
\begin{proof}
  Since we can already derive decidable equality from a proof of manifest Bishop
  finiteness, it suffices to show that decidable equality is itself a
  proposition.
  \begin{equation}
    \text{isProp}(\Pi(x, y : A) , \mathbf{Dec}(x \equiv y))
  \end{equation}
  First, it is clear that \(x \equiv y\) is a proposition: since the type \(A\)
  has decidable equality, by Hedburg's theorem it is a set,
  meaning precisely that \(x \equiv y\) is a proposition.

  Secondly, we know that any decision over a proposition is itself a
  proposition.
  For any two terms \(x, y: \mathbf{Dec}(A)\) we cannot have the case that one
  is a yes decision and the other is no: from that we could derive \(\bot\).
  If both are no then they are both equal since \(A \rightarrow \bot\) is a
  proposition through function extensionality.
  And finally if both are yes then we know they must be equal because the type
  decided over is itself a proposition.

  Finally, since we know that \(\mathbf{Dec}(x \equiv y)\) is a proposition, we
  can derive that \(\Pi(x, y : A) , \mathbf{Dec}(x \equiv y)\) is a proposition
  (through function extensionality), proving our goal.
\end{proof}
\subsection{Restrictiveness}
So far our explorations into finiteness predicates have pushed us in the
direction of ``less informative'': however, as mentioned in the introduction, we
can \emph{also} ask how \emph{restrictive} certain predicates are.
Since split enumerability and manifest Bishop finiteness imply each other we
know that there can be no type which satisfies one but not the other.
We also know that manifest Bishop finiteness implies cardinal finiteness, but we
do \emph{not} have a function in the other direction:
\begin{equation}
  \mathcal{C}(A) \rightarrow \AgdaDatatype{\ensuremath{\mathcal{B}}}(A)
\end{equation}
So the question arises naturally: is there a cardinally finite type which is
\emph{not} manifest Bishop finite?

It turns out the answer is no!
\begin{lemma}
  \begin{equation}
    \neg (\Sigma(A : \mathbf{Type}) , \mathcal{C}(A) \times \neg \AgdaDatatype{\ensuremath{\mathcal{B}}}(A))
  \end{equation}
\end{lemma}
\begin{proof}
  We will actually prove a slightly more general statement.
  For any type \(A\), the following holds:
  \begin{equation}
    \neg (\lVert A \rVert \times \neg A)
  \end{equation}
  The solution becomes more clear if we write out the definition of \(\neg\):
  \begin{equation}
    \frac{\lVert A \rVert \;\;\; A \rightarrow \bot }{\bot}
  \end{equation}
  We clearly need to apply a function of type \(A \rightarrow \bot\) to a value
  of type \(\lVert A \rVert\).
  Luckily, this is permissible, as \(\bot\) is a mere proposition.
\end{proof}

\subsection{Going from Cardinal Finiteness to Manifest Bishop Finiteness}
\begin{lemma} \label{manifest-bishop-to-cardinal}
  Any manifest Bishop finite type is cardinal finite.
\end{lemma}
\begin{theorem} \label{cardinal-to-manifest-bishop}
  Any cardinal finite type with a total order is Bishop finite.
\end{theorem}
The proof for this particular theorem is quite involved in the formalisation, so
we only give its sketch here.
First, note that we actually convert to manifest enumerability first: this can
be converted to split enumerability with decidable equality, which is provided
by cardinal finiteness.

Next, we define permutations.
\begin{definition}[List Permutations]
  Two lists are permutations of each other if their membership proofs are all
  equivalent\footnotemark \cite{danielssonBagEquivalenceProofRelevant2012}.
  \begin{equation}
    \mathit{xs} \leftrightsquigarrow \mathit{ys} = \Pi {(x : A)} , x \in \mathit{xs} \simeq x \in \mathit{ys}
  \end{equation}
\end{definition}

\footnotetext{
  The definition in \cite{danielssonBagEquivalenceProofRelevant2012} and our
  formalisation is slightly different: we say permutations are lists with
  \emph{isomorphic} membership proofs.
  The distinction, as it happens, does not affect our work here.
}

Next, we define a sort function which relies on the provided total order.
We further prove the following fact about this sort function:
\begin{equation}
  \Pi(\mathit{xs}, \mathit{ys} : \mathbf{List}(A)) , \mathit{xs} \leftrightsquigarrow \mathit{ys} \rightarrow \text{sort}(\mathit{xs}) \equiv \text{sort}(\mathit{ys})
\end{equation}

Next, notice that the support lists of any two proofs of manifest Bishop
finiteness must be permutations of each other.
This will allow us to sort the support list of a proof of cardinal finiteness in
a coherently constant (definition~\ref{prop-trunc},
eliminator~\ref{elim-prop-coh}) way, pulling the support list out from the
truncation.
The cover proof emerges naturally from the definition of the permutation.
\subsection{Closure}
Since we don't have a function of type \(\mathcal{C}(A) \rightarrow
\AgdaDatatype{\ensuremath{\mathcal{B}}}(A)\), closure proofs on \(\AgdaDatatype{\ensuremath{\mathcal{B}}}\) do not transfer over to
\(\mathcal{C}\) trivially (unlike with \(\AgdaDatatype{\ensuremath{\mathcal{E}!}}\) and \(\AgdaDatatype{\ensuremath{\mathcal{B}}}\)).
The cases for \(\bot\), \(\top\), and \(\AgdaDatatype{Bool}\) are simple to adapt: we
can just propositionally truncate their Bishop finiteness proof.

Non-dependent operators like \(\times\), \(\uplus\), and \(\rightarrow\) are
also relatively straightforward: since \(\lVert {\wc} \rVert\) forms a monad, we
can apply \(n\)-ary functions to values inside it, combining them together.
\begin{agdalisting}
  The fact that \(\lVert \wc \rVert\) forms a monad means that we can lift
  \(n\)-ary functions like the following:
  \ExecuteMetaData[agda/Cardinality/Finite/ManifestBishop.tex]{times-clos-sig}
  Into a truncated context:
  \ExecuteMetaData[agda/Cardinality/Finite/Cardinal.tex]{times-clos-impl}
\end{agdalisting}

Unfortunately, for the dependent type formers like \(\Sigma\) and \(\Pi\), the
same trick does not work.
We have closure proofs like:
\begin{equation}
  \frac{
    \AgdaDatatype{\ensuremath{\mathcal{B}}}(A) \; \; \; \Pi(x : A) , \AgdaDatatype{\ensuremath{\mathcal{B}}}(U(x))
  }{
    \AgdaDatatype{\ensuremath{\mathcal{B}}}(\Pi \; A \; U)
  }
\end{equation}
If we apply the monadic truncation trick we can derive closure proofs like the
following:
\begin{equation}
  \frac{
    \lVert \AgdaDatatype{\ensuremath{\mathcal{B}}}(A) \rVert \; \; \; \lVert \Pi(x : A) , \AgdaDatatype{\ensuremath{\mathcal{B}}}(U(x)) \rVert
  }{
    \lVert \AgdaDatatype{\ensuremath{\mathcal{B}}}(\Pi \; A \; U) \rVert
  }
\end{equation}
However our \emph{desired} closure proof is the following:
\begin{equation}
  \frac{
    \lVert \AgdaDatatype{\ensuremath{\mathcal{B}}}(A) \rVert \; \; \; \Pi(x : A) , \lVert \AgdaDatatype{\ensuremath{\mathcal{B}}}(U(x)) \rVert
  }{
    \lVert \AgdaDatatype{\ensuremath{\mathcal{B}}}(\Pi \; A \; U) \rVert
  }
\end{equation}
They don't match!

The solution would be to find a function of the following type:
\begin{equation}
  (\Pi(x : A) , \lVert \AgdaDatatype{\ensuremath{\mathcal{B}}}(U(x)) \rVert) \rightarrow
  \lVert \Pi(x : A) , \AgdaDatatype{\ensuremath{\mathcal{B}}}(U(x)) \rVert
\end{equation}
However we might be disheartened at realising that this is a required goal: the
above equation is \emph{extremely} similar to the axiom of choice!
\begin{definition}[Axiom of Choice]
  In HoTT, the axiom of choice is commonly defined as follows \cite[lemma
  3.8.2]{hottbook}.
  For any set \(A\), and a type family \(U\) which is a set at all the points
  of \(A\), the following function exists:
  \begin{equation}
    \left( \Pi(x : A) ,  \lVert U(x) \rVert \right) \rightarrow \lVert \Pi(x : A) , U(x) \rVert
  \end{equation}
\end{definition}
Luckily the axiom of choice \emph{does} hold for cardinally finite types,
allowing us to prove the following:
\begin{lemma}
  \begin{equation}
    \mathcal{C}(A) \rightarrow (\Pi(x : A) , \lVert U(x) \rVert) \rightarrow \lVert \Pi(x : A) , U(x) \rVert
  \end{equation}
\end{lemma}
\begin{proof}
  Let \(A\) be a cardinally finite type, \(U\) be a type family on \(A\), and
  \(f\) be a dependent function of type \(\Pi(x : A) , \lVert U(x) \rVert\).

  First, since our goal is itself propositionally truncated, we have access to
  values under truncations: put another way, in the context of proving our goal,
  we can rely on the fact that \(A\) is manifestly Bishop finite.
  Using the same technique as we did in lemma~\ref{split-enum-pi}, we can switch
  from working with dependent functions from \(A\) to \(n\)-tuples, where \(n\)
  is the cardinality of \(A\).
  This changes our goal to the following:
  \begin{equation}
    \mathbf{Tuple}(n, \lVert \wc \rVert \circ U) \rightarrow \lVert \mathbf{Tuple}(n, U) \rVert
  \end{equation}
  Since \(\lVert \wc \rVert\) is closed under finite products, this function
  exists (in fact, using the fact that \(\lVert \wc \rVert\) forms a monad, we
  can recognise this function as \verb+sequenceA+ from the \verb+Traversable+
  class in Haskell).
\end{proof}


This gets us all of the necessary closure proofs on \(\mathcal{C}\).
\section{Manifest Enumerability} \label{manifest-enumerability}
We have now explored quite far in the ``less informative'' direction.
However, all three predicates we have examined are equally \emph{restrictive}:
in this section we will see a predicate which is much less restrictive.
In particular, this predicate ranges over non-set types.


\begin{definition}[Manifest Enumerability]
  Manifest enumerability is an enumeration predicate like Bishop finiteness or
  split enumerability with the only difference being a propositionally truncated
  membership proof.
  \begin{equation}
    \mathcal{E}(A) \coloneqq \Sigma {(\mathit{xs} : \mathbf{List}(A))} , \Pi {(x : A)} , \lVert x \in \mathit{xs} \rVert
  \end{equation}
\end{definition}
As a function-based definition, this predicate represents surjections.

\begin{definition}[Surjections] \label{surjections}
  We define proper surjections (not split surjections) here \cite[definition
  4.6.1]{hottbook}.
  \begin{alignat}{3}
    &\text{surj}(f)             &&\coloneqq \Pi(y : B) , \lVert \text{fib}_f(y) \rVert \label{surj-eqn} \\
    &A \twoheadrightarrow B     &&\coloneqq \Sigma (f : A \rightarrow B) , \text{surj}(f) \label{surj-arrow-eqn}
  \end{alignat}
\end{definition}
\begin{lemma}
  Manifest enumerability is equivalent to a surjection from a finite prefix of
  the natural numbers.
  \begin{equation}
    \mathcal{E}(A) \simeq \Sigma(n : \mathbb{N}) , (\mathbf{Fin}(n) \twoheadrightarrow A)
  \end{equation}
\end{lemma}
\begin{proof}
  \begin{align*}
     \mathcal{E}(A) &
    \simeq \Sigma (\mathit{xs} : \textbf{List}(A)) , \Pi {(x : A)} , \lVert x \in \mathit{xs} \rVert
    && \text{def.~\ref{split-enum-def} }(\mathcal{E})
    \\
    & \simeq \Sigma (\mathit{xs} : \textbf{List}(A)) , \Pi {(x : A)} , \lVert \text{fib}_{\text{snd}(\mathit{xs})}(x) \rVert
    && \text{eqn.~\ref{container-membership} } (\in)
    \\
    & \simeq \Sigma (\mathit{xs} : \textbf{List}(A)) , \text{surj}(\text{snd}(\mathit{xs}))
    && \text{eqn.~\ref{surj-eqn} (surj)}
    \\
    & \simeq \Sigma (\mathit{xs} : \llbracket \mathbb{N} , \mathbf{Fin} \rrbracket (A)) , \text{surj}(\text{snd}(\mathit{xs}))
    && \text{def.~\ref{List} } (\mathbf{List})
    \\
    & \simeq \Sigma (\mathit{xs} : \Sigma (n : \mathbb{N}) , \Pi (i : \mathbf{Fin}(n)) , A) , \text{surj}(\text{snd}(\mathit{xs}))
    && \text{eqn.~\ref{container-interp} } (\llbracket \wc \rrbracket)
    \\
    & \simeq \Sigma (n : \mathbb{N}) , \Sigma (f : \mathbf{Fin}(n) \rightarrow A) , \text{surj}(f)
    && \text{Reassociation of } \Sigma
    \\
    & \simeq \Sigma (n : \mathbb{N}) , ( \mathbf{Fin}(n) \twoheadrightarrow A )
    && \text{eqn.~\ref{surj-arrow-eqn} } (\twoheadrightarrow) \; \qedhere
  \end{align*}
\end{proof}
\subsection{Instances for Non-Set Types}
The truncation has another very important implication: it means that the
predicate doesn't provide decidable equality on the underlying type.
Remember, this is how we knew that our previous predicates wouldn't allow for
non-set types: because they implied decidable equality, they also implied that
all conforming types had homotopy levels of at most 2. \todo{Are we doing the
  homotopy levels starting from -2 thing or from 0?}
This suggests that non-set types like the circle could conform to this
finiteness predicate.
\begin{definition}[\(S^1\)] \label{circle-def}
  The circle, \(S^1\), can be represented in HoTT as a higher inductive type.
  \begin{equation}
    \begin{alignedat}{3}
      S^1 \coloneqq & \; \text{base} &&: S^1 ; \\
      | & \; \text{loop} &&: \text{base} \equiv \text{base} ; 
    \end{alignedat}
  \end{equation}
  We will use it here as an example of a non-set type, i.e. a type for which not
  all paths are equal.
  This also means that it does not have decidable equality.
\end{definition}
\begin{lemma}
  The circle \(S^1\) is manifestly enumerable.
\end{lemma}
\begin{proof}
  The support list firstly is a list containing the point constructor for the
  circle.
  Since the cover proof is truncated, we need only consider the point
  constructors of the circle: as such, the cover proof is essentially the same
  as the one for \(\AgdaDatatype{\ensuremath{\mathcal{E}!}}(\top)\).
\end{proof}
\subsection{Relation To Split Enumerability}
It is trivially easy to construct a proof that any split enumerable type is
manifest enumerable: we simply truncate the membership proof.
Going the other was requires us to extract a non-truncated proof from a
truncated one.
This proof relies on the following lemma:
\begin{lemma}
  We can ``recompute'' a truncated proof given a decision over a proof of the
  same type.
  \begin{equation}
    \frac{\lVert A \rVert \; \; \; \mathbf{Dec}(A)}{A}
  \end{equation}
\end{lemma}
\begin{proof}
  We proceed by case-analysis over the decision over \(A\).
  In the case where \(A\) is proven, we are done.
  In the case where \(A\) is disproven, we use lemma~\ref{prop-refute} to
  derive impossibility.
\end{proof}

\begin{lemma} \label{manifest-enum-to-split-enum}
  A manifestly enumerable type with decidable equality is split enumerable.
\end{lemma}
\begin{proof}
  The only difference between manifest enumerability and split enumerability is
  the membership proof: therefor our goal for this proof is to construct a
  function of the following type:
  \begin{equation}
    \lVert x \in \mathit{xs} \rVert \rightarrow x \in \mathit{xs}
  \end{equation}
  Given decidable equality over the type of \(x\).

  We do this using the previous recompute lemma: that tells us that all we need
  to construct is a decision for \(x \in \mathit{xs}\), and it will be able to
  derive the proof itself.
  Such a decision procedure is not difficult to construct: for any value \(x\)
  and list \(\mathit{xs}\), we proceed through the list \(\mathit{xs}\), testing
  if \(x\) is equal to any of its contents.
  If it is, we return that we have proven the goal, and that \(x\) is indeed
  present in \(\mathit{xs}\).
  Otherwise, we know that \(x\) cannot be in \(\mathit{xs}\) (since we've tested
  every value), so we return that the goal has been disproven.
\end{proof}
\section{Kuratowski Finiteness} \label{kuratowski}
The one big missing definition of finiteness to cover is \emph{Kuratowski}
finiteness.
While it's quite important, it's also quite different from the definitions we've
seen so far.
It starts with an encoding of the free join semilattice.
\begin{definition}[Free Join Semilattice]
  \(\mathcal{K}(A)\) is the free join semilattice, or, alternatively, the type
  of Kuratowski-finite subsets of \(A\).
  \begin{inductivetype}{\mathcal{K}(A)}
    \inductivetypeclause{\left[ \right]}{\mathcal{K}(A)}
    \inductivetypeclause{\wc \dblcolon \wc}{A \rightarrow \mathcal{K}(A) \rightarrow \mathcal{K}(A)}
    \inductivetypeclause{\text{com}}{\Pi (x, y: A) , \Pi (\mathit{xs} : \mathcal{K}(A)) , x \dblcolon y \dblcolon \mathit{xs} \equiv y \dblcolon x \dblcolon \mathit{xs}}
    \inductivetypeclause{\text{dup}}{\Pi (x : A) , \Pi (\mathit{xs} : \mathcal{K}(A)) , x \dblcolon x \dblcolon \mathit{xs} \equiv x \dblcolon \mathit{xs}}
    \inductivetypeclause{\text{trunc}}{\Pi (\mathit{xs}, \mathit{ys}: \mathcal{K}(A)) , \Pi (p, q : \mathit{xs} \equiv \mathit{ys}) , p \equiv q}
  \end{inductivetype}
  We define it as a HIT (definition~\ref{HITs}).
  The first two constructors are point constructors, giving ways to create
  values of type \(\mathcal{K}(A)\).
  They are also recognisable as the two constructors for finite lists, a type
  which represents the free monoid.

  The next two constructors add extra paths to the type: equations that usage of
  the type must obey.
  These extra paths turn the free monoid into the free \emph{commutative} (com)
  \emph{idempotent} (dup) monoid.

  The final constructor enforces that the type \(\mathcal{K}(A)\) must be a set.
\end{definition}
The Kuratowski finite subset is a free join semilattice (or, equivalently, a
free commutative idempotent monoid).
More prosaically, \(\mathcal{K}\) is the abstract data type for finite sets, as
defined in the Boom hierarchy \cite{boomFurtherThoughtsAbstracto1981,
  bunkenburgBoomHierarchy1994}.
However, rather than just being a specification, \(\mathcal{K}\) is fully usable
as a data type in its own right, thanks to HITs.

Other definitions of \(\mathcal{K}\) exist (such as the one in
\cite{fruminFiniteSetsHomotopy2018}) which make the fact that \(\mathcal{K}\) is
the free join semilattice more obvious.
We have included such a definition in our formalisation, and proven it
equivalent to the one above.

Next, we need a way to say that an entire type is Kuratowski finite.
For that, we will need to define membership of \(\mathcal{K}\).
\begin{definition}[Membership of \(\mathcal{K}\)]
  Membership is defined by pattern-matching on \(\mathcal{K}\).
  The two point constructors are handled like so:
  \begin{equation}
    \begin{alignedat}{2}
      x \in&& \; []                      &\coloneqq \bot ; \\
      x \in&& \; y \dblcolon \mathit{ys} &\coloneqq \lVert x \equiv y \uplus x \in \mathit{ys} \rVert ;
    \end{alignedat}
  \end{equation}
  The \(\text{com}\) and \(\text{dup}\) constructors are handled by proving that
  the truncated form of \(\uplus\) is itself commutative and idempotent.
  The type of propositions is itself a set, satisfying the \(\text{trunc}\)
  constructor.
\end{definition}
Finally, we have enough background to define Kuratowski finiteness.
\begin{definition}[Kuratowski Finiteness]
  \begin{equation}
    \mathcal{K}^{f}(A) = \Sigma {(\mathit{xs} : \mathcal{K}(A))} , \Pi (x : A) , x \in \mathit{xs}
  \end{equation}
\end{definition}

We also have the following two lemmas, proven in both
\cite{fruminFiniteSetsHomotopy2018} and our formalisation.
\begin{lemma}
  \(\mathcal{K}^f\) is a mere proposition.
\end{lemma}
\begin{lemma}
  This circle \(S^1\) is Kuratowski finite.
\end{lemma}
\subsection{Relation to Cardinal Finiteness}
\begin{lemma} \label{cardinal-kuratowski}
  Cardinal finiteness is equivalent to Kuratowski finiteness over a discrete
  set.
  \begin{equation}
    \mathcal{C}(A) \simeq \mathcal{K}^f(A) \times \text{Discrete}(A)
  \end{equation}
\end{lemma}
This proof is constructed by providing a pair of functions: one from
\(\mathcal{C}(A)\) to \(\mathcal{K}^f(A) \times \text{Discrete}(A)\), and one the
other way.
This pair implies an equivalence, because both source and target are
propositions.
The actual functions themselves are proven in our formalisation, as well as in
\cite{fruminFiniteSetsHomotopy2018}.


%%% Local Variables:
%%% mode: latex
%%% TeX-master: "../paper"
%%% End:
\chapter{Topos} \label{topos}
In this section we will examine the categorical interpretation of finite sets.
In particular, we will prove that decidable Kuratowski finite types form a
\(\Pi\)-pretopos.
A lot of the work for this proof has been done already: in
Theorem~\ref{cardinal-kuratowski} we saw that discrete Kuratowski finite types
were equivalent to Cardinally finite types.
We will use the latter definition implementation-wise from now on, as it is
slightly easier to work with: CuTT's transport means we can do this without loss
of generality.

There are two reasons we're interested in the categorical and topos-theoretic
interpretation of finite sets: first, it's an important theoretical grounding
for finite sets, which allows us to understand them in the context of other
set-like constructions.
Secondly, and more practically, the language of a topos is (or in our case the
\(\Pi\)-pretopos) is a common standard framework for doing mathematics
generally.
This makes it a good basis for an API for building QuickCheck-like generators,
for example.
\section{Categories in HoTT}
At first glance, HoTT seems like a perfect setting for category theory: the
univalence axiom identifies isomorphisms with equality, a useful tool for
category theory missing from MLTT.
While this initial impression is broadly true, the construction of categories in
HoTT is unfortunately quite complex and involved (much of the following is a
summary of \citet[chapter 9]{hottbook}).

\todo{references here are tricky, need to disentangle the contributions quite
  precisely}
Much of this section is simply a summary of parts of \citet[chapter
9]{hottbook}.
The formal proofs we provide are part translation of those proofs in that
chapter, part from \cite{iversenFredefoxCat2018}
\cite{huProofrelevantCategoryTheory2020}, and part our own.

First, we need to think about the type of objects and arrows.
We cannot, unfortunately, leave them unrestricted: because of the potential for
higher homotopy in HoTT types \todo{This sentence is a tongue twister}, we have
to restrict the type of arrows to just the sets.
This notion: that of a category with all the usual laws such that arrows are a
set, is called a \emph{precategory}.
\begin{agdalisting}
  \ExecuteMetaData[agda/Categories.tex]{precategory}
\end{agdalisting}
We will use long arrows to refer to morphisms within a category:
\begin{agdalisting}
  \ExecuteMetaData[agda/Categories.tex]{morph-arrow}
\end{agdalisting}

From here, we can define a notion of isomorphisms.
\begin{agdalisting}
  \ExecuteMetaData[agda/Categories.tex]{isomorphism}
\end{agdalisting}
It's a condition on this type which separates the precategories from the
categories: if it satisfies a form of univalence, it the precategory is a full
category.
\begin{agdalisting}
  \ExecuteMetaData[agda/Categories.tex]{cat-univalence}
\end{agdalisting}
\section{The Category of Sets}
Next we'll look at how to construct the category of sets (in the HoTT sense).
Much of this work comes directly from \citet[chapter 10]{hottbook} and
\citet{rijkeSetsHomotopyType2015}.
The formalisation, however, is novel, as far as we know.

The objects are represented by a \(\Sigma\):
\begin{agdalisting}
  \ExecuteMetaData[agda/Snippets/Category.tex]{hset}
\end{agdalisting}
This will be quite similar to our objects for finite sets.

Since sets in HoTT don't form a topos, there are quite a few smaller lemmas we
need to prove to get as close as we can (a \(\Pi W\)-pretopos): we won't include
them here, other than the closure proofs in the following section.
\section{Closure}
The two most involved proofs for showing that discrete Kuratowski sets form a
\(\Pi\)-pretopos are those proofs that show closure under \(\Pi\) and
\(\Sigma\).
We will describe them here.
\subsection{Closure of the Ordered Predicates}
First, we will show that split enumerability (and, by extension, manifest
enumerability) are closed under \(\Pi\) and \(\Sigma\).
This is the first stepping stone on our way to prove that cardinal finiteness is
closed under the same.

Practically speaking, these proofs also open up a wide number of other closure
proofs to us.
By proving that dependent products and sums are finite, we get the non-dependent
cases for free.

\begin{lemma} \label{split-enum-sigma} \todo{Convert to Agda}
  Split enumerability is closed under \(\Sigma\).
  \begin{agdalisting}
    \ExecuteMetaData[agda/Cardinality/Finite/SplitEnumerable.tex]{split-enum-sigma}
  \end{agdalisting}
\end{lemma}
\begin{proof}
  Our task is to construct the two components of the output pair: the support
  list, and the cover proof.
  We'll start with the support list: this is constructed by taking the Cartesian
  product of the input support lists.
  \begin{agdalisting}
    \ExecuteMetaData[agda/Cardinality/Finite/SplitEnumerable.tex]{sup-sigma}
  \end{agdalisting}
  We use do notation here because we're working the list monad: this applies the
  latter function (\(ys\)) to every element of the list \(xs\), and concatenates
  the results.

  To show that this does indeed cover every element of the target type is a
  little intricate, but not necessarily difficult. \todo{Should a proof of this
    be included?}
\end{proof}

Next we'll look at closure under \(\Pi\).
In MLTT, this is of course not provable: since all of the finiteness predicates
we have seen so far imply decidable equality, and since we don't have any kind
of decidable equality on functions in MLTT, we know that we won't be able to
show that any kind of function is finite; even one like \(\AgdaDatatype{Bool}
\rightarrow \AgdaDatatype{Bool}\).

CuTT is not so restricted.
Since we have things like function extensionality and transport, we can indeed
prove the finiteness of function types.
Our proof here makes use directly of the univalence axiom, and makes use
furthermore of all the previous closure proofs.
\begin{theorem} \label{split-enum-pi}
  Split enumerability is closed under dependent functions
  (\(\Pi\)-types).
  \begin{agdalisting}
    \ExecuteMetaData[agda/Cardinality/Finite/ManifestBishop.tex]{pi-clos}
  \end{agdalisting}
\end{theorem}
\begin{proof}
  Let \(A\) be a split enumerable type, and \(U\) be a type family from \(A\),
  which is split enumerable over all points of \(A\).

  As \(A\) is split enumerable, we know that it is also manifestly Bishop finite
  (lemma~\ref{split-enum-to-manifest-bishop}), and consequently we know \(A
  \simeq \AgdaDatatype{Fin}\;n\), for some \(n\) (lemma~\ref{bishop-equiv}).
  We can therefore replace all occurrences of \(A\) with \(\AgdaDatatype{Fin}\;n\),
  changing our goal to:
  \begin{equation}
    \frac{
      \AgdaDatatype{\ensuremath{\mathcal{E}!}}\;(\AgdaDatatype{Fin}\;n) \; \; \; \left((x : \AgdaDatatype{Fin}\;n) \rightarrow \AgdaDatatype{\ensuremath{\mathcal{E}!}}\;\left( U\;x \right)\right)
    }{
      \AgdaDatatype{\ensuremath{\mathcal{E}!}}\left((x : \AgdaDatatype{Fin}\;n) \rightarrow U\;x\right)
    }
  \end{equation}
  
  We then define the type of \(n\)-tuples over some type family.
  \begin{agdalisting}
    \ExecuteMetaData[agda/Data/Tuple/UniverseMonomorphic.tex]{tuple-def}
  \end{agdalisting}
  We can show that this type is equivalent to functions (proven in our formalisation):
  \begin{agdalisting}
    \ExecuteMetaData[agda/Data/Tuple/UniverseMonomorphic.tex]{tuple-iso}
  \end{agdalisting}
  And therefore we can simplify again our goal to the following:
  \begin{equation}
    \frac{
      \AgdaDatatype{\ensuremath{\mathcal{E}!}}\;(\AgdaDatatype{Fin}\;n) \; \; \; ((x : \AgdaDatatype{Fin}\;n) \rightarrow \AgdaDatatype{\ensuremath{\mathcal{E}!}}\left( U\;x \right))
    }{
      \AgdaFunction{\ensuremath{\mathcal{E}!}}\;\left(\AgdaFunction{Tuple}\;n\;U\right)
    }
  \end{equation}
  
  We can prove this goal by showing that \(\AgdaFunction{Tuple}\;n\;U\) is split
  enumerable: it is made up of finitely many products of points of \(U\), which
  are themselves split enumerable, and \agdatop, which is also split enumerable.
  Lemma~\ref{split-enum-sigma} shows us that the product of finitely many split
  enumerable types is itself split enumerable, proving our goal.
\end{proof}
\subsection{Closure on Cardinal Finiteness}
Since we don't have a function of type \(\mathcal{C}(A) \rightarrow
\AgdaDatatype{\ensuremath{\mathcal{B}}}\;A\), closure proofs on \(\AgdaDatatype{\ensuremath{\mathcal{B}}}\) do not transfer over to
\(\mathcal{C}\) trivially (unlike with \(\AgdaDatatype{\ensuremath{\mathcal{E}!}}\) and \(\AgdaDatatype{\ensuremath{\mathcal{B}}}\)).
The cases for \(\bot\), \(\top\), and \(\AgdaDatatype{Bool}\) are simple to adapt: we
can just propositionally truncate their Bishop finiteness proof.

Non-dependent operators like \(\times\), \(\uplus\), and \(\rightarrow\) are
also relatively straightforward: since \(\AgdaDatatype{\ensuremath{\lVert\_\rVert}}\) forms a monad, we
can apply \(n\)-ary functions to values inside it, combining them together.
\begin{agdalisting}
  \ExecuteMetaData[agda/Cardinality/Finite/ManifestBishop.tex]{times-clos-sig}
\end{agdalisting}
Into a truncated context:
\begin{agdalisting}
  \ExecuteMetaData[agda/Cardinality/Finite/Cardinal.tex]{times-clos-impl}
\end{agdalisting}


Unfortunately, for the dependent type formers like \(\Sigma\) and \(\Pi\), the
same trick does not work.
We have closure proofs like:
\begin{equation}
  \frac{
    \AgdaDatatype{\ensuremath{\mathcal{B}}}\;A \; \; \; ((x : A) \rightarrow \AgdaDatatype{\ensuremath{\mathcal{B}}}\;(U\;x))
  }{
    \AgdaDatatype{\ensuremath{\mathcal{B}}}\;((x : A) \rightarrow U\;x)
  }
\end{equation}
If we apply the monadic truncation trick we can derive closure proofs like the
following:
\begin{equation}
  \frac{
    \AgdaDatatype{\ensuremath{\lVert}}\; \AgdaDatatype{\ensuremath{\mathcal{B}}}\;A \;\AgdaDatatype{\ensuremath{\rVert}} \; \; \; \AgdaDatatype{\ensuremath{\lVert}}\; ((x : A) \rightarrow \AgdaDatatype{\ensuremath{\mathcal{B}}}\;(U\;x)) \;\AgdaDatatype{\ensuremath{\rVert}}
  }{
    \AgdaDatatype{\ensuremath{\lVert}}\; \AgdaDatatype{\ensuremath{\mathcal{B}}}\;((x : A) \rightarrow U\;x) \;\AgdaDatatype{\ensuremath{\rVert}}
  }
\end{equation}
However our \emph{desired} closure proof is the following:
\begin{equation}
  \frac{
    \AgdaDatatype{\ensuremath{\lVert}}\; \AgdaDatatype{\ensuremath{\mathcal{B}}}\;A \;\AgdaDatatype{\ensuremath{\rVert}} \; \; \; ((x : A) \rightarrow \AgdaDatatype{\ensuremath{\lVert}}\; \AgdaDatatype{\ensuremath{\mathcal{B}}}\;(U\;x) \;\AgdaDatatype{\ensuremath{\rVert}})
  }{
    \AgdaDatatype{\ensuremath{\lVert}}\; \AgdaDatatype{\ensuremath{\mathcal{B}}}\;((x : A) \rightarrow U\;x) \;\AgdaDatatype{\ensuremath{\rVert}}
  }
\end{equation}
They don't match!

The solution would be to find a function of the following type:
\begin{equation}
  ((x : A) \rightarrow \AgdaDatatype{\ensuremath{\lVert}}\; \AgdaDatatype{\ensuremath{\mathcal{B}}}\;(U\;x) \;\AgdaDatatype{\ensuremath{\rVert}}) \rightarrow
  \AgdaDatatype{\ensuremath{\lVert}}\; (x : A) \rightarrow \AgdaDatatype{\ensuremath{\mathcal{B}}}\;(U\;x) \;\AgdaDatatype{\ensuremath{\rVert}}
\end{equation}
However we might be disheartened at realising that this is a required goal: the
above equation is \emph{extremely} similar to the axiom of choice!
\begin{definition}[Axiom of Choice]
  In HoTT, the axiom of choice is commonly defined as follows \cite[lemma
  3.8.2]{hottbook}.
  For any set \(A\), and a type family \(U\) which is a set at all the points
  of \(A\), the following function exists:
  \begin{equation}
    \left( (x : A) \rightarrow  \AgdaDatatype{\ensuremath{\lVert}}\; U(x) \;\AgdaDatatype{\ensuremath{\rVert}} \right) \rightarrow \AgdaDatatype{\ensuremath{\lVert}}\; (x : A) \rightarrow U(x) \;\AgdaDatatype{\ensuremath{\rVert}}
  \end{equation}
\end{definition}
Luckily the axiom of choice \emph{does} hold for cardinally finite types,
allowing us to prove the following:
\begin{lemma}
  \begin{equation}
    \agdacal{C}\;A \rightarrow ((x : A) \rightarrow \AgdaDatatype{\ensuremath{\lVert}}\; U(x) \;\AgdaDatatype{\ensuremath{\rVert}}) \rightarrow \AgdaDatatype{\ensuremath{\lVert}}\; (x : A) \rightarrow U(x) \;\AgdaDatatype{\ensuremath{\rVert}}
  \end{equation}
\end{lemma}
\begin{proof}
  Let \(A\) be a cardinally finite type, \(U\) be a type family on \(A\), and
  \(f\) be a dependent function of type \(\Pi(x : A) , \AgdaDatatype{\ensuremath{\lVert}}\; U(x) \;\AgdaDatatype{\ensuremath{\rVert}}\).

  First, since our goal is itself propositionally truncated, we have access to
  values under truncations: put another way, in the context of proving our goal,
  we can rely on the fact that \(A\) is manifestly Bishop finite.
  Using the same technique as we did in lemma~\ref{split-enum-pi}, we can switch
  from working with dependent functions from \(A\) to \(n\)-tuples, where \(n\)
  is the cardinality of \(A\).
  This changes our goal to the following:
  \begin{equation}
    \mathbf{Tuple}(n, \AgdaDatatype{\ensuremath{\lVert}}\; \wc \;\AgdaDatatype{\ensuremath{\rVert}} \circ U) \rightarrow \AgdaDatatype{\ensuremath{\lVert}}\; \mathbf{Tuple}(n, U) \;\AgdaDatatype{\ensuremath{\rVert}}
  \end{equation}
  Since \(\AgdaDatatype{\ensuremath{\lVert\_ \rVert}}\) is closed under finite products, this function
  exists (in fact, using the fact that \(\AgdaDatatype{\ensuremath{\lVert\_ \rVert}}\) forms a monad, we
  can recognise this function as \verb+sequenceA+ from the \verb+Traversable+
  class in Haskell).
\end{proof}


This gets us all of the necessary closure proofs on \(\mathcal{C}\).
\section{The Absence of the Subobject Classifier}
\begin{agdalisting} \label{filter-subobject}
  \ExecuteMetaData[agda/Cardinality/Finite/SplitEnumerable.tex]{subobject}
\end{agdalisting}


\section{Closure}
For the first three closure proofs, we only consider split enumerability:
as it is the strongest of the finiteness predicates, we can derive the other
closure proofs from it.

\section{The Category of Finite Sets}
HoTT and CuTT seem to be especially suitable settings for formalisations of
category theory.
The univalence axiom in particular allows us to treat categorical isomorphisms
as equalities, saving us from the dreaded ``setoid hell''.

We follow \cite[chapter 9]{hottbook} in its treatment of
categories in HoTT, and in its proof that sets do indeed form a category.
We will first briefly go through the construction of the category
\(\mathit{Set}\), as it differs slightly from the usual method in type theory.

First, the type of objects and arrows:
\begin{alignat}{3}
  &\text{Obj}_\mathit{Set}      &&\coloneqq \Sigma(x : \mathbf{Type}) , \text{isSet}(x) \\
  &\text{Hom}_\mathit{Set}(x , y) &&\coloneqq  \text{fst}(x) \rightarrow \text{fst}(y)
\end{alignat}
As the type of objects makes clear, we have already departed slightly from the
simpler \(\text{Obj}_\mathit{Set} \coloneqq \mathbf{Type}\) way of doing things:
of course we have to, as HoTT allows non-set types.
Furthermore, after proving the usual associativity and identity laws for
composition (which are definitionally true in this case), we must further show
\(\text{isSet}(\text{Hom}_\mathit{Set}(x,y))\); even then we only have a
precategory.

To show that \(\mathit{Set}\) is a category, we must show that categorical
isomorphisms are equivalent to equivalences.
In a sense, we must give a univalence rule for the category we are working in.

We have provided formal proofs that \(\mathit{Set}\) does indeed form a
category, and the following:
\begin{theorem}[The Category of Finite Sets]
  Finite sets form a category in HoTT when defined like so:
  \begin{equation}
    \begin{alignedat}{3}
      &\text{Obj}_\mathit{FinSet}      &&\coloneqq \Sigma(x : \mathbf{Type}) , \mathcal{C}(x) \\
      &\text{Hom}_\mathit{FinSet}(x , y) &&\coloneqq  \text{fst}(x) \rightarrow \text{fst}(y)
    \end{alignedat}
  \end{equation}
\end{theorem}
\section{The \(\Pi\)-pretopos of Finite Sets}
For this proof, we follow again the proof that \(\mathit{Set}\) forms a \(\Pi
W\)-pretopos from \cite[chapter 10]{hottbook} and
\cite{rijkeSetsHomotopyType2015}.
The difference here is that clearly we do not have access to \(W\)-types, as
they would permit infinitary structures.

We first must show that \(\mathit{Set}\) has an initial object and finite,
disjoint sums, which are stable under pullback.
We also must show that \(\mathit{Set}\) is a regular category with effective
quotients.
We now have a pretopos: the presence of \(\Pi\) types make it a
\(\Pi\)-pretopos.

We have proven the above statements for both \(\mathit{Set}\) and
\(\mathit{FinSet}\).
As far as we know, this is the first formalisation of either.
\begin{theorem} \label{finite-topos}
  The category of finite sets, \(\mathit{FinSet}\), forms a \(\Pi\)-pretopos.
\end{theorem}


%%% Local Variables:
%%% mode: latex
%%% TeX-master: "../paper"
%%% End:
\chapter{Countably Infinite Types} \label{infinite}
In the previous sections we saw different flavours of finiteness which were
really just different flavours of relations to \(\mathbf{Fin}\).
In this section we will see that we can construct a similar classification of
relations to \(\mathbb{N}\), in the form of the countably infinite types.
\section{Two Countable Types}
The two types for countability we will consider are analogous to split
enumerability and cardinal finiteness.
The change will be a simple one: we will swap out lists for streams.
\begin{definition}[Streams]
  \begin{equation}
    \mathbf{Stream}(A) \coloneqq (\mathbb{N} \rightarrow A)
    \simeq \llbracket \top , \text{const}(\mathbb{N}) \rrbracket
  \end{equation}
\end{definition}
\begin{definition}[Split Countability]
  \begin{equation}
    \aleph_0!(A) \coloneqq \Sigma {(\mathit{xs} : \mathbf{Stream}(A))} , \Pi {(x : A)} , x \in \mathit{xs}
  \end{equation}
\end{definition}
This type is definitionally equal to it surjection equivalent (\(\mathbb{N}
\twoheadrightarrow ! \; A\)).
We construct the unordered, propositional version of the predicate in much the
same way as we constructed cardinal finiteness.
\begin{definition}[Countability]
  \begin{equation}
    \aleph_0(A) \coloneqq \lVert \aleph_0!(A) \rVert
  \end{equation}
\end{definition}

From both of these types we can derive decidable equality.
\begin{lemma}
  Any countable type has decidable equality.
\end{lemma}
\section{Closure}
\begin{figure}
  \centering
  \begin{subfigure}[b]{.5\linewidth}
    \begin{tikzcd}[sep=tiny,font=\footnotesize]
      (1,e) \ar[rdddd, out=-45, in=135] & (2,e) \ar[rdddd, out=-45, in=135] & (3,e) \ar[rdddd, out=-45, in=135] & (4,e) \ar[rdddd, out=-45, in=135] & (5,e)        \\
      (1,d) \ar[u]     & (2,d) \ar[u]    & (3,d) \ar[u]    & (4,d) \ar[u]    & (5,d) \ar[u] \\
      (1,c) \ar[u]     & (2,c) \ar[u]    & (3,c) \ar[u]    & (4,c) \ar[u]    & (5,c) \ar[u] \\
      (1,b) \ar[u]     & (2,b) \ar[u]    & (3,b) \ar[u]    & (4,b) \ar[u]    & (5,b) \ar[u] \\
      (1,a) \ar[u]     & (2,a) \ar[u]    & (3,a) \ar[u]    & (4,a) \ar[u]    & (5,a) \ar[u]
    \end{tikzcd}
    \caption{Cartesian}
    \label{cartesian}
  \end{subfigure}%
  \begin{subfigure}[b]{.5\linewidth}
    \begin{tikzcd}[sep=tiny,font=\footnotesize]
      (1,e) \ar[dr] & (2,e) \ar[dr]  & (3,e) \ar[dr]    & (4,e) \ar[dr] & (5,e) \\
      (1,d) \ar[dr] & (2,d) \ar[dr]  & (3,d) \ar[dr]    & (4,d) \ar[dr] & (5,d) \ar[u] \\
      (1,c) \ar[dr] & (2,c) \ar[dr]  & (3,c) \ar[dr]    & (4,c) \ar[dr] & (5,c) \ar[uul] \\
      (1,b) \ar[dr] & (2,b) \ar[dr]  & (3,b) \ar[dr]    & (4,b) \ar[dr] & (5,b) \ar[uuull, out=130, in=-50] \\
      (1,a) \ar[u]  & (2,a) \ar[uul, out=130, in=-50] & (3,a) \ar[uuull, out=130, in=-50] & (4,a) \ar[uuuulll, out=130, in=-50] & (5,a) \ar[uuuulll, out=130, in=-50]
    \end{tikzcd}
    \caption{Cantor}
    \label{cantor}
  \end{subfigure}
  \caption{Two possible products for the sets \(\left[ 1 \dots 5 \right]\) and
    \(\left[  a \dots e \right]\)}
  \label{pairings}
\end{figure}
We know that countable infinity is not closed under the exponential (function
arrow), so the only closure we need to prove is \(\Sigma\) to cover all of
what's left.
\begin{theorem} \label{split-countability-sigma}
  Split countability is closed under \(\Sigma\).
\end{theorem}
We know that countable infinity is not closed under the exponential (function
arrow), so the only closure we need to prove is \(\Sigma\) to cover all of
what's left.
To do this we have to take a slightly different approach to the functions we
defined before.
Figure~\ref{pairings} illustrates the reason why: previously, we used the
depth-first product pairing for each support list.
This diverges if the first list is infinite, never exploring anything other than
the first element in the second list.
Instead, we use here the cantor pairing function, which performs a breadth-first
search of the pairings of both lists.

Finally, while we have lost certain closure proofs by allowing for infinite
types, we also \emph{gain} some: in particular the Kleene star.
\begin{theorem}
  Split countability is closed under Kleene star.
  \begin{equation}
    \aleph_0!(A) \rightarrow \aleph_0!(\mathbf{List}(A))
  \end{equation}
\end{theorem}
Again, this proof requires a particular pattern to ensure productivity.
The pattern here builds an intermediate stream \(\mathcal{KV}\) of non-empty
lists from the input support stream \(\mathit{xs}\), which is subsequently
flattened.
\begin{equation}
  \mathcal{KV}_i \coloneqq \left[ \left[ \mathit{xs}_{j - 1} \mid j \in \mathit{js} \right] \mid \mathit{js} \in \mathbf{List}(\mathbb{N}) ; \text{sum}(\mathit{js}) = i ; 0 \notin \mathit{js}  \right]
\end{equation}

%%% Local Variables:
%%% mode: latex
%%% TeX-master: "../paper"
%%% End:
\chapter{Search} \label{search}
A common theme in dependently-typed programming is that proofs of interesting
theoretical things often correspond to useful algorithms in some way
related to that thing.
Finiteness is one such case: if we have a proof that a type \(A\) is finite,
we should be able to search through all the elements of that type in a
systematic, automated way.

As it happens, this kind of search is a very common method of proof automation
in dependently-typed languages like Agda.
Proofs of statements like ``the following function is associative''
\begin{agdalisting}
  \ExecuteMetaData[agda/Snippets/Bool.tex]{and-def}
\end{agdalisting}
can be tedious: the associativity proof in particular would take \(2^3 = 8\)
cases.
This is unacceptable!
There are only finitely many cases to examine, after all, and we're
\emph{already} on a computer: why not automate it?
A proof that \(\AgdaDatatype{Bool}\) is finite can get us much of the way to a
library to do just that.
\todo{These examples so far are pretty focused on the bool associativity
  example.
  I'm not sure I can think of a good way to put countdown in instead: will we
  try switch?
  Or will we keep the bool for this short bit?
  }

Similar automation machinery can be leveraged to provide search algorithms for
certain ``logic programming''-esque problems.
Using the machinery we will describe in this section, though, when the program
says it finds a solution to some problem that solution will be accompanied by a
formal \emph{proof} of its correctness.

In this section, we will describe the theoretical underpinning and
implementation of a library for proof search over finite domains, based on the
finiteness predicates we have introduced already.
The library will be able to prove statements like the proof of associativity
above, as well as more complex statements.
As a running example for a ``more complex statement'' we will use the countdown
problem, which we have been using throughout: we will demonstrate how to
construct a prover for the existence of, or absence of, a solution to a given
countdown puzzle.

The API for writing searches over finite domains comes from the language of the
\(\Pi\)-pretopos: with it we will show how to compose QuickCheck-like generators
for proof search, with the addition of some automation machinery that allows us
to prove things like the associativity in a couple of lines:
\begin{agdalisting} \label{bool-assoc-auto-proof}
  \ExecuteMetaData[agda/Snippets/Bool.tex]{bool-assoc-auto-proof}
\end{agdalisting}

We have already, in previous sections, explored the theoretical implications of
Cubical Type Theory on our formalisation.
With this library for proof search, however, we will see two distinct
practical applications which would simply not be possible without
computational univalence.
First and foremost: our proofs of finiteness, constructed with the API we will
describe, have all the power of full equalities.
Put another way any proof over a finite type \(A\) can be lifted to any other
type with the same cardinality.
Secondly our proof search can range over functions: we could, for instance, have
asked the prover to find if \emph{any} function over \(\AgdaDatatype{Bool}\) is
associative, and if so return it to us.
\begin{agdalisting}
  \ExecuteMetaData[agda/Snippets/Bool.tex]{some-assoc}
\end{agdalisting}
The usefulness of which is dubious, but we will see a more interesting
application soon.
\section{Proof Automation And Search Techniques}
For this prover we will not resort to reflection or similar techniques: instead,
we will trick the type checker to do our automation for us.
This is a relatively common technique, although not so much outside of Agda, so
we will briefly explain it here.

To understand the technique we should first notice that some proof automation
\emph{already} happens in Agda, like the following:
\begin{agdalisting}
  \ExecuteMetaData[agda/Snippets/Bool.tex]{obvious}
\end{agdalisting}
The type checker does not require us to manually explain each step of evaluation
of
\(\AgdaInductiveConstructor{true}\;\AgdaFunction{∧}\;\AgdaInductiveConstructor{false}\).
While it's not a particularly impressive example of automation, it does nonetheless
demonstrate a principle we will exploit: closed terms will compute to a normal
form if they're needed to type check.
The type checker will perform \(\beta\)-reduction as much as it can.

So our task is to rewrite proof obligations like the one in
Equation~\ref{bool-assoc-auto-proof} into ones which can reduce completely. 
As it turns out, we have already described the type of proofs which can ``reduce
completely'': \emph{decidable} proofs.
If we have a decision procedure over some proposition \(P\) we can run that
decision during type checking, because the decision procedure itself is a proof
that the decision will terminate.
In code, we capture this idea with the following pair of functions:
\begin{multicols}{2}
  \begin{agdalisting}
    \ExecuteMetaData[agda/Snippets/Bool.tex]{is-true}
  \end{agdalisting} \columnbreak
  \begin{agdalisting}
    \ExecuteMetaData[agda/Snippets/Bool.tex]{from-true}
  \end{agdalisting}
\end{multicols}
The first is a function which derives a type from whether a decision is
successful or not.
This function is important because if we use the output of this type at any
point we will effectively force the unifier to run the decision computation.
The second takes---as an implicit argument---an inhabitant of the type generated
from the first, and uses it to prove that the decision can only be true, and the
extracts the resulting proof from that decision.
All in all, we can use it like this:
\begin{agdalisting}
  \ExecuteMetaData[agda/Snippets/Bool.tex]{extremely-obvious}
\end{agdalisting}
This technique will allow us to automatically compute any decidable predicate.
\section{Omniscience}
So we now know what is needed of us for proof automation: we need to take our
proofs and make them decidable.
In particular, we need to be able to ``lift'' decidability back over a
function arrow.
For instance, given \(x\), \(y\), and \(z\) we already have
\(\AgdaDatatype{Dec}\;((x\;\AgdaFunction{∧}\;y)\;\AgdaFunction{∧}\;z\;\AgdaFunction{≡}\;x\;\AgdaFunction{∧}\;(y\;\AgdaFunction{∧}\;z))\)
(because equality over booleans is decidable).
In order to turn this into a proof that \AgdaFunction{∧} is associative we need
\(\AgdaDatatype{Dec}\;(\forall \; x \; y \; z \rightarrow (x\;\AgdaFunction{∧}\;y)\;\AgdaFunction{∧}\;z\;\AgdaFunction{≡}\;x\;\AgdaFunction{∧}\;(y\;\AgdaFunction{∧}\;z))\).
The ability to do this is described formally by the notion of
``Exhaustibility''.
\begin{agdalisting}
  \ExecuteMetaData[agda/Relation/Nullary/Omniscience.tex]{exhaustible}
\end{agdalisting}
We say a type \(A\) is exhaustible if, for any decidable predicate \(P\) on
\(A\), the universal quantification of the predicate is decidable.

This property of \AgdaDatatype{Bool} would allow us to automate the proof of
associativity, but it is in fact not strong enough to find individual
representatives of a type which support some property.
For that we need the more well-known related property of
\emph{omniscience}.
\begin{agdalisting}
  \ExecuteMetaData[agda/Relation/Nullary/Omniscience.tex]{omniscient}
\end{agdalisting}
The ``limited principle of omniscience''
\cite{bishopFoundationsConstructiveAnalysis1967} is a classical principle which
says that omniscience holds for all sets.
It doesn't hold constructively, of course: it lies a little bit below LEM in
terms of its non-constructiveness, given that it can be derived from LEM but LEM
cannot be derived from it.

Omniscience implies exhaustibility: we can use the usual rule of \todo{what is
  this called/from again?}
\begin{equation}
  \neg \exists x. P(x) \iff \forall x. \neg P(x)
\end{equation}
to turn omniscience for some predicate \(P\) into exhaustibility for some
predicate \(\neg \neg P\).
Usually we don't have double negation elimination constructively, but since
\(P\) is decidable it's actually present in this case:
\begin{agdalisting}
  \ExecuteMetaData[agda/Relation/Nullary/Decidable/Properties.tex]{dec-double-neg-elim}
\end{agdalisting}
All together, this gives us the following proof:
\begin{agdalisting}
  \ExecuteMetaData[agda/Relation/Nullary/Omniscience.tex]{omniscient-to-exhaustible}
\end{agdalisting}

Our focus here is on those types for which omniscience \emph{does} hold,
which includes the (ordered) finite types.
Perhaps surprisingly, it is not \emph{only} finite types which are exhaustible.
Certain infinite types can be
exhaustible~\cite{escardoInfiniteSetsThat2007}, but an exploration of that is
beyond the scope of this work.

All of the finiteness predicates imply exhaustibility.
To prove that fact we'll just show that the Kuratowski finite types are
exhaustible: since it's the weakest predicate, and can be derived from all the
others.
\begin{lemma}
  Kuratowski finiteness implies exhaustibility.
\end{lemma} \todo{Proof?}
Manifest enumerability is similarly the weakest of the ordered predicates, so we
will prove here that it implies omniscience.
\begin{lemma}
  Manifest enumerability implies omniscience.
\end{lemma} \todo{proof?}

Finally, there is a form of omniscience which works with Kuratowski finiteness:
propositional omniscience.
\begin{agdalisting}
  \ExecuteMetaData[agda/Relation/Nullary/Omniscience.tex]{prop-omniscient}
\end{agdalisting}
By truncating the returned \AgdaDatatype{\ensuremath{\Sigma}} we don't reveal
which \(A\) we've chosen which satisfies the predicate: this means that it can
be pulled out of the Kuratowski finite subset without issue.
\begin{agdalisting}
  \ExecuteMetaData[agda/Cardinality/Finite/Kuratowski.tex]{kuratowski-prop-omniscient}
\end{agdalisting}
\section{Countdown}
The Countdown problem~\cite{huttonCountdownProblem2002} is a well-known puzzle
in functional programming (which was apparently turned into a TV show).
As a running example in this paper, we will produce a verified program which
lists all solutions to a given countdown puzzle: here we will briefly explain
the game and our strategy for solving it.

\begin{wrapfigure}{L}[\marginparwidth]{9cm}
  \tikzset{>=Triangle[open]}
  \begin{tikzpicture}
    \node (o1)  at (-1, 1.5) { 1} ;
    \node (o3)  at ( 0, 1.5) { 3} ;
    \node (o7)  at ( 1, 1.5) { 7} ;
    \node (o10) at ( 2, 1.5) {10} ;
    \node (o25) at ( 3, 1.5) {25} ;
    \node (o50) at ( 4, 1.5) {50} ;

    \node(t1) at (-1,  0) {\xcancel{1}}  ;
    \node(t3)   at (0 ,  0) {3}  ;
    \node(t7)   at (1 ,  0) {7}  ;
    \node(t10)  at (2 ,  0) {10} ;
    \node(t25)  at (3 ,  0) {25} ;
    \node(t50)  at (4 ,  0) {50} ;

    \draw [-{Rays[n=4]}] (o1.south)  -- (t1.north) ;
    \draw [->] (o3.south)  -- (t3.north) ;
    \draw [->] (o7.south)  -- (t7.north) ;
    \draw [->] (o10.south) -- (t10.north) ;
    \draw [->] (o25.south) -- (t25.north) ;
    \draw [->] (o50.south) -- (t50.north) ;

    \node(b3)   at (0 , -1.5) {3}  ;
    \node(b7)   at (1 , -1.5) {7}  ;
    \node(b50)  at (2 , -1.5) {50} ;
    \node(b10)  at (3 , -1.5) {10} ;
    \node(b25)  at (4 , -1.5) {25} ;
  
    \draw [->, rounded corners] (t3.south) --  (b3.north) ;
    \draw [->, rounded corners] (t7.south) --  (b7.north) ;
    \draw [->, rounded corners] (t10.south) to[out=-90, in=90] (b10.north) ;
    \draw [->, rounded corners] (t25.south) to[out=-90, in=90] (b25.north) ;
    \draw [->, rounded corners] (t50.south) to[out=-90, in=90] (b50.north) ;

    \node[draw] at (0.5 , -2.25) {$\times$} ;
    \node[draw] at (1.5 , -2.25) {$\times$} ;
    \node[draw] at (2.5 , -2.25) {$-$} ;
    \node[draw] at (3.5 , -2.25) {$-$} ;
  
    \node at (0.5 , -3) {$\times$} ;
    \node at (1.5 , -3) {$\times$} ;
    \node at (2.5 , -3) {$-$} ;
    \node at (3.5 , -3) {$-$} ;

    \node(e3) at (0   , -3) {3}  ;
    \node(e7) at (1   , -3) {7}  ;
    \node(e50) at (2   , -3) {50} ;
    \node(e10) at (3   , -3) {10} ;
    \node(e25) at (4   , -3) {25} ;

    \draw[->] (b3)  -- (e3) ;
    \draw[->] (b7)  -- (e7) ;
    \draw[->] (b50) -- (e50) ;
    \draw[->] (b10) -- (e10) ;
    \draw[->] (b25) -- (e25) ;

    \node(r40)  [circle, inner sep=0pt, fill=white, anchor=center] at (2.5, -4) {40} ;
    \node(r280) [circle, inner sep=0pt, fill=white, anchor=center] at (2  , -5) {280} ;
    \node(r255) [circle, inner sep=0pt, fill=white, anchor=center] at (2.5, -6) {255} ;
    \node(r765) [circle, inner sep=0pt, fill=white, anchor=center] at (2  , -7) {765} ;

    \draw[->] (e50)  to[out=-90, in=90] (r40) ;
    \draw[->] (e10)  to[out=-90, in=90] (r40) ;
    \draw[->] (e25)  to[out=-90, in=90] (r255) ;
    \draw[->] (e7)   to[out=-90, in=90] (r280) ;
    \draw[->] (e3)   to[out=-90, in=90] (r765) ;
    \draw[->] (r255) to[out=-90, in=90] (r765) ;
    \draw[->] (r280) to[out=-90, in=90] (r255) ;
    \draw[->] (r40)  to[out=-90, in=90] (r280) ;

    \node at (6, 0.7) {\parbox{2.2cm}{\subcaption{Filter     \hfill \; \label{countdown-filter}}}} ;
    \node at (6,-0.8) {\parbox{2.2cm}{\subcaption{Permutation\hfill \; \label{countdown-permutation}}}} ;
    \node at (6,-2.3) {\parbox{2.2cm}{\subcaption{Operators  \hfill \; \label{countdown-operators}}}} ;
    \node at (6,-5.3) {\parbox{2.2cm}{\subcaption{Parentheses\hfill \; \label{countdown-parens}}}} ;
  \end{tikzpicture}
  \caption{The components of a transformation which makes up a Countdown
    candidate solution}
  \label{countdown-transform}
\end{wrapfigure}

The idea behind countdown is simple: given a list of numbers, contestants must
construct an arithmetic expression (using a small set of functions) using some
or all of the numbers, to reach some target.
Here's an example puzzle:
\begin{displayquote}
  Using some or all of the numbers 1, 3, 7, 10, 25, and 50 (using each at most
  once), construct an expression which equals 765.
\end{displayquote}
We'll allow the use of \(+\), \(-\), \(\times\), and \(\div\).
The answer is at the bottom of this page\footnotemark.

\footnotetext{\rotatebox[origin=c]{180}{Answer: \(3 \times (7 \times (50 - 10) - 25)\)}}

Our strategy for finding solutions to a given puzzle is to describe precisely
the type of solutions to a puzzle, and then show that that type is finite.
So what is a ``solution'' to a countdown puzzle?
Broadly, it has two parts:
\begin{description}
  \item[A Transformation] from a list of numbers to an expression.
  \item[A Predicate] showing that the expression is valid and evaluates to the
    target.
\end{description}
The first part is described in Figure~\ref{countdown-transform}.


This transformation has four steps.
First (Fig.~\ref{countdown-selection}) we have to pick which numbers we include
in our solution.
We will need to show there are finitely many ways to filter \(n\) numbers.

Secondly (Fig.~\ref{countdown-permutation}) we have to permute the chosen
numbers.
The representation for a permutation is a little trickier to envision: proving
that it's finite is trickier still.
We will need to rely on some of the more involved lemmas later on for this
problem.

The third step (Fig.~\ref{countdown-operators}) is a vector of length \(n\) of finite objects (in this case operators
chosen from \(+\), \(\times\), \(-\), and \(\div\)).
Although it is complicated slightly by the fact that the \(n\) in this
\(n\)-tuple is dependent on the amount of numbers we let through in the filter
in step one.
(in terms of types, that means we'll need a \(\Sigma\) rather than a
\(\times\), explanations of which are forthcoming).

Finally (Fig.~\ref{countdown-parens}), we have to parenthesise the expression in
a certain way.
This can be encapsulated by a binary tree with a certain number of leaves:
proving that that is finite is tricky again.

Once we have proven that there are finitely many transformations for a list of
numbers, we will then have to filter them down to those transformations which
are valid, and evaluate to the target.
This amounts to proving that the decidable subset of a finite set is also
finite.

Finally, we will also want to optimise our solutions and solver: for this we
will remove equivalent expressions, which can be accomplished with quotients.
We have already introduced and described countdown: in this section, we will
fill in the remaining parts of the solver, glue the pieces together, and show
how the finiteness proofs can assist us to write the solver.
\subsection{Finite Vectors}
We'll start with a simple example: for both the selection
(Fig.~\ref{countdown-selection}) and operators (Fig.~\ref{countdown-operators})
section, all we need to show is that a vector of some finite type is itself
finite.
To describe which elements to keep from an \(n\)-element list, so instance, we
only need a vector of Booleans of length \(n\).
Similarly, to pick \(n\) operators requires us only to provide a vector of \(n\)
operators.
And we can prove in a straightforward way that a vector of finite things is
itself finite.
\begin{agdalisting}
  \ExecuteMetaData[agda/Countdown.tex]{vec-fin}
\end{agdalisting}
We've already shown that there are finitely many booleans, the fact that there
are finitely many operators is similarly simple to prove:
\begin{agdalisting}
  \ExecuteMetaData[agda/Countdown.tex]{op-fin}
\end{agdalisting}
\subsection{Finite Permutations}
A more complex, and interesting, step of the transformation is the first step
(Fig.~\ref{countdown-permutation}), where we need to specify the permutation to
apply to the chosen numbers.

Our first attempt at representing permutations might look something like this:
\begin{agdalisting}
  \ExecuteMetaData[agda/Countdown.tex]{wrong-perm}
\end{agdalisting}
the idea is that \(\AgdaDatatype{Perm}\;n\) represents a permutation of \(n\)
things, as a function from positions to positions.
Unfortunately such a simple answer won't work: there are no restrictions on the
operation of the function, so it could (for instance), send more than one input
position into the same output.

What we actually need is not just a function between positions, but an
\emph{isomorphism} between them.
In types:
\begin{agdalisting}
  \ExecuteMetaData[agda/Countdown.tex]{iso-perm}
\end{agdalisting}
Where an isomorphism is defined as follows:
\begin{agdalisting}
  \ExecuteMetaData[agda/Countdown.tex]{isomorphism}
\end{agdalisting}
\todo{Should this isomorphism definition be put earlier in the intro with the
  equivalences etc?}
While it may look complex, this term is actually composed of individual
components we've already proven finite.
First we have \(\AgdaDatatype{Fin}\;n\rightarrow\AgdaDatatype{Fin}\;n\):
functions between finite types are, as we know, finite
(Theorem~\ref{split-enum-pi}).
We take a pair of them: pairs of finite things are \emph{also} finite
(Lemma~\ref{split-enum-sigma}).
To get the next two components we can filter to the subobject: this requires
these predicates to be decidable. \todo{Need to do filter subobject in topos section}
We will construct a term of the following type:
\begin{agdalisting}
  \ExecuteMetaData[agda/Cardinality/Finite/ManifestBishop.tex]{dec-type}
\end{agdalisting}
So can we construct such a term? Yes!

We basically need to construct decidable equality for functions between
\(\AgdaDatatype{Fin}\;n\)s: of course, this decidable equality is provided by
the fact that such functions are themselves finite.

All in all we can now prove that the isomorphism, and by extension the
permutation, is finite:
\begin{agdalisting}
  \ExecuteMetaData[agda/Cardinality/Finite/ManifestBishop.tex]{iso-finite}
\end{agdalisting}

Unfortunately this implementation is too slow to be useful.
As nice and declarative as it is, fundamentally it builds a list of all possible
pairs of functions between \(\AgdaDatatype{Fin}\;n\) and itself (an operation
which takes in the neighbourhood of \(\mathcal{O}(n^n)\) time), and then tests
each for equality (which is likely worse than \(\mathcal{O}(n^2)\) time).
We will instead use a factoriadic encoding: this is a relatively simple encoding
of permutations which will reduce our time to a blazing fast
\(\mathcal{O}(n!)\).
It is expressed in Agda as follows:
\begin{agdalisting}
  \ExecuteMetaData[agda/Countdown.tex]{perm-def}
\end{agdalisting}
It is a vector of positions, each represented with a \(\AgdaDatatype{Fin}\).
Each position can only refer to the length of the tail of the list at that
point: this prevents two input positions mapping to the same output point, which
was the problem with the naive encoding we had previously.
And it also has a relatively simple proof of finiteness:
\begin{agdalisting}
  \ExecuteMetaData[agda/Countdown.tex]{perm-fin}
\end{agdalisting}
\subsection{Parenthesising}
Our next step is figuring out a way to encode the parenthesisation of the
expression (Fig.~\ref{countdown-parens}). \todo{There's no way
  ``parenthesisation'' is a real word}
At this point of the transformation, we already have our numbers picked out, we
have ordered them a certain way, and we have also selected the operators we're
interested in.
We have, in other words, the following:
\begin{equation}
  3 \times 7 \times 50 - 10 - 25
\end{equation}
Without parentheses, however, (or a religious adherence to BOMDAS) this
expression is still ambiguous.
\begin{align}
  3 \times ((7 \times (50 - 10)) - 25) &= 765 \\
  (((3 \times 7) \times 50) - 10) - 25 &= 1015
\end{align}
The different ways to parenthesise the expression result in different outputs
of evaluation.

So what data type encapsulates the ``different ways to parenthesise'' a given
expression?
That's what we will figure out in this section, and what we will prove finite.

One way to approach the problem is with a binary tree.
A binary tree with \(n\) leaves corresponds in a straightforward way to a
parenthesisation of \(n\) numbers (Fig.~\ref{countdown-parens}).
\todo{Tree diagram? Or link to previous tree?}
This doesn't get us much closer to a finiteness proof, however: for that we will
need to rely on \emph{Dyck} words.
\begin{definition}[Dyck words]
  A Dyck word is a string of balanced parentheses.
  In Agda, we can express it as the following:
  \begin{agdalisting}
    \ExecuteMetaData[agda/Dyck.tex]{dyck-def}
  \end{agdalisting}
  A fully balanced string of \(n\) parentheses has the type
  \(\AgdaDatatype{Dyck}\;\AgdaInductiveConstructor{zero}\;n\).
  Here are some example strings:
  \begin{multicols}{2}
    \begin{agdalisting}
      \ExecuteMetaData[agda/Dyck.tex]{dyck-0-2}
    \end{agdalisting}
    \begin{agdalisting}
      \ExecuteMetaData[agda/Dyck.tex]{dyck-0-3}
    \end{agdalisting}
  \end{multicols}
  The first parameter on the type represents the amount of unbalanced closing
  parens, for instance:
  \begin{agdalisting}
    \ExecuteMetaData[agda/Dyck.tex]{dyck-1-2}
  \end{agdalisting}
\end{definition}

Already Dyck words look easier to prove finite than straight binary trees, but
for that proof to be useful we'll have to relate Dyck words and binary trees
somehow.
As it happens, Dyck words of length \(2n\) are isomorphic to binary trees with
\(n-1\) leaves, but we only need to show this relation in one direction: from
Dyck to binary tree.
To demonstrate the algorithm we'll use a simple tree definition:
\begin{agdalisting}
  \ExecuteMetaData[agda/Dyck.tex]{tree-simpl-def}
\end{agdalisting}
The algorithm itself is quite similar to stack-based parsing algorithms.
\begin{agdalisting}
  \ExecuteMetaData[agda/Dyck.tex]{from-dyck}
\end{agdalisting}
\subsection{Putting It All Together}
At this point we have each of the four components of the transformation defined.
From this we can define what an expression is:
\begin{agdalisting}
  \ExecuteMetaData[agda/Countdown.tex]{expr-def}
\end{agdalisting}
Notice that we don't allow expressions with no numbers.

The proof that this type is finite mirrors its definition closely:
\begin{agdalisting}
  \ExecuteMetaData[agda/Countdown.tex]{expr-finite}
\end{agdalisting}
\subsection{Filtering to Correct Expressions}
We now have a way to construct, formally, every expression we can generate from
a given list of numbers.
This is incomplete in two ways, however.
Firstly, some expressions are invalid: we should not, for instance, be able to
divide two numbers which do not divide evenly.
Secondly, we are only interested in those expressions which actually represent
solutions: those which evaluate to the target, in other words.
We can write a function which tells us if both of these things hold for a given
expression like so:
\begin{multicols}{2}
  \begin{agdalisting}
    \ExecuteMetaData[agda/Countdown.tex]{eval}
  \end{agdalisting} \columnbreak
  \begin{agdalisting}
    \ExecuteMetaData[agda/Countdown.tex]{app-op}
  \end{agdalisting}
\end{multicols}
\section{Automating Proofs}

%%% Local Variables:
%%% mode: latex
%%% TeX-master: "../paper"
%%% End:
\chapter{Countdown}
Countdown is a well-known functional programming puzzle (with a spin-off TV show
in the UK), first popularised as a puzzle in which to demonstrate functional
algorithms in \cite{huttonCountdownProblem2002}.
The idea is simple: given a list of numbers, contestants must construct an
arithmetic expression (using a small set of functions) using some or all of the
numbers, to reach some target.

Take the following problem as an example:
\begin{gather*}
  \boxed{1} \boxed{3} \boxed{7} \boxed{10} \boxed{25} \boxed{50} \\
  \boxed{765} \tag{Target}
\end{gather*}
It has the following answer:
\begin{equation}
  3 \times ((7 \times (50 - 10)) - 25)
\end{equation}
Importantly, we do not have to use every number given to get to the target.
For our problem, we will use the operators \(\times\), \(+\), and \(-\).

In this section we will develop a program/proof which can decide countdown
problems totally.
In other words, given a list of numbers and a target, our program will prove
whether a solution exists, and if so, it will provide just that solution.
\section{Classifying The Problem}
So what is a ``solution'' to the countdown problem?
Put simply, it is a way to:
\begin{enumerate}
  \item Arrange some or all of the given numbers into a valid expression
  \item Such that the expression, when evaluated, is equal to the target.
\end{enumerate}


What is the type of the following transformation?

Firstly, we can notice that not every number was used in our solution.
So the transformation must information about what to keep and what to discard on
an item-by-item basis: a subsequence, in other words.


\begin{figure}
  \tikzset{>=Triangle[open]}
  \begin{tikzpicture}
    \node (o1)  at (-1, 1.5) { 1} ;
    \node (o3)  at ( 0, 1.5) { 3} ;
    \node (o7)  at ( 1, 1.5) { 7} ;
    \node (o10) at ( 2, 1.5) {10} ;
    \node (o25) at ( 3, 1.5) {25} ;
    \node (o50) at ( 4, 1.5) {50} ;
  
  
    \node(t1) at (-1,  0) {\xcancel{1}}  ;
    \node(t3)   at (0 ,  0) {3}  ;
    \node(t7)   at (1 ,  0) {7}  ;
    \node(t10)  at (2 ,  0) {10} ;
    \node(t25)  at (3 ,  0) {25} ;
    \node(t50)  at (4 ,  0) {50} ;

  
    \draw [-{Rays[n=4]}] (o1.south)  -- (t1.north) ;
    \draw [->] (o3.south)  -- (t3.north) ;
    \draw [->] (o7.south)  -- (t7.north) ;
    \draw [->] (o10.south) -- (t10.north) ;
    \draw [->] (o25.south) -- (t25.north) ;
    \draw [->] (o50.south) -- (t50.north) ;
  
    \node(b3)   at (0 , -1.5) {3}  ;
    \node(b7)   at (1 , -1.5) {7}  ;
    \node(b50)  at (2 , -1.5) {50} ;
    \node(b10)  at (3 , -1.5) {10} ;
    \node(b25)  at (4 , -1.5) {25} ;
  
    \draw [->, rounded corners] (t3.south) --  (b3.north) ;
    \draw [->, rounded corners] (t7.south) --  (b7.north) ;
    \draw [->, rounded corners] (t10.south) to[out=-90, in=90] (b10.north) ;
    \draw [->, rounded corners] (t25.south) to[out=-90, in=90] (b25.north) ;
    \draw [->, rounded corners] (t50.south) to[out=-90, in=90] (b50.north) ;

  
    \node[draw] at (0.5 , -2.25) {$\times$} ;
    \node[draw] at (1.5 , -2.25) {$\times$} ;
    \node[draw] at (2.5 , -2.25) {$-$} ;
    \node[draw] at (3.5 , -2.25) {$-$} ;
  
    \node at (0.5 , -3) {$\times$} ;
    \node at (1.5 , -3) {$\times$} ;
    \node at (2.5 , -3) {$-$} ;
    \node at (3.5 , -3) {$-$} ;
  

    \node(e3) at (0   , -3) {3}  ;
    \node(e7) at (1   , -3) {7}  ;
    \node(e50) at (2   , -3) {50} ;
    \node(e10) at (3   , -3) {10} ;
    \node(e25) at (4   , -3) {25} ;
  
    \draw[-] (e50.south) -- ++(0.5,-0.75) ;
    \draw[-] (e10.south) -- ++(-1 ,-1.5  ) ;
    \draw[-] (e7.south)  -- ++(1.5,-2.25) ;
    \draw[-] (e3.south)  -- ++(2  , -3 ) ;
    \draw[-] (e25.south) -- ++(-2 , -3 ) ;

    % \node(r40)  [circle, inner sep=0pt, fill=white, anchor=center] at (2.5 , -3.75) {40} ;
    % \node(r280) [circle, inner sep=0pt, fill=white, anchor=center] at (2   , -4.5) {280} ;
    % \node(r255) [circle, inner sep=0pt, fill=white, anchor=center] at (2.5   , -5.25) {255} ;

    % \draw[->] (e50) -- (r40) ;
    % \draw[->] (e10) -- (r40) ;
  
    \node at (2, -6.5) {$765$} ;

    \node at (6, 0.7) {\parbox{3cm}{\subcaption{Filter        \hfill \; \label{countdown-filter}}}} ;
    \node at (6,-0.8) {\parbox{3cm}{\subcaption{Permutation   \hfill \; \label{countdown-permutation}}}} ;
    \node at (6,-2.3) {\parbox{3cm}{\subcaption{Operators     \hfill \; \label{countdown-operators}}}} ;
    \node at (6,-4.8) {\parbox{3cm}{\subcaption{Parenthesising\hfill \; \label{countdown-parens}}}} ;
  \end{tikzpicture}
  \caption{The components of a transformation which makes up a Countdown
    candidate solution}
\end{figure}

\section{Enumerating Combinations and Permutations}
\section{Enumerating Binary Trees}
\section{Filtering Out Invalid Expressions with the Subobject Classifier}
\section{Filtering Out Duplicates with Quotients}


%%% Local Variables:
%%% mode: latex
%%% TeX-master: "../paper"
%%% End:
\chapter{Related Work}
The univalent foundations program is the main basis for this work
\cite{hottbook}.
In particular, our formalisation in section~\ref{topos} relied heavily on
\cite[chapter 10]{hottbook}, and \cite{rijkeSetsHomotopyType2015}, a paper which
contains much of the same material.

Finite sets in a constructive setting has been studied extensively before:
In \cite{coquandConstructivelyFinite2010} four separate predicates for
finiteness were considered (split-enumerable being the only one explored in this
work), and \cite{firsovVariationsNoetherianness2016} explores Noetherianness.
\cite{firsovDependentlyTypedProgramming2015} explored what we have called split
enumerability and manifest Bishop finiteness (although they are stated slightly
differently), and they use these to build a library for proof search.
In \cite{fruminFiniteSetsHomotopy2018} the topic of Kuratowski finite sets in
HoTT is studied extensively: we have focused more on the non-truncated versions
of finiteness (the ``manifest'' predicates), and we have provided the missing
\(\Pi\)-pretopos proof of decidable Kuratowski finite sets.

\cite{iversenUnivalentCategoriesFormalization2018} provided a starting point for
our categorical formalisation: it contains a proof, for instance, that homotopy
sets form a category.


%%% Local Variables:
%%% mode: latex
%%% TeX-master: "../paper"
%%% End:

% \begin{acks}
%   This work has been supported by the Science Foundation Ireland under the
%   following grant: 13/RC/2D94 to Irish Software Research Centre.
% \end{acks}
\bibliography{bibliography}
\end{document} 