\documentclass[small,review]{acmart}

\usepackage{catchfilebetweentags}
\input{agda}

\usepackage{tikz-cd}
\tikzcdset{
  arrow style=tikz,
  diagrams={>={Straight Barb[scale=0.8]}}
}
\usepackage{amsmath}
\usepackage{amsthm}
\renewcommand{\qed}{\hfill\blacksquare}

% Theorem envs without italics
% \theoremstyle{definition}
% \newtheorem{romdefinition}{Definition}
% \newtheorem{romlemma}{Lemma}
% \newtheorem{romtheorem}{Theorem}

% Haskell bind operator >>=
\newcommand\hbind{%
  \ensuremath{\gg\mkern-5.5mu=}%
}
% Haskell apply operator <*>
\newcommand\hap{%
  \ensuremath{\mathbin{<\mkern-9mu*\mkern-9mu>}}%
}
% Haskell alternative operator <|>
\newcommand\halt{%
  \ensuremath{\mathbin{<\mkern-5mu\vert\mkern-5mu>}}%
}
% Haskell fmap operator <$>$
\newcommand\hfmap{%
  \ensuremath{\mathbin{<\mkern-6mu\$\mkern-6mu>}}%
}
% Haskell mappend operator <>
\newcommand\hcmb{%
  \ensuremath{\mathbin{\diamond}}%
}
% Haskell fish operator <><
\newcommand\hcmbin{%
  \ensuremath{\mathbin{\diamond\mkern-7.4mu\rtimes}}%
}
% Haskell Kleisli composition <=<
\newcommand\hkcomp{%
  \ensuremath{\mathbin{<\mkern-10mu=\mkern-7mu<}}
}
% Haskell forall
\newcommand\hforall{%
  \ensuremath{\forall}%
}
% Unique membership \in!
\newcommand\inunique{%
  \mathrel{\in\mkern-6mu\raisebox{-0.5pt}{!}}
}
% Decide membership \in?
\newcommand\decin{%
  \mathrel{\in\mkern-6mu\raisebox{-0.5pt}{?}}
}
% Dot as wildcard character
\newcommand\wc{{\mkern 2mu\cdot\mkern 2mu}}
% Binary operator !
\newcommand\ind{\mathbin{!}}
\usepackage{subcaption}
\usepackage{locallhs2TeX}
\renewcommand{\Varid}[1]{\mathit{#1}}
\renewcommand{\Conid}[1]{\textcolor{OliveGreen}{\textsf{#1}}}
\usepackage{multicol}
\usepackage{todonotes}
\usepackage{tabularx}
\usepackage{pict2e}

% Proper double left paren [( (also ``plano-concave lens'')
\newcommand{\lbparen}{%
  \mathopen{%
    \sbox0{$()$}%
    \setlength{\unitlength}{\dimexpr\ht0+\dp0}%
    \raisebox{-\dp0}{%
      \begin{picture}(.32,1)
        \linethickness{\fontdimen8\textfont3}
        \roundcap
        \put(0,0){\raisebox{\depth}{$($}}
        \polyline(0.32,0)(0,0)(0,1)(0.32,1)
      \end{picture}%
    }%
  }%
}

% Proper double right paren [( (also ``plano-concave lens'')
\newcommand{\rbparen}{%
  \mathclose{%
    \sbox0{$()$}%
    \setlength{\unitlength}{\dimexpr\ht0+\dp0}%
    \raisebox{-\dp0}{%
      \begin{picture}(.32,1)
        \linethickness{\fontdimen8\textfont3}
        \roundcap
        \put(-0.08,0){\raisebox{\depth}{$)$}}
        \polyline(0,0)(0.32,0)(0.32,1)(0,1)
      \end{picture}%
    }%
  }%
}

\usepackage[backend=bibtex]{biblatex}
\addbibresource{bibliography.bib}

\PrerenderUnicode{í}

\usepackage{hyperref}

\author{
  Donnacha Oisín Kidney \and
  Gregory Provan \and
  Nicolas Wu
}
% \affiliation{University College Cork}
% \email{o.kidney@cs.ucc.ie}
% \author{Gregory Provan}
% \affiliation{University College Cork}
% \email{g.provan@cs.ucc.ie}
% \author{Nicolas Wu}
% \affiliation{Imperial College London}
% \email{n.wu@imperial.ac.uk}

\title{Finiteness in Cubical Type Theory}

% \keywords{Agda, Homotopy Type Theory, Cubical Type Theory,
%   Dependent Types, Finiteness, Topos, Kuratowski finite}


\begin{document}
% \setlength{\abovedisplayshortskip}{-\baselineskip}%

\maketitle

% \begin{abstract}
%   We study five different notions of finiteness in Cubical Type Theory and prove
%   the relationship between them.
%   In particular we show that any totally ordered Kuratowski finite type is
%   manifestly Bishop finite.

%   We also prove closure properties for each finite type, and classify them
%   topos-theoretically.
%   This includes a proof that the category of decidable Kuratowski finite sets
%   (also called the category of cardinal finite sets) form a \(\Pi\)-pretopos.

%   We then develop a parallel classification for the countably infinite types, as
%   well as a proof of the countability of \(A^\star\) for a countable type \(A\).

%   We formalise our work in Cubical Agda, where we implement a library for proof
%   search (including combinators for level-polymorphic fully generic currying).
%   Through this library we demonstrate a number of uses for the computational
%   content of the univalence axiom, including searching for and synthesising
%   functions.
% \end{abstract}
\chapter{Introduction}
\section{Foreword}
\todo[inline]{Foreword}
\section{Contributions}
We are interested in constructive notions of finiteness, formalised in Cubical
Type Theory \cite{cohenCubicalTypeTheory2016}.
In this paper we will explore five such notions of finiteness, including their
categorical interpretation, and use them to build a simple proof-search library
facilitated in a fundamental way by univalence.
Along the way we will use the Countdown problem
\cite{huttonCountdownProblem2002} as an example, and provide a program which
produces verified solutions to the puzzle.
We will also briefly examine countability, and demonstrate its parallels and
differences with finiteness.
\subsection{The Varieties of Finiteness}
In Section~\ref{finiteness-predicates} we will explore a number of different
predicates for finiteness.
In contrast to classical finiteness, in a constructive setting there is a wide
variety of predicates which all have some claim to being the formal
interpretation of ``finiteness'' \cite{coquandConstructivelyFinite2010}.
The particular predicates we are interested in are organised in
Figure~\ref{finite-classification}: each arrow in the diagram represents a proof
that one predicate can be derived from another.

\begin{figure*}
  \centering
  \begin{tikzcd}[cramped, row sep=small, column sep=2.3em]
    {} &
    {} \ar[ddddd, dash, start anchor={[yshift=4ex]}, end anchor={[yshift=-4ex]}] &
    \text{Non Discrete} &
    \ar[ddddd, dash, dashed, start anchor={[yshift=4ex]}, end anchor={[yshift=-4ex]}] &
    \text{Discrete} &
    {}
    \\ %%%%%%%%%%%%%%%%%%%%%%%%%%%%%%%%%%%%%%%%%%%%%%%%%%%%%%%%%%%%%%%%%%%%%%%%%
    \ar[rrrr, dash, start anchor={[xshift=-6ex]}, end anchor={[xshift=10ex]}] &
    &
    &
    &
    {}
    &
    {}
    \\ %%%%%%%%%%%%%%%%%%%%%%%%%%%%%%%%%%%%%%%%%%%%%%%%%%%%%%%%%%%%%%%%%%%%%%%%%
    \text{Ordered} &
    &
    \text{\parbox{1.8cm}{\centering Manifest Enumerable}}
    \ar[
      from=rr,
      bend left=15,
      crossing over,
      start anchor=south west,
      end anchor={[yshift=-2ex]east}
      ]
    \ar[
      rr,
      bend left=15,
      "\text{Discrete}" description,
      start anchor={[yshift=2ex]east},
      end anchor=north west
      ]
    \ar[ddd, crossing over]
    &
    &
    \text{Split Enumerable}
      \ar[d, xshift=-1ex, bend right=30]
    &
    \\ %%%%%%%%%%%%%%%%%%%%%%%%%%%%%%%%%%%%%%%%%%%%%%%%%%%%%%%%%%%%%%%%%%%%%%%%%
    &
    &
    &
    &
    \text{Manifest Bishop}
      \ar[u, xshift=1ex, bend right=30]
    &
    \\ %%%%%%%%%%%%%%%%%%%%%%%%%%%%%%%%%%%%%%%%%%%%%%%%%%%%%%%%%%%%%%%%%%%%%%%%%
    \ar[rrrr, dash, dashed, start anchor={[xshift=-6ex]}, end anchor={[xshift=10ex]}] &
    &
    &
    &
    {}
    &
    \\ %%%%%%%%%%%%%%%%%%%%%%%%%%%%%%%%%%%%%%%%%%%%%%%%%%%%%%%%%%%%%%%%%%%%%%%%%
    \text{Unordered} &
    {} &
    \text{Kuratowski}
    \ar[rr, crossing over, bend left=15, "\text{Discrete}" description, start anchor=north east, end anchor=north west]
    \ar[from=rr, crossing over, bend left=15, start anchor=south west, end anchor=south east]
    &
    {}
    &
    \text{Cardinal}
      \ar[from=uu, crossing over]
      \ar[uuu, crossing over, "\text{Ord}" description, end anchor=east, start
      anchor=east, in=-15, out=20]
    &
    \\
    & & & & &
    \\
    & \ar[rrrr, "\text{More Restrictive}" description, end anchor={[xshift=-6ex]}]
    &
    &
    &
    &
    \ar[uuuuuu, "\text{More Informative}" {description, near start}]
    {}
  \end{tikzcd}
  \caption{Classification of finiteness predicates according to whether they are
    discrete (imply decidable equality) and whether they imply a total order.}
  \label{finite-classification}
\end{figure*}%

%%% Local Variables:
%%% mode: latex
%%% TeX-master: "../paper"
%%% End:

These finiteness predicates differ along two main axes: informativeness, and
restrictiveness.
More ``informative'' predicates have proofs which contain extraneous information
other than the finiteness of the underlying type: a proof of split enumerability
(Section~\ref{split-enumerability}), for instance, comes with a strict total
order on the underlying type.
We will prove that a more informative finiteness predicate can be
derived from a less informative one by providing the missing information
(Theorem~\ref{cardinal-to-manifest-bishop}).

The ``restrictiveness'' of a predicate refers to how many types it admits into
its notion of ``finite''.
There are strictly more Kuratowski finite (Section~\ref{kuratowski}) types than
there are Cardinally finite (Section~\ref{cardinal-finiteness}).
We will prove that we can always derive the less restrictive predicate from the
more restrictive one, and that we can go in the other direction by satisfying
the missing requirement (decidable equality in all of these cases).

Proofs coming with extra information is a common theme in constructive
mathematics: often this extra information is in the form of an algorithm which
can do something useful related to the proof itself.
Indeed, our proofs of finiteness here will provide an algorithm to solve the
countdown puzzle.
Occasionally, however, the extra information is undesirable: we may want to
assert the existence of some value \(x : A\) which satisfies a predicate \(P\)
without revealing \emph{which} \(A\) we're referring to.
More concretely, we will need in this paper to prove that two types are in
bijection without specifying a particular bijection.
This facility is provided by Homotopy Type Theory \cite{hottbook} in the form of
propositional truncation, and it is what allows us to prove the bulk of
propositions in this paper.

For each predicate we will also prove its closure properties (i.e. that the
product of two finite sets is finite).
The most significant of these closure proofs is that of closure under \(\Pi\)
(dependent functions) (Theorem~\ref{split-enum-pi}).
\subsection{Toposes and Finite Sets}
In Section~\ref{topos}, we will explore the categorical interpretation of
decidable Kuratowski finite sets.
The motivation here is partially a practical one: by the end of this work we
will have provided a library for proof search over finite types, and the
``language'' of a topos is a reasonable choice for a principled language for
constructing proofs of finiteness in the style of QuickCheck
\cite{claessenQuickCheckLightweightTool2011} generators.

Theoretically speaking, showing that sets in Homotopy Type Theory form a topos
(with some caveats) is an important step in characterising the categorical
implications of Homotopy Type Theory, first proven in
\cite{rijkeSetsHomotopyType2015}.
Our work is a formalisation of this result (and the first such formalisation
that we are aware of).
The proof that decidable Kuratowski finite sets form a \(\Pi\)-pretopos is
additional to that.
\subsection{Countability Predicates}
After the finite predicates, we will briefly look at the infinite countable
types, and classify them in a parallel way to the finite predicates
(Section~\ref{infinite}).
We will see that we lose closure under function arrows, but we gain it under the
Kleene star (Theorem~\ref{split-countability-sigma}).
\subsection{Search}
All of our work is formalised in Cubical Agda
\cite{vezzosiCubicalAgdaDependently2019}: as a result, the constructive
interpretation of each proof is actually a program which can be run on a
computer.
In finiteness in particular, these programs are particularly useful for
exhaustive search.

We will use the countdown problem as a running example throughout the paper: we
will show how to prove that any given puzzle has a finite number of solutions,
and from that we will show how to enumerate those solutions, thereby solving the
puzzle in a verified way.

In Section~\ref{search} we will package up the ``search'' aspect of finiteness
into a library for proof search: similar libraries have been built in
\cite{fruminFiniteSetsHomotopy2018} and
\cite{firsovDependentlyTypedProgramming2015}.
Our library differs from those in three important ways: firstly, it is strictly
more powerful, as it allows for search over function types.
Secondly, finiteness proofs also provide equivalence proofs to any other finite
type: this allows transport of proofs between types of the same cardinality.
Finally, through generic programming we provide a simple syntax for stating
properties which mimics that of QuickCheck.
We also ground the library in the theoretical notions of omniscience.
\section{Countdown}
Countdown is a well-known functional programming puzzle (later turned into a TV
show in the UK), first popularised as a puzzle in which to demonstrate
functional algorithms in \cite{huttonCountdownProblem2002}.
As a running example in this paper, we will produce a verified program which
lists all solutions to a given countdown puzzle: here we will briefly explain
the game and our strategy for solving it.

The idea behind countdown is simple: given a list of numbers, contestants must
construct an arithmetic expression (using a small set of functions) using some
or all of the numbers, to reach some target.
Here's an example puzzle:
\begin{gather*}
  \boxed{1} \boxed{3} \boxed{7} \boxed{10} \boxed{25} \boxed{50} \\
  \boxed{765} \tag{Target}
\end{gather*}
We'll allow the use of \(\times\), \(+\), and \(-\).
The answer is at the bottom of this page\footnotemark.

\footnotetext{\rotatebox[origin=c]{180}{Answer: \(3 \times (7 \times (50 - 10) - 25)\)}}

Our strategy for finding solutions to a given puzzle is to describe precisely
the type of solutions to a puzzle, and then show that that type is finite.
So what is a ``solution'' to a countdown puzzle?
Broadly, it has two parts:
\begin{description}
  \item[A Transformation] from a list of numbers to an expression.
  \item[A Predicate] showing that the expression is valid and evaluates to the
    target.
\end{description}
The first part is described in Figure~\ref{countdown-transform}.

\input{figures/countdown-transformation}

This transformation has four steps.
First (Fig.~\ref{countdown-selection}) we have to pick which numbers we include
in our solution.
The representation of this transformation should be fairly simple: to pick from
\(n\) objects we can have an \(n\)-tuple of booleans.

Secondly (Fig.~\ref{countdown-permutation}) we have to permute the chosen
numbers.
The representation for a permutation is a little trickier to envision: proving
that it's finite is trickier still.
We will need to rely on some of the more involved lemmas later on for this
problem.

The third step (Fig.~\ref{countdown-operators}) is quite like the first: it's
just another \(n\)-tuple of finite objects (in this case operators chosen from
\(+\), \(\times\), \(-\), and \(\div\)).
Although it is complicated slightly by the fact that the \(n\) in this
\(n\)-tuple is dependent on the number of trues in the previously chosen tuple
of booleans (in terms of types, that means we'll need a \(\Sigma\) rather than a
\(\times\), explanations of which are forthcoming).

Finally (Fig.~\ref{countdown-parens}), we have to parenthesise the expression in
a certain way.
This can be encapsulated by a binary tree with a certain number of leaves:
proving that that is finite is tricky again.

Once we have proven that there are finitely many transformations for a list of
numbers, we will then have to filter them down to those transformations which
are valid, and evaluate to the target.
This amounts to proving that the decidable subset of a finite set is also
finite.

Finally, we will also want to optimise our solutions and solver: for this we
will remove equivalent expressions, which can be accomplished with quotients.
\section{Notation and Background}
We work in Cubical Type Theory \cite{cohenCubicalTypeTheory2016}, specifically
Cubical Agda \cite{vezzosiCubicalAgdaDependently2019}.
Cubical Agda is a dependently-typed functional programming language, based on
Martin-Löf Intuitionistic Type Theory, with a Haskell-like syntax.

Being a dependently-typed language, we'll have to be clear about what we mean
when we say ``type'' in Agda.
\begin{definition}[Type]
  We use \(\AgdaDatatype{Type}\) to denote the universe of (small) types.
  The universe level is denoted with a subscript number, starting at 0.
  ``Type families'' are functions into \(\AgdaDatatype{Type}\).
\end{definition}

The are two broad ways to define types in Agda: as an inductive
\(\AgdaKeyword{data}\) type, similar to data type definitions in Haskell, or as
a \(\AgdaKeyword{record}\).
Here we'll define the basic type formers used in MLTT.
\begin{definition}[Basic Types]
  The three basic types---often called 0, 1, and 2 in MLTT---here will be
  denoted with their more common names: \(\bot\), \(\top\), and
  \(\mathbf{Bool}\), respectively.
  \begin{multicols}{3} \centering
    \begin{agdalisting}
      \ExecuteMetaData[agda/Snippets/Introduction.tex]{bot}
    \end{agdalisting}

    \begin{agdalisting}
      \ExecuteMetaData[agda/Snippets/Introduction.tex]{top}
    \end{agdalisting}

    \begin{agdalisting}
      \ExecuteMetaData[agda/Snippets/Introduction.tex]{bool}
    \end{agdalisting}
  \end{multicols}
\end{definition}
\begin{definition}[The Dependent Sum]
  Dependent sums are denoted with the usual \(\Sigma\) symbol, and has the
  following definition in Agda:

  \begin{center}
    \begin{agdalisting}
      \ExecuteMetaData[agda/Snippets/Introduction.tex]{sigma}
    \end{agdalisting}
  \end{center}
  We will use different notations to refer to this type depending on the
  setting.
  The following four expressions all denote the same type:

  \begin{multicols}{4} \centering
    \begin{agdalisting}
      \ExecuteMetaData[agda/Snippets/Introduction.tex]{sigma-syntax-1}
    \end{agdalisting}

    \begin{agdalisting}
      \ExecuteMetaData[agda/Snippets/Introduction.tex]{sigma-syntax-2}
    \end{agdalisting}

    \begin{agdalisting}
      \ExecuteMetaData[agda/Snippets/Introduction.tex]{sigma-syntax-3}
    \end{agdalisting}

    \begin{agdalisting}
      \ExecuteMetaData[agda/Snippets/Introduction.tex]{sigma-syntax-4}
    \end{agdalisting}
  \end{multicols}

  The non-dependent product is a special instance of the dependent.
  We indicate a simple pair of types \(A\) and \(B\) as \(A \times B\).
\end{definition}
\begin{definition}[Dependent Product]
  Dependent products (dependent functions) use the \(\Pi\) symbol.

  The three following expressions all denote the same type:

  \begin{multicols}{3}
    \begin{agdalisting}
      \ExecuteMetaData[agda/Snippets/Introduction.tex]{pi-syntax-1}
    \end{agdalisting}

    \begin{agdalisting}
      \ExecuteMetaData[agda/Snippets/Introduction.tex]{pi-syntax-2}
    \end{agdalisting}

    \begin{agdalisting}
      \ExecuteMetaData[agda/Snippets/Introduction.tex]{pi-syntax-3}
    \end{agdalisting}
  \end{multicols}

  Again, the non-dependent case is a special case of the dependent case.
  Non-dependent functions are denoted with the arrow (\(\rightarrow\)).
\end{definition}

\begin{definition}[Disjoint Union]
  We define disjoint union as an inductive type.
  \begin{agdalisting}
    \ExecuteMetaData[agda/Snippets/Introduction.tex]{disj-union}
  \end{agdalisting}
  It is also expressible with only \(\Sigma\):
  \begin{agdalisting}
    \ExecuteMetaData[agda/Snippets/Introduction.tex]{sigma-disj-union}
  \end{agdalisting}
\end{definition}
\begin{definition}[Equalities, equivalences, and paths] We use the symbol \(\coloneqq\)
  for definitions.
  \(\simeq\) will be used for equivalences, and \(\equiv\) for equalities.
  Of course, we know that \((A \simeq B) \simeq (A \equiv B)\) by univalence,
  so the distinction isn't terribly important in usage: we will only use one
  or the other as a suggestion of how we constructed it or how it is to be
  used.
\end{definition}

\begin{definition}[Path Types] \label{path-types}
  The equality type (which we denote with \(\equiv\)) in CuTT is the type of
  Paths\footnotemark.
  The internal structure of paths is largely irrelevant to us here, as we will
  generally treat \(\equiv\) as a black-box equivalence relation with
  substitution and congruence.
\end{definition}

\footnotetext{
  Actually, CuTT does have an identity type with similar semantics to the
  identity type in MLTT.
  We do not use this type anywhere in our work, however, so we will not consider
  it here.
}
\begin{definition}[Homotopy Levels] \label{homotopy-types}
  Types in HoTT and CuTT are not necessarily sets, as they are in MLTT.
  Some have higher homotopies (paths which aren't unique).
  We actually have a hierarchy of complexity of structure of path spaces in
  types, starting with the contractions \cite[definition 3.11.1]{hottbook}, then
  the mere propositions \cite[definition 3.3.1]{hottbook}, and the sets
  \cite[definition 3.1.1]{hottbook}.
  \begin{alignat}{2}
    &\text{isContr}(A)    &&\coloneqq \Sigma(x : A) , \Pi(y : A) , (x \equiv y) \\
    &\text{isProp}(A)     &&\coloneqq \Pi(x, y : A) , (x \equiv y) \\
    &\text{isSet}(A)      &&\coloneqq \Pi(x, y : A) , \text{isProp}(x \equiv y)
  \end{alignat}
\end{definition}
\begin{definition}[Fibres] \label{fibres}
  A fibre \cite[definition 4.2.4]{hottbook} is defined over some function \(f :
  A \rightarrow B\).
  \begin{equation}
    \text{fib}_f(y) \coloneqq \Sigma(x : A) , (f (x) \equiv y)
  \end{equation}
\end{definition}
\begin{definition}[Equivalences] \label{equivalences}
  We will take contractible maps \cite[definition 4.4.1]{hottbook} as our
  ``default'' definition of equivalences.
  \begin{alignat}{2}
    &\text{isEquiv}(f) &&\coloneqq \Pi(y : B) , \text{isContr}(\text{fib}_f(y)) \label{is-equiv-def} \\
    &A \simeq B        &&\coloneqq \Sigma(f : A \rightarrow B) , \text{isEquiv}(f) \label{equiv-def}
  \end{alignat}
\end{definition}
\begin{definition}[Decidable Types]
  \begin{equation}
    \mathbf{Dec}(A) \coloneqq A \uplus \neg A
  \end{equation}
\end{definition}
\begin{definition}[Discrete Types]
  A discrete type is one with decidable equality.
  \begin{equation}
    \text{Discrete}(A) \coloneqq \Pi(x, y : A) , \mathbf{Dec}(x \equiv y)
  \end{equation}
  By Hedberg's theorem \cite{hedbergCoherenceTheoremMartinLof1998} any discrete
  type is a set.
\end{definition}
\begin{definition}[Higher Inductive Types] \label{HITs}
  Normal inductive types have \emph{point} constructors: constructors which
  construct values of the type.
  Higher Inductive Types (HITs) also have \emph{path} constructors: ways to
  construct paths in the type.
\end{definition}
\begin{definition}[Propositional Truncation] \label{prop-trunc}
  The type \(\lVert A \rVert\) on some type \(A\) is a propositionally truncated
  proof of \(A\) \cite[3.7]{hottbook}.
  In other words, it is a proof that some \(A\) exists, but it does not tell you
  \emph{which} \(A\).

  It is defined as a Higher Inductive Type:
  \begin{equation} {
    \begin{alignedat}{3}
      \lVert A \rVert \coloneqq & \; \lvert \wc \rvert &:& \; A \rightarrow \lVert A \rVert ; \\
                              | & \; \text{squash}     &:& \; \Pi {(x, y : \lVert A \rVert)} , x \equiv y  ; 
    \end{alignedat} }
  \end{equation}
  We will use two eliminators from \(\lVert A \rVert\) in this paper.
  \begin{enumerate}
  \item \label{elim-prop-prop} For any function \(A \rightarrow B\), where
    \(\text{isProp}(B)\), we have a function \(\lVert A \rVert \rightarrow B\).
  \item \label{elim-prop-coh} We can eliminate from \(\lVert A \rVert\) with a
    function \(f : A \rightarrow B\) iff \(f\) ``doesn't care'' about the
    choice of \(A\):
    \[\Pi {(x , y : A)} , f(x) \equiv f(y) \]
    Formally speaking, \(f\) needs to be ``coherently constant''
    \cite{krausGeneralUniversalProperty2015}, and \(B\) needs to be an
    \(n\)-type for some finite \(n\).
  \end{enumerate}
\end{definition}

%%% Local Variables:
%%% mode: latex
%%% TeX-master: "../paper"
%%% End:
\section{Finiteness Predicates} \label{finiteness-predicates}
In this section, we will define and briefly describe each of the five predicates
in Figure~\ref{finite-classification}.
The reason we explore predicates other than our focus (decidable Kuratowski
finiteness) is that we can often prove things like closure much more readily on
the simpler predicates.
The relations (which we will prove in the next section) then allow us to
transfer those proofs.
\subsection{Split Enumerability}
\begin{definition}[Split Enumerable Set] \label{split-enum-def}
  \begin{equation} \label{split-enum-def-eqn}
    \mathcal{E}!(A) \coloneqq \Sigma {(\mathit{xs} : \textbf{List}(A))} , \Pi {(x : A)} , x \in \mathit{xs}
  \end{equation}
  We call the first component of this pair the ``support'' list, and the second
  component the ``cover'' proof.
  An equivalent version of this predicate was called \verb+Listable+ in
  \cite{firsovDependentlyTypedProgramming2015}.
\end{definition}

We used some extra types in the above definition, which we will define here:
\begin{definition}[Containers] \label{container-def}
  A container \cite{abbottContainersConstructingStrictly2005} is a pair
  \(S , P\) where \(S\) is a type, the elements of which are called
  the \emph{shapes} of the container, and \(P\) is a type family on \(S\), where
  the elements of \(P(s)\) are called the \emph{positions} of a container.
  We ``interpret'' a container into a functor defined like so:
  \begin{equation} \label{container-interp}
    \llbracket S , P \rrbracket(A) \coloneqq \Sigma {(s : S)} , \left( P(s) \rightarrow A \right)
  \end{equation}
  Membership of a container can be defined like so:
  \begin{equation} \label{container-membership}
    x \in \mathit{xs} \coloneqq \text{fib}_{\text{snd}(\mathit{xs})}(x)
  \end{equation}
\end{definition}
\begin{definition}[\(\mathbf{List}\)] \label{List}
  \begin{equation}
    \mathbf{List} \coloneqq \llbracket \mathbb{N} , \mathbf{Fin} \rrbracket
  \end{equation}
\end{definition}
\begin{definition}[\(\mathbf{Fin}\)] \label{Fin}
  \(\mathbf{Fin}(n)\) is the type of natural numbers smaller than \(n\). We
  define it the standard way:
  \begin{equation}
    \begin{alignedat}{3}
      & \mathbf{Fin}(0)     && \coloneqq \bot ; \\
      & \mathbf{Fin}(n + 1) && \coloneqq \top \uplus \mathbf{Fin}(n) ;
    \end{alignedat}
  \end{equation} 
\end{definition}

We tend to prefer list-based definitions of finiteness, rather than ones based
on bijections or surjections.
This is purely a matter of perspective, however: the definition above is
precisely equivalent to a split surjection from a finite prefix of the natural
numbers.
\begin{definition}[Surjections] \label{surjections}
  We define both surjections and \emph{split} surjections here \cite[definition
  4.6.1]{hottbook}.
  \begin{alignat}{3}
    &\text{surj}(f)             &&\coloneqq \Pi(y : B) , \lVert \text{fib}_f(y) \rVert \\
    &A \twoheadrightarrow B     &&\coloneqq \Sigma (f : A \rightarrow B) , \text{surj}(f) \label{surj-arrow-eqn} \\
    &\text{sp-surj}(f)          &&\coloneqq \Pi(y : B) , \text{fib}_f(y) \label{sp-surj-eqn} \\
    &A \twoheadrightarrow! \; B &&\coloneqq \Sigma (f : A \rightarrow B) , \text{sp-surj}(f) \label{sp-surj-arrow-eqn}
  \end{alignat}
\end{definition}
\begin{lemma} \label{split-enum-is-split-surj}
  \begin{equation}
    \mathcal{E}!(A) \simeq \Sigma (n : \mathbb{N}) , \left( \mathbf{Fin}(n) \twoheadrightarrow ! \; A \right)
  \end{equation}
\end{lemma}
\begin{proof}
  \begin{align*}
     \mathcal{E}!(A) &
    \simeq \Sigma (\mathit{xs} : \textbf{List}(A)) , \Pi {(x : A)} , x \in \mathit{xs}
    && \text{def.~\ref{split-enum-def} }(\mathcal{E}!)
    \\
    & \simeq \Sigma (\mathit{xs} : \textbf{List}(A)) , \Pi {(x : A)} , \text{fib}_{\text{snd}(\mathit{xs})}(x)
    && \text{eqn.~\ref{container-membership} } (\in)
    \\
    & \simeq \Sigma (\mathit{xs} : \textbf{List}(A)) , \text{sp-surj}(\text{snd}(\mathit{xs}))
    && \text{eqn.~\ref{sp-surj-eqn} (sp-surj)}
    \\
    & \simeq \Sigma (\mathit{xs} : \llbracket \mathbb{N} , \mathbf{Fin} \rrbracket (A)) , \text{sp-surj}(\text{snd}(\mathit{xs}))
    && \text{def.~\ref{List} } (\mathbf{List})
    \\
    & \simeq \Sigma (\mathit{xs} : \Sigma (n : \mathbb{N}) , \Pi (i : \mathbf{Fin}(n)) , A) , \text{sp-surj}(\text{snd}(\mathit{xs}))
    && \text{eqn.~\ref{container-interp} } (\llbracket \wc \rrbracket)
    \\
    & \simeq \Sigma (n : \mathbb{N}) , \Sigma (f : \mathbf{Fin}(n) \rightarrow A) , \text{sp-surj}(f)
    && \text{Reassociation of } \Sigma
    \\
    & \simeq \Sigma (n : \mathbb{N}) , ( \mathbf{Fin}(n) \twoheadrightarrow ! \; A )
    && \text{eqn.~\ref{sp-surj-arrow-eqn} } (\twoheadrightarrow!)
  \end{align*}
\end{proof}
\begin{agdalisting}
In our formalisation, the proof is a single line:
\ExecuteMetaData[agda/Cardinality/Finite/SplitEnumerable.tex]{is-split-inj}
The only step which isn't definitional equality is the reassociation of
\(\Sigma\).
\ExecuteMetaData[agda/Data/Sigma/Properties.tex]{reassoc}
\end{agdalisting}

Split enumerability implies decidable equality on the underlying type.
To prove this, we will make use of the following lemma, proven in the
formalisation:
\begin{lemma} \label{discrete-surj}
  \begin{equation}
    \frac{
        A \twoheadrightarrow! \; B \; \; \; \text{Discrete}(A)
      }{
       \text{Discrete}(B) 
      }
  \end{equation}
\end{lemma}
\begin{lemma} \label{split-enum-discrete}
  Every split enumerable type is discrete.
\end{lemma}
\begin{proof}
  Let \(A\) be a split enumerable type.
  By lemma~\ref{split-enum-is-split-surj}, there is a surjection from
  \(\mathbf{Fin}(n)\) for some \(n\).
  Also, we know that \(\mathbf{Fin}(n)\) is discrete (proven in our
  formalisation).
  Therefore, by lemma~\ref{discrete-surj}, \(A\) is discrete.
\end{proof}
\subsection{Manifest Bishop Finiteness}
\begin{definition}[Manifest Bishop Finiteness]
  \begin{equation} \label{bish-def}
    \mathcal{B}(A) \coloneqq \Sigma {(\mathit{xs} : \textbf{List}(A))} , \Pi {(x : A)} , x \inunique xs
  \end{equation}
\end{definition}
The only difference between manifest Bishop finiteness and split enumerability
is the membership term: here we require unique membership (\(\inunique\)),
rather than simple membership (\(\in\)).
\begin{definition}[Unique Membership] \label{uniq-memb-def}
  \begin{equation}
    x \inunique \mathit{xs} \coloneqq \text{isContr}(x \in \mathit{xs})
  \end{equation}
\end{definition}

We use the word ``manifest'' here to distinguish from another common
interpretation of Bishop finiteness, which we have called cardinal finiteness in
this paper.
The ``manifest'' refers to the fact that we have a concrete, non-truncated list
of the elements in the proof.

Where split enumerability was the enumeration form of a split surjection from
\(\mathbf{Fin}\), manifest Bishop finiteness is the enumeration form of an
\emph{equivalence} with \(\mathbf{Fin}\).
\begin{lemma} \label{bishop-equiv}
  \begin{equation}
    \mathcal{B}(A) \simeq \Sigma {(n : \mathbb{N})} , \left( \mathbf{Fin}(n) \simeq A \right)
  \end{equation}
\end{lemma}
This proof is effectively the same as that of
lemma~\ref{split-enum-is-split-surj}.
\subsection{Cardinal Finiteness}
Each finiteness predicate so far has contained an \emph{ordering} of the
underlying type.
For our purposes, this is too much information: it means that when constructing
the ``category of finite sets'' later on, instead of each type having one
canonical representative, it will have \(n!\), where \(n\) is the cardinality of
the type\footnotemark.

\footnotetext{
  We actually do get a category (a groupoid, even) from manifest Bishop
  finiteness \cite{yorgeyCombinatorialSpeciesLabelled2014}: it's the groupoid of
  finite sets equipped with a linear order, whose morphisms are order-preserving
  bijections.
  We do not explore this particular construction in any detail.
}

To remedy the problem, we will use propositional truncation
(def.~\ref{prop-trunc}).
\begin{definition}[Cardinal Finiteness]
  \begin{equation}
    \mathcal{C}(A) \coloneqq \lVert \mathcal{B}(A) \rVert \simeq \lVert \Sigma(n : \mathbb{N}) , (\mathbf{Fin}(n) \simeq A) \rVert
  \end{equation}
\end{definition}
At first glance, it might seem that we lose any useful properties we could
derive from \(\mathcal{B}\).
Luckily, this is not the case: by eliminator \ref{elim-prop-coh} of
def.~\ref{prop-trunc}, we need only show that the output is uniquely determined.

The following two lemmas are proven in
\cite{yorgeyCombinatorialSpeciesLabelled2014} (Proposition 2.4.9 and 2.4.10,
respectively), in much the same way as we have done here.
Our contribution for this section is simply the formalisation.
\begin{lemma}
  Given a cardinally finite type, we can derive the type's cardinality, as well
  as a propositionally truncated proof of equivalence with \(\textbf{Fin}\)s of
  the same cardinality.
  \begin{equation}
    \mathcal{C}(A) \rightarrow \Sigma {(n : \mathbb{N})} , \lVert \textbf{Fin}(n) \simeq A \rVert
  \end{equation}
\end{lemma}
\begin{lemma} \label{cardinal-finite-discrete}
  Any cardinal-finite set has decidable equality.
\end{lemma}
\subsection{Manifest Enumerability}
\begin{definition}[Manifest Enumerability]
  \begin{equation}
    \mathcal{E}(A) \coloneqq \Sigma {(\mathit{xs} : \mathbf{List}(A))} , \Pi {(x : A)} , \lVert x \in \mathit{xs} \rVert
  \end{equation}
\end{definition}
As with manifest Bishop finiteness, the only difference with this type and split
enumerability is the membership proof: here we have propositionally truncated
it.
This has two effects.
First, it means that this proof represents a true surjection (rather than a
split surjection) from \(\mathbf{Fin}\).
\begin{lemma}
  \begin{equation}
    \mathcal{E}(A) \simeq \Sigma(n : \mathbb{N}) , (\mathbf{Fin}(n) \twoheadrightarrow A)
  \end{equation}
\end{lemma}
The proof for this lemma is similar in structure to
lemma~\ref{split-enum-is-split-surj} and lemma~\ref{bishop-equiv}.

Secondly, it means the predicate does not imply decidable equality.
More significantly, it allows the predicate to be defined over non-set types,
like the circle.
\begin{definition}[\(S^1\)] \label{circle-def}
  The circle, \(S^1\), can be represented in HoTT as a higher inductive type.
  \begin{equation}
    \begin{alignedat}{3}
      S^1 \coloneqq & \; \text{base} &&: S^1 ; \\
      | & \; \text{loop} &&: \text{base} \equiv \text{base} ; 
    \end{alignedat}
  \end{equation}
  We will use it here as an example of a non-set type, i.e. a type for which not
  all paths are equal.
  This also means that it does not have decidable equality.
\end{definition}
\begin{lemma}
  The circle \(S^1\) is manifestly enumerable.
\end{lemma}
\subsection{Kuratowski Finiteness}
The first thing we must define is a representation of subsets.
\begin{definition}[Kuratowski Finite Subset]
  \(\mathcal{K}(A)\) is the type of Kuratowski-finite subsets of \(A\).
  \begin{equation}
    \begin{alignedat}{3}
      \mathcal{K}(A) \coloneqq&
             \; []                &:& \; \mathcal{K}(A) ; \\
      \vert& \; \wc \dblcolon \wc &:& \; A \rightarrow \mathcal{K}(A) \rightarrow \mathcal{K}(A) ; \\
      \vert& \; \text{com}        &:& \; \Pi (x, y: A) , \Pi (\mathit{xs} : \mathcal{K}(A)) ,
                                  x \dblcolon y \dblcolon \mathit{xs} \equiv y \dblcolon x \dblcolon \mathit{xs} ; \\
      \vert& \; \text{dup}        &:& \; \Pi (x : A) , \Pi (\mathit{xs} : \mathcal{K}(A)) , x \dblcolon x \dblcolon \mathit{xs} \equiv x \dblcolon \mathit{xs} ; \\
      \vert& \; \text{trunc}      &:& \; \Pi (\mathit{xs}, \mathit{ys}: \mathcal{K}(A)) , \Pi (p, q : \mathit{xs} \equiv \mathit{ys}) , p \equiv q ;
    \end{alignedat}
  \end{equation}
  We define it as a HIT (definition~\ref{HITs}).
  The first two constructors are point constructors, giving ways to create
  values of type \(\mathcal{K}(A)\).
  They are also recognisable as the two constructors for finite lists, a type
  which represents the free monoid.

  The next two constructors add extra paths to the type: equations that usage of
  the type must obey.
  These extra paths turn the free monoid into the free \emph{commutative} (com)
  \emph{idempotent} (dup) monoid.

  The final constructor enforces that the type \(\mathcal{K}(A)\) must be a set.
\end{definition}
The Kuratowski finite subset is a free join semilattice (or, equivalently, a
free commutative idempotent monoid).
More prosaically, \(\mathcal{K}\) is the abstract data type for finite sets, as
defined in the Boom hierarchy \cite{boomFurtherThoughtsAbstracto1981,
  bunkenburgBoomHierarchy1994}.
However, rather than just being a specification, \(\mathcal{K}\) is fully usable
as a data type in its own right, thanks to HITs.

Other definitions of \(\mathcal{K}\) exist (such as the one in
\cite{fruminFiniteSetsHomotopy2018}) which make the fact that \(\mathcal{K}\) is
the free join semilattice more obvious.
We have included such a definition in our formalisation, and proven it
equivalent to the one above.

Next, we need a way to say that an entire type is Kuratowski finite.
For that, we will need to define membership of \(\mathcal{K}\).
\begin{definition}[Membership of \(\mathcal{K}\)]
  Membership is defined by pattern-matching on \(\mathcal{K}\).
  The two point constructors are handled like so:
  \begin{equation}
    \begin{alignedat}{2}
      x \in&& \; []                      &\coloneqq \bot ; \\
      x \in&& \; y \dblcolon \mathit{ys} &\coloneqq \lVert x \equiv y \uplus x \in \mathit{ys} \rVert ;
    \end{alignedat}
  \end{equation}
  The \(\text{com}\) and \(\text{dup}\) constructors are handled by proving that
  the truncated form of \(\uplus\) is itself commutative and idempotent.
  The type of propositions is itself a set, satisfying the \(\text{trunc}\)
  constructor.
\end{definition}
Finally, we have enough background to define Kuratowski finiteness.
\begin{definition}[Kuratowski Finiteness]
  \begin{equation}
    \mathcal{K}^{f}(A) = \Sigma {(\mathit{xs} : \mathcal{K}(A))} , \Pi (x : A) , x \in \mathit{xs}
  \end{equation}
\end{definition}

We also have the following two lemmas, proven in both
\cite{fruminFiniteSetsHomotopy2018} and our formalisation.
\begin{lemma}
  \(\mathcal{K}^f\) is a mere proposition.
\end{lemma}
\begin{lemma}
  This circle \(S^1\) is Kuratowski finite.
\end{lemma}
\section{Relations Between Each Finiteness Definition} \label{relations}
We will now look at the arrows in figure~\ref{finite-classification}.
\subsection{Split Enumerability and Manifest Bishop Finiteness}
While manifest Bishop finiteness might seem stronger than split enumerability,
it turns out this is not the case.
Both predicates imply the other.
\begin{lemma} \label{manifest-bishop-to-split-enum}
  Any manifest Bishop finite type is split enumerable.
\end{lemma}
\begin{proof}
  To construct a proof of split enumerability from one of manifest Bishop
  finiteness, it suffices to convert a proof of \(x \inunique \mathit{xs}\) to
  one of \(x \in \mathit{xs}\), for all \(x\) and \(\mathit{xs}\).
  Since \(\inunique\) is defined as a contraction of \(\in\), such a conversion
  is simply the \(\text{fst}\) function.
\end{proof}

\begin{lemma} \label{split-enum-to-manifest-bishop}
  Any split enumerable set is manifest Bishop finite.
\end{lemma}
This proof takes significantly more work.
The ``unique membership'' condition in \(\mathcal{B}\) means that we are not
permitted duplicates in the support list.
The first step in the proof, then, is to filter those duplicates out from the
support list of the \(\mathcal{E}!\) proof: we can do this using the decidable
equality provided by \(\mathcal{E}!\) (lemma~\ref{split-enum-discrete}).
From there, all we need to show is that the membership proof carries over
appropriately.
\subsection{Split Enumerability and Manifest Enumerability}
\begin{lemma} \label{split-enum-to-manifest-enum}
  Any split enumerable type is manifestly enumerable.
\end{lemma}
This lemma is proven by truncating the membership proof in split enumerability.

\begin{lemma} \label{manifest-enum-to-split-enum}
  A manifestly enumerable type with decidable equality is split enumerable.
\end{lemma}

\subsection{Manifest Bishop Finiteness and Cardinal Finiteness}
\begin{lemma} \label{manifest-bishop-to-cardinal}
  Any manifest Bishop finite type is cardinal finite.
\end{lemma}
\begin{theorem} \label{cardinal-to-manifest-bishop}
  Any cardinal finite type with a total order is Bishop finite.
\end{theorem}
The proof for this particular theorem is quite involved in the formalisation, so
we only give its sketch here.
First, note that we actually convert to manifest enumerability first: this can
be converted to split enumerability with decidable equality, which is provided
by cardinal finiteness.

Next, we define permutations.
\begin{definition}[List Permutations]
  Two lists are permutations of each other if their membership proofs are all
  equivalent\footnotemark \cite{danielssonBagEquivalenceProofRelevant2012}.
  \begin{equation}
    \mathit{xs} \leftrightsquigarrow \mathit{ys} = \Pi {(x : A)} , x \in \mathit{xs} \simeq x \in \mathit{ys}
  \end{equation}
\end{definition}

\footnotetext{
  The definition in \cite{danielssonBagEquivalenceProofRelevant2012} and our
  formalisation is slightly different: we say permutations are lists with
  \emph{isomorphic} membership proofs.
  The distinction, as it happens, does not affect our work here.
}

Next, we define a sort function which relies on the provided total order.
We further prove the following fact about this sort function:
\begin{equation}
  \Pi(\mathit{xs}, \mathit{ys} : \mathbf{List}(A)) , \mathit{xs} \leftrightsquigarrow \mathit{ys} \rightarrow \text{sort}(\mathit{xs}) \equiv \text{sort}(\mathit{ys})
\end{equation}

Next, notice that the support lists of any two proofs of manifest Bishop
finiteness must be permutations of each other.
This will allow us to sort the support list of a proof of cardinal finiteness in
a coherently constant (definition~\ref{prop-trunc},
eliminator~\ref{elim-prop-coh}) way, pulling the support list out from the
truncation.
The cover proof emerges naturally from the definition of the permutation.
\subsection{Cardinal Finiteness and Kuratowski Finiteness}
\begin{lemma} \label{cardinal-kuratowski}
  \begin{equation}
    \mathcal{C}(A) \simeq \mathcal{K}^f(A) \times \text{Discrete}(A)
  \end{equation}
\end{lemma}
This proof is constructed by providing a pair of functions: one from
\(\mathcal{C}(A)\) to \(\mathcal{K}^f(A) \times \text{Discrete}(A)\), and one the
other way.
This pair implies an equivalence, because both source and target are
propositions.
The actual functions themselves are proven in our formalisation, as well as in
\cite{fruminFiniteSetsHomotopy2018}.

%%% Local Variables:
%%% mode: latex
%%% TeX-master: "../paper"
%%% End:
\chapter{Topos} \label{topos}
In this section we will examine the categorical interpretation of finite sets.
In particular, we will prove that decidable Kuratowski finite types form a
\(\Pi\)-pretopos.
A lot of the work for this proof has been done already: in
Theorem~\ref{cardinal-kuratowski} we saw that discrete Kuratowski finite types
were equivalent to Cardinally finite types.
We will use the latter definition implementation-wise from now on, as it is
slightly easier to work with: CuTT's transport means we can do this without loss
of generality.

There are two reasons we're interested in the categorical and topos-theoretic
interpretation of finite sets: first, it's an important theoretical grounding
for finite sets, which allows us to understand them in the context of other
set-like constructions.
Secondly, and more practically, the language of a topos is (or in our case the
\(\Pi\)-pretopos) is a common standard framework for doing mathematics
generally.
This makes it a good basis for an API for building QuickCheck-like generators,
for example.
\section{Categories in HoTT}
At first glance, HoTT seems like a perfect setting for category theory: the
univalence axiom identifies isomorphisms with equality, a useful tool for
category theory missing from MLTT.
While this initial impression is broadly true, the construction of categories in
HoTT is unfortunately quite complex and involved (much of the following is a
summary of \citet[chapter 9]{hottbook}).

\todo{references here are tricky, need to disentangle the contributions quite
  precisely}
Much of this section is simply a summary of parts of \citet[chapter
9]{hottbook}.
The formal proofs we provide are part translation of those proofs in that
chapter, part from \cite{iversenFredefoxCat2018}
\cite{huProofrelevantCategoryTheory2020}, and part our own.

First, we need to think about the type of objects and arrows.
We cannot, unfortunately, leave them unrestricted: because of the potential for
higher homotopy in HoTT types \todo{This sentence is a tongue twister}, we have
to restrict the type of arrows to just the sets.
This notion: that of a category with all the usual laws such that arrows are a
set, is called a \emph{precategory}.
\begin{agdalisting}
  \ExecuteMetaData[agda/Categories.tex]{precategory}
\end{agdalisting}
We will use long arrows to refer to morphisms within a category:
\begin{agdalisting}
  \ExecuteMetaData[agda/Categories.tex]{morph-arrow}
\end{agdalisting}

From here, we can define a notion of isomorphisms.
\begin{agdalisting}
  \ExecuteMetaData[agda/Categories.tex]{isomorphism}
\end{agdalisting}
It's a condition on this type which separates the precategories from the
categories: if it satisfies a form of univalence, it the precategory is a full
category.
\begin{agdalisting}
  \ExecuteMetaData[agda/Categories.tex]{cat-univalence}
\end{agdalisting}
\section{The Category of Sets}

\section{Closure}
\todo{This is just the closure bits pulled over from the finiteness predicates
  section. They need to be reworked to fit in here.}
\subsection{Split Enumerability Closure}
Now that we have a suitable definition of finiteness, we will next prove that
some things are finite.
With the most basic simple types out of the way, the obvious next choice is the
(non-dependent) sums and products: \(\uplus\) and \(\times\).
Both of these types can be constructed from the \emph{dependent} sum, however,
so that is the type we will prove finite.
From that we can derive a much wider array of finiteness proofs.

\begin{lemma} \label{split-enum-sigma} \todo{Convert to Agda}
  Split enumerability is closed under \(\Sigma\).
  \begin{equation}
    \frac{
      \AgdaDatatype{\ensuremath{\mathcal{E}!}}\;A \; \; \; (x : A) \rightarrow \AgdaDatatype{\ensuremath{\mathcal{E}!}}(U\;x)
    }{
      \AgdaDatatype{\ensuremath{\mathcal{E}!}}(\AgdaDatatype{\ensuremath{\Sigma [}}\;x \AgdaDatatype{:} A \AgdaDatatype{]} U\;x)
    }
  \end{equation}
\end{lemma}
\begin{proof}
  Let \(A\) be a type which is split enumerable, and \(U\) be a type family over
  \(A\) which is split enumerable at every point.
  Formally, we have the following proofs:
  \begin{align}
    \AgdaDatatype{\ensuremath{\mathcal{E}!}}_A &: \AgdaDatatype{\ensuremath{\mathcal{E}!}}(A) \\
    \AgdaDatatype{\ensuremath{\mathcal{E}!}}_U &: \Pi(x : A) , \AgdaDatatype{\ensuremath{\mathcal{E}!}}(U(x))
  \end{align}

  Our task is to construct a proof of type:
  \begin{equation}
    \AgdaDatatype{\ensuremath{\mathcal{E}!}}(\Sigma(x : A) , U(x))
  \end{equation}
  This proof itself is composed of two components:
  \begin{align}
    \mathit{support} &: \AgdaDatatype{List}(\Sigma(x : A) , U(x)) \\
    \mathit{cover} &: \Pi(x : \Sigma(y : A) , U(y)) , x \in \mathit{support}
  \end{align}
  To construct the support list, we apply the function \(\AgdaDatatype{\ensuremath{\mathcal{E}!}}_U\) to
  every element in the support list of \(\AgdaDatatype{\ensuremath{\mathcal{E}!}}_A\), extract the support
  lists from the resulting finiteness proofs, and concatenate them.


  To prove that this support list does in fact cover the entirety of the type
  \(\Sigma \; A \; U\), we note that any element of type \(\Sigma \; A \; U\)
  must have a first component in the support list of \(\AgdaDatatype{\ensuremath{\mathcal{E}!}}_A\), and its
  second component must be in the result of applying \(\AgdaDatatype{\ensuremath{\mathcal{E}!}}_U\) to that
  first element (since that support list contains every element of type
  \(U(x)\)).
  Therefore, the pair itself must be in our constructed support list.
\end{proof}

This pattern of applying a function to each element in a list and
concatenating the result is of course well-known in functional programming,
and is in fact the pattern that makes lists a monad.
While this insight isn't strictly relevant to our work here, it does mean
the implementation of this function can use Agda's do notation, resulting
in the following extremely clean implementation:
\begin{agdalisting}
  \ExecuteMetaData[agda/Cardinality/Finite/SplitEnumerable.tex]{sup-sigma}
\end{agdalisting}

We now have two components we'll need for the proof that the countdown
transformation is finite.
The component we'll look at is step~\ref{countdown-operators}: selection of the
operators.
We'll first need a type representing the operators available to us.
\begin{agdalisting}
  \ExecuteMetaData[agda/Countdown.tex]{ops-def}
\end{agdalisting}
Proving that this type is finite takes much the same form as the proof of
finiteness for bool.
\begin{agdalisting} \label{op-fin}
  \ExecuteMetaData[agda/Countdown.tex]{op-fin}
\end{agdalisting}

Next, we will need to build a proof of finiteness for vectors of length \(n\).
This uses the proof of finiteness for \(\Sigma\).
\begin{agdalisting}
  \ExecuteMetaData[agda/Countdown.tex]{vec-fin}
\end{agdalisting}

\subsection{Manifest Bishop Closure Under \(\Pi\)}
The glaring omission from our closure proofs under type formers so far has been
the \(\Pi\) type: we have not proved closure under functions, dependent or
otherwise.
In MLTT, this is of course not provable: since all of the finiteness predicates
we have seen so far imply decidable equality, and since we don't have any kind
of decidable equality on functions in MLTT, we know that we won't be able to
show that any kind of function is finite; even one like \(\AgdaDatatype{Bool}
\rightarrow \AgdaDatatype{Bool}\).

CuTT is not so restricted.
Since we have things like function extensionality and transport, we can indeed
prove the finiteness of function types.
Our proof here makes use directly of the univalence axiom, and makes use
furthermore of all the previous closure proofs.
We will prove this closure on split enumerability, rather than on manifest
Bishop finiteness, as it requires slightly less legwork in the proof itself, but
of course we can derive the proof on manifest Bishop finiteness in a few lines.
\begin{theorem} \label{split-enum-pi}
  Split enumerability is closed under dependent functions.
  (\(\Pi\)-types).
  \begin{equation}
    \frac{
      \AgdaDatatype{\ensuremath{\mathcal{E}!}}(A) \; \; \; \Pi {(x : A)} , \AgdaDatatype{\ensuremath{\mathcal{E}!}}\left( U(x) \right)
    }{
      \AgdaDatatype{\ensuremath{\mathcal{E}!}}\left(\Pi {(x : A)} , U(x)\right)
    }
  \end{equation}
\end{theorem}
\begin{proof}
  Let \(A\) be a split enumerable type, and \(U\) be a type family from \(A\),
  which is split enumerable over all points of \(A\).

  As \(A\) is split enumerable, we know that it is also manifestly Bishop finite
  (lemma~\ref{split-enum-to-manifest-bishop}), and consequently we know \(A
  \simeq \AgdaDatatype{Fin}\;n\), for some \(n\) (lemma~\ref{bishop-equiv}).
  We can therefore replace all occurrences of \(A\) with \(\AgdaDatatype{Fin}\;n\),
  changing our goal to:
  \begin{equation}
    \frac{
      \AgdaDatatype{\ensuremath{\mathcal{E}!}}(\AgdaDatatype{Fin}\;n) \; \; \; \Pi (x : \AgdaDatatype{Fin}\;n) , \AgdaDatatype{\ensuremath{\mathcal{E}!}}\left( U(x) \right)
    }{
      \AgdaDatatype{\ensuremath{\mathcal{E}!}}\left(\Pi (x : \AgdaDatatype{Fin}\;n) , U(x)\right)
    }
  \end{equation}
  
  We then define the type of \(n\)-tuples over some type family \(T :
  \AgdaDatatype{Fin}\;n \rightarrow \mathbf{Type}\).
  \begin{equation}
    \begin{alignedat}{3}
      & \mathbf{Tuple}(0, T)   &&\coloneqq \top \\
      & \mathbf{Tuple}(n+1, T) &&\coloneqq T(0) \times \mathbf{Tuple}(n, T \circ \text{suc})
    \end{alignedat}
  \end{equation}
  We can show that this type is equivalent to functions (proven in our formalisation):
  \begin{equation}
    \Pi(x : \AgdaDatatype{Fin}\;n) , U(x) \simeq \mathbf{Tuple}(n, U)
  \end{equation}
  And therefore we can simplify again our goal to the following:
  \begin{equation}
    \frac{
      \AgdaDatatype{\ensuremath{\mathcal{E}!}}(\AgdaDatatype{Fin}\;n) \; \; \; \Pi (x : \AgdaDatatype{Fin}\;n) , \AgdaDatatype{\ensuremath{\mathcal{E}!}}\left( U(x) \right)
    }{
      \AgdaDatatype{\ensuremath{\mathcal{E}!}}\left(\mathbf{Tuple}(n, U)\right)
    }
  \end{equation}
  
  We can prove this goal by showing that \(\mathbf{Tuple}(n, U)\) is split
  enumerable: it is made up of finitely many products of points of \(U\), which
  are themselves split enumerable, and \(\top\), which is also split enumerable.
  Lemma~\ref{split-enum-sigma} shows us that the product of finitely many split
  enumerable types is itself split enumerable, proving our goal.
\end{proof}

This proof can again give us insight into how to prove finiteness of our
countdown transformation.
In the first step (Fig.~\ref{countdown-selection}), we need to select some
numbers from an input list: this can be described with a function of type
\(\AgdaDatatype{Fin}\;n \rightarrow \AgdaDatatype{Bool}\), from indices in the
original list into whether we keep the values or not.
We now know that we can prove functions finite without difficulty: in this case,
we can do it even more simply by proving that an \(n\)-tuple of booleans is
finite.

\subsection{Closure on Cardinal Finiteness}
Since we don't have a function of type \(\mathcal{C}(A) \rightarrow
\AgdaDatatype{\ensuremath{\mathcal{B}}}(A)\), closure proofs on \(\AgdaDatatype{\ensuremath{\mathcal{B}}}\) do not transfer over to
\(\mathcal{C}\) trivially (unlike with \(\AgdaDatatype{\ensuremath{\mathcal{E}!}}\) and \(\AgdaDatatype{\ensuremath{\mathcal{B}}}\)).
The cases for \(\bot\), \(\top\), and \(\AgdaDatatype{Bool}\) are simple to adapt: we
can just propositionally truncate their Bishop finiteness proof.

Non-dependent operators like \(\times\), \(\uplus\), and \(\rightarrow\) are
also relatively straightforward: since \(\lVert {\wc} \rVert\) forms a monad, we
can apply \(n\)-ary functions to values inside it, combining them together.
\begin{agdalisting}
  The fact that \(\lVert \wc \rVert\) forms a monad means that we can lift
  \(n\)-ary functions like the following:
  \ExecuteMetaData[agda/Cardinality/Finite/ManifestBishop.tex]{times-clos-sig}
  Into a truncated context:
  \ExecuteMetaData[agda/Cardinality/Finite/Cardinal.tex]{times-clos-impl}
\end{agdalisting}

Unfortunately, for the dependent type formers like \(\Sigma\) and \(\Pi\), the
same trick does not work.
We have closure proofs like:
\begin{equation}
  \frac{
    \AgdaDatatype{\ensuremath{\mathcal{B}}}(A) \; \; \; \Pi(x : A) , \AgdaDatatype{\ensuremath{\mathcal{B}}}(U(x))
  }{
    \AgdaDatatype{\ensuremath{\mathcal{B}}}(\Pi \; A \; U)
  }
\end{equation}
If we apply the monadic truncation trick we can derive closure proofs like the
following:
\begin{equation}
  \frac{
    \lVert \AgdaDatatype{\ensuremath{\mathcal{B}}}(A) \rVert \; \; \; \lVert \Pi(x : A) , \AgdaDatatype{\ensuremath{\mathcal{B}}}(U(x)) \rVert
  }{
    \lVert \AgdaDatatype{\ensuremath{\mathcal{B}}}(\Pi \; A \; U) \rVert
  }
\end{equation}
However our \emph{desired} closure proof is the following:
\begin{equation}
  \frac{
    \lVert \AgdaDatatype{\ensuremath{\mathcal{B}}}(A) \rVert \; \; \; \Pi(x : A) , \lVert \AgdaDatatype{\ensuremath{\mathcal{B}}}(U(x)) \rVert
  }{
    \lVert \AgdaDatatype{\ensuremath{\mathcal{B}}}(\Pi \; A \; U) \rVert
  }
\end{equation}
They don't match!

The solution would be to find a function of the following type:
\begin{equation}
  (\Pi(x : A) , \lVert \AgdaDatatype{\ensuremath{\mathcal{B}}}(U(x)) \rVert) \rightarrow
  \lVert \Pi(x : A) , \AgdaDatatype{\ensuremath{\mathcal{B}}}(U(x)) \rVert
\end{equation}
However we might be disheartened at realising that this is a required goal: the
above equation is \emph{extremely} similar to the axiom of choice!
\begin{definition}[Axiom of Choice]
  In HoTT, the axiom of choice is commonly defined as follows \cite[lemma
  3.8.2]{hottbook}.
  For any set \(A\), and a type family \(U\) which is a set at all the points
  of \(A\), the following function exists:
  \begin{equation}
    \left( \Pi(x : A) ,  \lVert U(x) \rVert \right) \rightarrow \lVert \Pi(x : A) , U(x) \rVert
  \end{equation}
\end{definition}
Luckily the axiom of choice \emph{does} hold for cardinally finite types,
allowing us to prove the following:
\begin{lemma}
  \begin{equation}
    \mathcal{C}(A) \rightarrow (\Pi(x : A) , \lVert U(x) \rVert) \rightarrow \lVert \Pi(x : A) , U(x) \rVert
  \end{equation}
\end{lemma}
\begin{proof}
  Let \(A\) be a cardinally finite type, \(U\) be a type family on \(A\), and
  \(f\) be a dependent function of type \(\Pi(x : A) , \lVert U(x) \rVert\).

  First, since our goal is itself propositionally truncated, we have access to
  values under truncations: put another way, in the context of proving our goal,
  we can rely on the fact that \(A\) is manifestly Bishop finite.
  Using the same technique as we did in lemma~\ref{split-enum-pi}, we can switch
  from working with dependent functions from \(A\) to \(n\)-tuples, where \(n\)
  is the cardinality of \(A\).
  This changes our goal to the following:
  \begin{equation}
    \mathbf{Tuple}(n, \lVert \wc \rVert \circ U) \rightarrow \lVert \mathbf{Tuple}(n, U) \rVert
  \end{equation}
  Since \(\lVert \wc \rVert\) is closed under finite products, this function
  exists (in fact, using the fact that \(\lVert \wc \rVert\) forms a monad, we
  can recognise this function as \verb+sequenceA+ from the \verb+Traversable+
  class in Haskell).
\end{proof}


This gets us all of the necessary closure proofs on \(\mathcal{C}\).
\section{The Absence of the Subobject Classifier}
\begin{agdalisting} \label{filter-subobject}
  \ExecuteMetaData[agda/Cardinality/Finite/SplitEnumerable.tex]{subobject}
\end{agdalisting}


\section{Closure}
For the first three closure proofs, we only consider split enumerability:
as it is the strongest of the finiteness predicates, we can derive the other
closure proofs from it.

\section{The Category of Finite Sets}
HoTT and CuTT seem to be especially suitable settings for formalisations of
category theory.
The univalence axiom in particular allows us to treat categorical isomorphisms
as equalities, saving us from the dreaded ``setoid hell''.

We follow \cite[chapter 9]{hottbook} in its treatment of
categories in HoTT, and in its proof that sets do indeed form a category.
We will first briefly go through the construction of the category
\(\mathit{Set}\), as it differs slightly from the usual method in type theory.

First, the type of objects and arrows:
\begin{alignat}{3}
  &\text{Obj}_\mathit{Set}      &&\coloneqq \Sigma(x : \mathbf{Type}) , \text{isSet}(x) \\
  &\text{Hom}_\mathit{Set}(x , y) &&\coloneqq  \text{fst}(x) \rightarrow \text{fst}(y)
\end{alignat}
As the type of objects makes clear, we have already departed slightly from the
simpler \(\text{Obj}_\mathit{Set} \coloneqq \mathbf{Type}\) way of doing things:
of course we have to, as HoTT allows non-set types.
Furthermore, after proving the usual associativity and identity laws for
composition (which are definitionally true in this case), we must further show
\(\text{isSet}(\text{Hom}_\mathit{Set}(x,y))\); even then we only have a
precategory.

To show that \(\mathit{Set}\) is a category, we must show that categorical
isomorphisms are equivalent to equivalences.
In a sense, we must give a univalence rule for the category we are working in.

We have provided formal proofs that \(\mathit{Set}\) does indeed form a
category, and the following:
\begin{theorem}[The Category of Finite Sets]
  Finite sets form a category in HoTT when defined like so:
  \begin{equation}
    \begin{alignedat}{3}
      &\text{Obj}_\mathit{FinSet}      &&\coloneqq \Sigma(x : \mathbf{Type}) , \mathcal{C}(x) \\
      &\text{Hom}_\mathit{FinSet}(x , y) &&\coloneqq  \text{fst}(x) \rightarrow \text{fst}(y)
    \end{alignedat}
  \end{equation}
\end{theorem}
\section{The \(\Pi\)-pretopos of Finite Sets}
For this proof, we follow again the proof that \(\mathit{Set}\) forms a \(\Pi
W\)-pretopos from \cite[chapter 10]{hottbook} and
\cite{rijkeSetsHomotopyType2015}.
The difference here is that clearly we do not have access to \(W\)-types, as
they would permit infinitary structures.

We first must show that \(\mathit{Set}\) has an initial object and finite,
disjoint sums, which are stable under pullback.
We also must show that \(\mathit{Set}\) is a regular category with effective
quotients.
We now have a pretopos: the presence of \(\Pi\) types make it a
\(\Pi\)-pretopos.

We have proven the above statements for both \(\mathit{Set}\) and
\(\mathit{FinSet}\).
As far as we know, this is the first formalisation of either.
\begin{theorem} \label{finite-topos}
  The category of finite sets, \(\mathit{FinSet}\), forms a \(\Pi\)-pretopos.
\end{theorem}


%%% Local Variables:
%%% mode: latex
%%% TeX-master: "../paper"
%%% End:
\section{Countably Infinite Types} \label{infinite}
\begin{figure*}
  \centering

  \tikzcdset{
  cramped,
  every matrix/.append style={nodes={font=\scriptsize}},
  row    sep/normal=1em,
  column sep/normal=1em,
  }
  \begin{subfigure}[b]{.45\textwidth}
    \centering
    \begin{tikzcd}
      (1,e) \ar[rdddd, out=-45, in=135] & (2,e) \ar[rdddd, out=-45, in=135] & (3,e) \ar[rdddd, out=-45, in=135] & (4,e) \ar[rdddd, out=-45, in=135] & (5,e)        \\
      (1,d) \ar[u]     & (2,d) \ar[u]    & (3,d) \ar[u]    & (4,d) \ar[u]    & (5,d) \ar[u] \\
      (1,c) \ar[u]     & (2,c) \ar[u]    & (3,c) \ar[u]    & (4,c) \ar[u]    & (5,c) \ar[u] \\
      (1,b) \ar[u]     & (2,b) \ar[u]    & (3,b) \ar[u]    & (4,b) \ar[u]    & (5,b) \ar[u] \\
      (1,a) \ar[u]     & (2,a) \ar[u]    & (3,a) \ar[u]    & (4,a) \ar[u]    & (5,a) \ar[u]
    \end{tikzcd}
    \caption{Depth-First}
    \label{depth-first}
  \end{subfigure} \hfill
  \begin{subfigure}[b]{.45\textwidth}
    \centering
    \begin{tikzcd}
      (1,e) \ar[dr] & (2,e) \ar[dr]  & (3,e) \ar[dr]    & (4,e) \ar[dr] & (5,e) \\
      (1,d) \ar[dr] & (2,d) \ar[dr]  & (3,d) \ar[dr]    & (4,d) \ar[dr] & (5,d) \ar[u] \\
      (1,c) \ar[dr] & (2,c) \ar[dr]  & (3,c) \ar[dr]    & (4,c) \ar[dr] & (5,c) \ar[uul] \\
      (1,b) \ar[dr] & (2,b) \ar[dr]  & (3,b) \ar[dr]    & (4,b) \ar[dr] & (5,b) \ar[uuull, out=130, in=-50] \\
      (1,a) \ar[u]  & (2,a) \ar[uul, out=130, in=-50] & (3,a) \ar[uuull, out=130, in=-50] & (4,a) \ar[uuuulll, out=130, in=-50] & (5,a) \ar[uuuulll, out=130, in=-50]
    \end{tikzcd}
    \caption{Breadth-First}
    \label{breadth-first}
  \end{subfigure}
  \caption{Two possible products for the sets \(\left[ 1 \dots 5 \right]\) and
    \(\left[  a \dots e \right]\)}
  \label{pairings}
\end{figure*}
In the previous sections we saw different flavours of finiteness which were
really just different flavours of relations to \(\mathbf{Fin}\).
In this section we will see that we can construct a similar classification of
relations to \(\mathbb{N}\), in the form of the countably infinite types.
\subsection{Two Countable Types}
The two types for countability we will consider are analogous to split
enumerability and cardinal finiteness.
The change will be a simple one: we will swap out lists for streams.
\begin{definition}[Streams]
  \begin{equation}
    \mathbf{Stream}(A) \coloneqq (\mathbb{N} \rightarrow A)
    \simeq \llbracket \top , \text{const}(\mathbb{N}) \rrbracket
  \end{equation}
\end{definition}
\begin{definition}[Split Countability]
  \begin{equation}
    \aleph_0!(A) \coloneqq \Sigma {(\mathit{xs} : \mathbf{Stream}(A))} , \Pi {(x : A)} , x \in \mathit{xs}
  \end{equation}
\end{definition}
This type is definitionally equal to it surjection equivalent (\(\mathbb{N}
\twoheadrightarrow ! \; A\)).
We construct the unordered, propositional version of the predicate in much the
same way as we constructed cardinal finiteness.
\begin{definition}[Countability]
  \begin{equation}
    \aleph_0(A) \coloneqq \lVert \aleph_0!(A) \rVert
  \end{equation}
\end{definition}

From both of these types we can derive decidable equality.
\begin{lemma}
  Any countable type has decidable equality.
\end{lemma}
\subsection{Closure}
We know that countable infinity is not closed under the exponential (function
arrow), so the only closure we need to prove is \(\Sigma\) to cover all of
what's left.
\begin{theorem} \label{split-countability-sigma}
  Split countability is closed under \(\Sigma\).
\end{theorem}
We know that countable infinity is not closed under the exponential (function
arrow), so the only closure we need to prove is \(\Sigma\) to cover all of
what's left.
To do this we have to take a slightly different approach to the functions we
defined before.
Figure~\ref{pairings} illustrates the reason why: previously, we used the
depth-first product pairing for each support list.
This diverges if the first list is infinite, never exploring anything other than
the first element in the second list.
Instead, we use here the cantor pairing function, which performs a breadth-first
search of the pairings of both lists.

Finally, while we have lost certain closure proofs by allowing for infinite
types, we also \emph{gain} some: in particular the Kleene star.
\begin{theorem}
  Split countability is closed under Kleene star.
  \begin{equation}
    \aleph_0!(A) \rightarrow \aleph_0!(\mathbf{List}(A))
  \end{equation}
\end{theorem}
Again, this proof requires a particular pattern to ensure productivity.
The pattern here builds an intermediate stream \(\mathcal{KV}\) of non-empty
lists from the input support stream \(\mathit{xs}\), which is subsequently
flattened.
\begin{equation}
  \mathcal{KV}_i \coloneqq \left[ \left[ \mathit{xs}_{j - 1} \mid j \in \mathit{js} \right] \mid \mathit{js} \in \mathbf{List}(\mathbb{N}) ; \text{sum}(\mathit{js}) = i ; 0 \notin \mathit{js}  \right]
\end{equation}

%%% Local Variables:
%%% mode: latex
%%% TeX-master: "../paper"
%%% End:
\chapter{Search} \label{search}
A common theme in dependently-typed programming is that proofs of interesting
theoretical things may actually correspond to useful algorithms in some way
related to that thing.
Finiteness is one such case: if we have a proof that a type \(A\) is finite,
we should be able to search through all the elements of that type in a
systematic, automated way.

As it happens, this kind of search is a very common method of proof automation
in dependently-typed languages like Agda.
Proofs of statements like ``the following function is associative''
\begin{agdalisting}
  \ExecuteMetaData[agda/Snippets/Bool.tex]{and-def}
\end{agdalisting}
can be tedious: the associativity proof in particular would take \(2^3 = 8\)
cases.
This is unacceptable!
There are only finitely many cases to examine, after all, and we're
\emph{already} on a computer: why not automate it?
A proof that \(\AgdaDatatype{Bool}\) is finite can get us much of the way to a
library to do just that.

Similar automation machinery can be leveraged to provide search algorithms for
certain ``logic programming''-esque problems.
Using the machinery we will describe in this section, though, when the program
says it finds a solution to some problem that solution will be accompanied by a
formal \emph{proof} of its correctness.

In this section, we will describe the theoretical underpinning and
implementation of a library for proof search over finite domains, based on the
finiteness predicates we have introduced already.
The library will be able to prove statements like the proof of associativity
above, as well as more complex statements.
As a running example for a ``more complex statement'' we will use the countdown
problem, which we have been using throughout: we will demonstrate how to
construct a prover for the existence of, or absence of, a solution to a given
countdown puzzle.

The API for writing searches over finite domains comes from the language of the
\(\Pi\)-pretopos: with it we will show how to compose QuickCheck-like generators
for proof search, with the addition of some automation machinery that allows us
to prove things like the associativity in a couple of lines:
\begin{agdalisting}
  \ExecuteMetaData[agda/Snippets/Bool.tex]{bool-assoc-auto-proof}
\end{agdalisting}

We have already, in previous sections, explored the theoretical implications of
Cubical Type Theory on our formalisation.
With this library for proof search, however, we will see two distinct
\emph{practical} applications which would simply not be possible without
computational univalence.
First and foremost: our proofs of finiteness, constructed with the API we will
describe, have all the power of full equalities.
Put another way any proof over a finite type \(A\) can be lifted to any other
type with the same cardinality.
Secondly our proof search can range over functions: we could, for instance, have
asked the prover to find if \emph{any} function over \(\AgdaDatatype{Bool}\) is
associative, and if so return it to us.
\begin{agdalisting}
  \ExecuteMetaData[agda/Snippets/Bool.tex]{some-assoc}
\end{agdalisting}
The usefulness of which is dubious, but we will see a more interesting
application soon.
\section{Omniscience}
\begin{definition}[Limited Principle of Omniscience]
  For any type \(A\) and predicate \(P\) on \(A\), the limited principle of
  omniscience \cite{myhillErrettBishopFoundations1972} is as follows:
  \begin{equation}
    \left( \Pi {(x : A)} , \mathbf{Dec}(P(x)) \right) \rightarrow \mathbf{Dec} \left( \Sigma {(x : A)} , P(x) \right)
  \end{equation}
  In other words, for any decidable predicate the existential quantification of
  that predicate is also decidable.
\end{definition}
The limited principle of omniscience is non-constructive, but individual types
can themselves satisfy omniscience.
In particular, cardinal finite types are omniscient.

There is also a universal form of omniscience, which we call exhaustibility.
\begin{definition}[Exhaustibility]
  We say a type \(A\) is exhaustible if, for any decidable predicate \(P\) on
  \(A\), the universal quantification of the predicate is decidable.
  \begin{equation}
    \left( \Pi {(x : A)} , \mathbf{Dec}(P(x)) \right) \rightarrow \mathbf{Dec} \left( \Pi {(x : A)} , P(x) \right)
  \end{equation}
\end{definition}

All of the finiteness predicates we have seen imply exhaustibility.
Omniscience is stronger than exhaustibility, as we can derive the latter from
the former.
All of the ordered finiteness predicates imply omniscience.
For the unordered finiteness definitions, we have omniscience for prop-valued
predicates.

\section{Countdown}
\section{Automating Proofs}
One use for above constructions is the automation of certain proofs.
In \cite{firsovDependentlyTypedProgramming2015}, which uses a similar approach
to ours, the \(\AgdaDatatype{Pauli}\) group is used as an example.
\ExecuteMetaData[agda/Data/Pauli.tex]{def}
As \(\AgdaDatatype{Pauli}\) has 4 constructors, \(n\)-ary functions on
\(\AgdaDatatype{Pauli}\) may require up to \(4^n\) cases, making even simple
proofs prohibitively verbose.

The alternative is to derive the things we need from omniscience, itself derived
from a finiteness predicate.
For proof search, the procedure is a well-known one in Agda
\cite{devrieseBrightSideType2011}: we ask for the result of a decision procedure
as an \emph{instance argument}, which will demand computation during
typechecking.
Our addition to this technique is a way to handle multiple arguments based on
fully level-polymorphic dependent currying and uncurrying, building on 
\cite{allaisGenericLevelPolymorphic2019}.
\ExecuteMetaData[agda/Data/Pauli.tex]{assoc-prf}

Finally, we can derive decidable equality on functions over finite types.
We can also use functions in our proof search.
Here, for instance, is an automated procedure which finds the
\(\AgdaFunction{not}\) function on \(\mathbf{Bool}\), given a specification.
\ExecuteMetaData[agda/Data/Pauli.tex]{not-spec}

%%% Local Variables:
%%% mode: latex
%%% TeX-master: "../paper"
%%% End:
\chapter{Countdown}
\input{figures/countdown-transformation}
Countdown is a well-known functional programming puzzle (with a spin-off TV show
in the UK), first popularised as a puzzle in which to demonstrate functional
algorithms in \cite{huttonCountdownProblem2002}.
The idea is simple: given a list of numbers, contestants must construct an
arithmetic expression (using a small set of functions) using some or all of the
numbers, to reach some target.

Take the following problem as an example:
\begin{gather*}
  \boxed{1} \boxed{3} \boxed{7} \boxed{10} \boxed{25} \boxed{50} \\
  \boxed{765} \tag{Target}
\end{gather*}
It has the following answer:
\begin{equation}
  3 \times ((7 \times (50 - 10)) - 25)
\end{equation}
Importantly, we do not have to use every number given to get to the target.
For our problem, we will use the operators \(\times\), \(+\), and \(-\).

In this section we will develop a program/proof which can decide countdown
problems totally.
In other words, given a list of numbers and a target, our program will prove
whether a solution exists, and if so, it will provide just that solution.
\section{Classifying The Problem}
So what is a ``solution'' to the countdown problem?
Put simply, it is a way to:
\begin{enumerate}
  \item Arrange some or all of the given numbers into a valid expression
  \item Such that the expression, when evaluated, is equal to the target.
\end{enumerate}
Our task is to construct a \emph{type} for this transformation.
We will mainly focus on the first of these two concerns: arranging a list of
numbers into an expression using the given rules of Countdown.

We can break this transformation into parts, displayed in
figure~\ref{countdown-transform}.
\section{Enumerating Combinations and Permutations}
\section{Enumerating Binary Trees}
\section{Filtering Out Invalid Expressions with the Subobject Classifier}
\section{Filtering Out Duplicates with Quotients}


%%% Local Variables:
%%% mode: latex
%%% TeX-master: "../paper"
%%% End:
\section{Related Work}
The univalent foundations program is the main basis for this work
\cite{hottbook}.
In particular, our formalisation in section~\ref{topos} relied heavily on
\cite[chapter 10]{hottbook}, and \cite{rijkeSetsHomotopyType2015}, a paper which
contains much of the same material.

Finite sets in a constructive setting has been studied extensively before:
In \cite{coquandConstructivelyFinite2010} four separate predicates for
finiteness were considered (split-enumerable being the only one explored in this
work), and \cite{firsovVariationsNoetherianness2016} explores Noetherianness.
\cite{firsovDependentlyTypedProgramming2015} explored what we have called split
enumerability and manifest Bishop finiteness (although they are stated slightly
differently), and they use these to build a library for proof search.
In \cite{fruminFiniteSetsHomotopy2018} the topic of Kuratowski finite sets in
HoTT is studied extensively: we have focused more on the non-truncated versions
of finiteness (the ``manifest'' predicates), and we have provided the missing
\(\Pi\)-pretopos proof of decidable Kuratowski finite sets.

\cite{iversenUnivalentCategoriesFormalization2018} provided a starting point for
our categorical formalisation: it contains a proof, for instance, that homotopy
sets form a category.


%%% Local Variables:
%%% mode: latex
%%% TeX-master: "../paper"
%%% End:

% \begin{acks}
%   This work has been supported by the Science Foundation Ireland under the
%   following grant: 13/RC/2D94 to Irish Software Research Centre.
% \end{acks}
\bibliography{bibliography}
\end{document} 