\PassOptionsToPackage{dvipsnames}{xcolor}
\documentclass{article}

\usepackage[mono=false]{libertine}
\usepackage{catchfilebetweentags}
\usepackage{mathtools}
\usepackage{mathrsfs}
\usepackage{turnstile}
\usepackage{bbm}
\usepackage[greek, english]{babel}
\usepackage{stmaryrd}
\usepackage{latexsym}
\newcommand\doubleplus{+\kern-1.3ex+\kern0.8ex}
\newcommand\mdoubleplus{\ensuremath{\mathbin{+\mkern-8mu+}}}
% \usepackage[dvipsnames]{xcolor}
\usepackage{relsize}
\newcommand\doublelbrace{%
  {\{\mkern-4mu\lvert}%
}
\newcommand\doublerbrace{%
  {\rvert\mkern-4mu\}}%
}

\makeatletter
\newcommand\incircbin
{%
  \mathpalette\@incircbin
}
\newcommand\@incircbin[2]
{%
  \mathbin%
  {%
    \ooalign{\hidewidth$#1#2$\hidewidth\crcr$#1\bigcirc$}%
  }%
}
\newcommand{\oeq}{\ensuremath{\incircbin{=}}}
\makeatother
\makeatletter
\newcommand\insquarebin
{%
  \mathpalette\@insquarebin
}
\newcommand\@insquarebin[2]
{%
  \mathbin%
  {%
    \ooalign{\hidewidth$#1#2$\hidewidth\crcr$#1\bigbox$}%
  }%
}
\newcommand{\sqtri}{\ensuremath{\insquarebin{\triangle}}}
\makeatother
\newcommand{\circeps}{\ensuremath{\insquarebin{\epsilon}}}
\makeatother
\newcommand{\qeq}{\mathrel{\mathpalette\qeqf\relax}}
\newcommand{\qeqf}[2]{%
  \ooalign{$#1=$\cr
    \hidewidth\raisebox{1ex}{$#1\mkern0mu{\mathsmaller{\mathsmaller{?}}}$}\hidewidth\cr}}
\usepackage{ucs}
\DeclareUnicodeCharacter{8605}{\ensuremath{\leadsto}}
\DeclareUnicodeCharacter{951}{\textgreek{\texteta}}
\DeclareUnicodeCharacter{957}{\textgreek{\textnu}}
\DeclareUnicodeCharacter{961}{\textgreek{\textrho}}
\DeclareUnicodeCharacter{929}{\textgreek{\textRho}}
\DeclareUnicodeCharacter{954}{\textgreek{\textkappa}}
\DeclareUnicodeCharacter{928}{\textgreek{\textPi}}
\DeclareUnicodeCharacter{922}{\textgreek{\textKappa}}
\DeclareUnicodeCharacter{931}{\textgreek{\textSigma}}
\DeclareUnicodeCharacter{916}{\textgreek{\textDelta}}
\DeclareUnicodeCharacter{921}{\textgreek{\textIota}}
\DeclareUnicodeCharacter{8759}{\ensuremath{\dblcolon}}
\DeclareUnicodeCharacter{737}{\ensuremath{^\text{l}}}
\DeclareUnicodeCharacter{691}{\ensuremath{^\text{r}}}
\DeclareUnicodeCharacter{7523}{\ensuremath{_\text{r}}}
\DeclareUnicodeCharacter{8343}{\ensuremath{_\text{l}}}
\DeclareUnicodeCharacter{8718}{\ensuremath{\blacksquare}}
\DeclareUnicodeCharacter{10214}{\ensuremath{\llbracket}}
\DeclareUnicodeCharacter{10215}{\ensuremath{\rrbracket}}
\DeclareUnicodeCharacter{8857}{\ensuremath{\odot}}
\DeclareUnicodeCharacter{8860}{\oeq}
\DeclareUnicodeCharacter{9043}{\ensuremath{\sqtri}}
\DeclareUnicodeCharacter{3870}{\ensuremath{\cong}}
\DeclareUnicodeCharacter{8799}{\ensuremath{\qeq}}
\DeclareUnicodeCharacter{10181}{\ensuremath{\lbag}}
\DeclareUnicodeCharacter{10182}{\ensuremath{\rbag}}
\DeclareUnicodeCharacter{8760}{\ensuremath{-}}
\DeclareUnicodeCharacter{9428}{\ensuremath{\textcircled{ε}}}
\DeclareUnicodeCharacter{8623}{\ensuremath{\lightning}}
\DeclareUnicodeCharacter{10627}{\ensuremath{\doublelbrace}}
\DeclareUnicodeCharacter{10628}{\ensuremath{\doublerbrace}}
\DeclareUnicodeCharacter{10631}{\ensuremath{\llparenthesis}}
\DeclareUnicodeCharacter{10632}{\ensuremath{\rrparenthesis}}
\DeclareUnicodeCharacter{9666}{\ensuremath{\scriptscriptstyle\blacktriangleleft}}
\DeclareUnicodeCharacter{8855}{\ensuremath{\otimes}}
\DeclareUnicodeCharacter{894}{\ensuremath{\fatsemi}}
\DeclareUnicodeCharacter{65373}{\ensuremath{\}}}
\DeclareUnicodeCharacter{65371}{\ensuremath{\{}}
\DeclareUnicodeCharacter{120021}{\ensuremath{\mathcal{F}}}
\DeclareUnicodeCharacter{120017}{\ensuremath{\mathcal{B}}}
\DeclareUnicodeCharacter{8761}{\ensuremath{\coloneq}}
\DeclareUnicodeCharacter{8788}{\ensuremath{\mathrel{\mathop:}=}}
\DeclareUnicodeCharacter{119974}{\ensuremath{\mathcal{K}}}
\DeclareUnicodeCharacter{8492}{\ensuremath{\mathcal{B}}}
\DeclareUnicodeCharacter{8495}{\ensuremath{\mathcal{E}}}
\DeclareUnicodeCharacter{8466}{\ensuremath{\mathcal{L}}}
\DeclareUnicodeCharacter{8779}{\ensuremath{\equiv}}
\DeclareUnicodeCharacter{10626}{\ensuremath{:}}
\DeclareUnicodeCharacter{10629}{\ensuremath{\llparenthesis}}
\DeclareUnicodeCharacter{10630}{\ensuremath{\rrparenthesis}}
\DeclareUnicodeCharacter{9671}{\ensuremath{\Diamond}}
\DeclareUnicodeCharacter{9723}{\ensuremath{\Box}}
\DeclareUnicodeCharacter{9655}{\ensuremath{\triangleright}}
\DeclareUnicodeCharacter{7584}{\ensuremath{^\text{f}}}
\DeclareUnicodeCharacter{8346}{\ensuremath{_\text{p}}}
\DeclareUnicodeCharacter{8621}{\ensuremath{\leftrightsquigarrow}}
\DeclareUnicodeCharacter{12310}{\ensuremath{\lbparen}}
\DeclareUnicodeCharacter{12311}{\ensuremath{\rbparen}}
\DeclareUnicodeCharacter{7470}{\ensuremath{^\text{B}}}
\DeclareUnicodeCharacter{7468}{\ensuremath{^\text{A}}}
\DeclareUnicodeCharacter{7580}{\ensuremath{^\text{c}}}
\usepackage[references]{agda}

\newcommand{\Nat}{\AgdaDatatype{ℕ}}
\newcommand{\Int}{\AgdaDatatype{ℤ}}
\colorlet{AgdaKeyword}{BlueViolet}
\colorlet{AgdaFunction}{OliveGreen}
\colorlet{AgdaDatatype}{OliveGreen}
\colorlet{AgdaPostulate}{OliveGreen}
\colorlet{AgdaPrimitive}{OliveGreen}
\colorlet{AgdaRecord}{OliveGreen}
\colorlet{AgdaRecordtype}{OliveGreen}
\colorlet{AgdaPrimitiveType}{OliveGreen}
\colorlet{AgdaModule}{OliveGreen}
\colorlet{AgdaInductiveConstructor}{Sepia}
\colorlet{AgdaField}{Sepia}
\renewcommand{\AgdaKeywordFontStyle}[1]{\textbf{#1}}
\usepackage{csquotes}


\usepackage{tikz-cd}
\tikzcdset{
  arrow style=tikz,
  diagrams={>={Straight Barb[scale=0.8]}},
}
\usepackage{amsmath}
\usepackage{amsthm}
\renewcommand{\qed}{\ensuremath{\hfill\blacksquare}}

% Boxed theorem envs without italics

\usepackage[framemethod=tikz]{mdframed}

\newtheoremstyle{roman}
{0} % Space above
{0} % Space below
{} % Body font
{} % Indent amount
{\bfseries} % Theorem head font
{.} % Punctuation after theorem head
{.5em} % Space after theorem head
{} % Theorem head spec (can be left empty, meaning `normal')

\theoremstyle{roman}
\newmdtheoremenv[default,nobreak=true]{definition}{Definition}
\newmdtheoremenv[default,nobreak=true]{lemma}{Lemma}
\newmdtheoremenv[default,nobreak=true]{theorem}{Theorem}
\newmdtheoremenv[
  default,
  nobreak=true,
  linecolor=white,
  tikzsetting = {draw=black, line width = 1pt,dotted},
  ]{agdalisting}{Listing}

% Haskell bind operator >>=
\newcommand\hbind{%
  \ensuremath{\gg\mkern-5.5mu=}%
}
% Haskell apply operator <*>
\newcommand\hap{%
  \ensuremath{\mathbin{<\mkern-9mu*\mkern-9mu>}}%
}
% Haskell alternative operator <|>
\newcommand\halt{%
  \ensuremath{\mathbin{<\mkern-5mu\vert\mkern-5mu>}}%
}
% Haskell fmap operator <$>$
\newcommand\hfmap{%
  \ensuremath{\mathbin{<\mkern-6mu\$\mkern-6mu>}}%
}
% Haskell mappend operator <>
\newcommand\hcmb{%
  \ensuremath{\mathbin{\diamond}}%
}
% Haskell fish operator <><
\newcommand\hcmbin{%
  \ensuremath{\mathbin{\diamond\mkern-7.4mu\rtimes}}%
}
% Haskell Kleisli composition <=<
\newcommand\hkcomp{%
  \ensuremath{\mathbin{<\mkern-10mu=\mkern-7mu<}}
}
% Haskell forall
\newcommand\hforall{%
  \ensuremath{\forall}%
}
% Unique membership \in!
\newcommand\inunique{%
  \mathrel{\in\mkern-6mu\raisebox{-0.5pt}{!}}
}
% Decide membership \in?
\newcommand\decin{%
  \mathrel{\in\mkern-6mu\raisebox{-0.5pt}{?}}
}
% Dot as wildcard character
\newcommand\wc{{\mkern 2mu\cdot\mkern 2mu}}
% Binary operator !
\newcommand\ind{\mathbin{!}}
\usepackage{subcaption}
\usepackage{locallhs2TeX}
\renewcommand{\Varid}[1]{\mathit{#1}}
\renewcommand{\Conid}[1]{\textcolor{OliveGreen}{\textsf{#1}}}
\usepackage{multicol}
\usepackage{todonotes}
\usepackage{tabularx}
\usepackage{pict2e}

% Proper double left paren [( (also ``plano-concave lens'')
\newcommand{\lbparen}{%
  \mathopen{%
    \sbox0{$()$}%
    \setlength{\unitlength}{\dimexpr\ht0+\dp0}%
    \raisebox{-\dp0}{%
      \begin{picture}(.32,1)
        \linethickness{\fontdimen8\textfont3}
        \roundcap
        \put(0,0){\raisebox{\depth}{$($}}
        \polyline(0.32,0)(0,0)(0,1)(0.32,1)
      \end{picture}%
    }%
  }%
}

% Proper double right paren [( (also ``plano-concave lens'')
\newcommand{\rbparen}{%
  \mathclose{%
    \sbox0{$()$}%
    \setlength{\unitlength}{\dimexpr\ht0+\dp0}%
    \raisebox{-\dp0}{%
      \begin{picture}(.32,1)
        \linethickness{\fontdimen8\textfont3}
        \roundcap
        \put(-0.08,0){\raisebox{\depth}{$)$}}
        \polyline(0,0)(0.32,0)(0.32,1)(0,1)
      \end{picture}%
    }%
  }%
}

\usepackage[square,numbers]{natbib}

\PrerenderUnicode{í}

\usepackage{hyperref}

% Nice typesetting for the definition of an inductive type.
% For the following Agda-like thing:
%
% data List a where
%   []   : List a
%   (::) : a -> List a -> List a    
%
% write:
% \begin{inductivetype}{\mathbf{List}(A)}
%   \inductivetypeclause{\left[ \right]}{\mathbf{List}(A)}
%   \inductivetypeclause{\wc \dblcolon \wc}{A \rightarrow \mathbf{List}(A) \rightarrow \mathbf{List}(A)}
% \end{inductivetype}

\newif\iffirstclauseofinductivetype
\newenvironment{inductivetype}[1]{%
  \firstclauseofinductivetypetrue
  \begin{equation}
    \begin{alignedat}{3}
      #1
    }{%
    \end{alignedat}
  \end{equation}%
}
\newcommand{\inductivetypeclause}[2]{%
  \iffirstclauseofinductivetype
    \coloneqq
  \else
    \\ \vert
  \fi
  & \; #1 &:&& \; #2 ;
  \firstclauseofinductivetypefalse
}
\newmdenv[
leftmargin = 0pt,
innerleftmargin = 1em,
innertopmargin = 0pt,
innerbottommargin = 0pt,
innerrightmargin = 0pt,
rightmargin = 0pt,
linewidth = 3pt,
topline = false,
rightline = false,
bottomline = false
]{leftbar}

\makeatletter
\renewenvironment{proof}[1][\proofname]{%
  \begin{leftbar}%
  \par
  \pushQED{\qed}%
  \normalfont \topsep6\p@\@plus6\p@\relax
  \list{}{%
    \leftmargin=.5em
    % \rightmargin=\leftmargin
    \settowidth{\itemindent}{\itshape#1}%
    \labelwidth=\itemindent
    % the following line is not needed with amsart, but might be with other classes
    \parsep=0pt \listparindent=\parindent 
  }
\item[\hskip\labelsep
  \itshape
  #1\@addpunct{.}]\ignorespaces
}{%
  \popQED\endlist\@endpefalse
  \end{leftbar}
}
\makeatother

%%% Local Variables:
%%% mode: latex
%%% TeX-master: "paper"
%%% End:
\title{Finiteness and Cardinality in Homotopy Type Theory}
\author{Donnacha Oisín Kidney}
\authorrunning{D. O. Kidney}
\institute{University College Cork \and
  \email{o.kidney@cs.ucc.ie}}
\begin{document}
\maketitle
\begin{abstract}
  We explore five notions of finiteness in Homotopy Type Theory \cite{hottbook}.
  We will classify these notions, and prove their closure properties,
  culminating in a proof that decidable Kuratowski-finite sets
  \cite{kuratowskiNotionEnsembleFini1920} form a topos.

  We then develop a parallel classification for the countably infinite types, as
  well as a proof of the countability of \(A^\star\) for a countable type \(A\).

  We formalise our work in Cubical Agda
  \cite{vezzosiCubicalAgdaDependently2019}, and we implement a library for proof
  search (including combinators for level-polymorphic fully generic currying),
  and demonstrate how it can be used to both prove properties and synthesise
  full functions given desired properties.
\end{abstract}
\section{Introduction}
\subsection{Strong Finiteness Predicates}
\begin{figure*}
  \centering
  \begin{tikzcd}[cramped, row sep=small, column sep=2.3em]
    {} &
    {} \ar[ddddd, dash, start anchor={[yshift=4ex]}, end anchor={[yshift=-4ex]}] &
    \text{Non Discrete} &
    \ar[ddddd, dash, dashed, start anchor={[yshift=4ex]}, end anchor={[yshift=-4ex]}] &
    \text{Discrete} &
    {}
    \\ %%%%%%%%%%%%%%%%%%%%%%%%%%%%%%%%%%%%%%%%%%%%%%%%%%%%%%%%%%%%%%%%%%%%%%%%%
    \ar[rrrr, dash, start anchor={[xshift=-6ex]}, end anchor={[xshift=10ex]}] &
    &
    &
    &
    {}
    &
    {}
    \\ %%%%%%%%%%%%%%%%%%%%%%%%%%%%%%%%%%%%%%%%%%%%%%%%%%%%%%%%%%%%%%%%%%%%%%%%%
    \text{Ordered} &
    &
    \text{\parbox{1.8cm}{\centering Manifest Enumerable}}
    \ar[
      from=rr,
      bend left=15,
      crossing over,
      start anchor=south west,
      end anchor={[yshift=-2ex]east}
      ]
    \ar[
      rr,
      bend left=15,
      "\text{Discrete}" description,
      start anchor={[yshift=2ex]east},
      end anchor=north west
      ]
    \ar[ddd, crossing over]
    &
    &
    \text{Split Enumerable}
      \ar[d, xshift=-1ex, bend right=30]
    &
    \\ %%%%%%%%%%%%%%%%%%%%%%%%%%%%%%%%%%%%%%%%%%%%%%%%%%%%%%%%%%%%%%%%%%%%%%%%%
    &
    &
    &
    &
    \text{Manifest Bishop}
      \ar[u, xshift=1ex, bend right=30]
    &
    \\ %%%%%%%%%%%%%%%%%%%%%%%%%%%%%%%%%%%%%%%%%%%%%%%%%%%%%%%%%%%%%%%%%%%%%%%%%
    \ar[rrrr, dash, dashed, start anchor={[xshift=-6ex]}, end anchor={[xshift=10ex]}] &
    &
    &
    &
    {}
    &
    \\ %%%%%%%%%%%%%%%%%%%%%%%%%%%%%%%%%%%%%%%%%%%%%%%%%%%%%%%%%%%%%%%%%%%%%%%%%
    \text{Unordered} &
    {} &
    \text{Kuratowski}
    \ar[rr, crossing over, bend left=15, "\text{Discrete}" description, start anchor=north east, end anchor=north west]
    \ar[from=rr, crossing over, bend left=15, start anchor=south west, end anchor=south east]
    &
    {}
    &
    \text{Cardinal}
      \ar[from=uu, crossing over]
      \ar[uuu, crossing over, "\text{Ord}" description, end anchor=east, start
      anchor=east, in=-15, out=20]
    &
    \\
    & & & & &
    \\
    & \ar[rrrr, "\text{More Restrictive}" description, end anchor={[xshift=-6ex]}]
    &
    &
    &
    &
    \ar[uuuuuu, "\text{More Informative}" {description, near start}]
    {}
  \end{tikzcd}
  \caption{Classification of finiteness predicates according to whether they are
    discrete (imply decidable equality) and whether they imply a total order.}
  \label{finite-classification}
\end{figure*}%

%%% Local Variables:
%%% mode: latex
%%% TeX-master: "../paper"
%%% End:
We will first explore the ``strong'' notions of finiteness (i.e. those at least
as strong as Kuratowski finiteness \cite{kuratowskiNotionEnsembleFini1920}),
with a special focus on cardinal finiteness (section~\ref{cardinal}), and
manifest enumerability (section~\ref{manifest-enumerability}), which is new, to
our knowledge.

Figure~\ref{finite-classification} organises the predicates according to
their strength.
We will go through each of the predicates, proving how to weaken each (i.e. we
will provide a proof that every cardinally finite type is Kuratowski finite),
and how to strengthen them, given the required property.
In terms of figure~\ref{finite-classification}, this amounts to providing proofs
for each arrow.
Our main proof in this regard is that we can derive manifest Bishop finiteness
from Kuratowski finiteness plus a total order.

We will---through the use of containers
\cite{abbottContainersConstructingStrictly2005}---formally prove the equivalence
these predicates have with the usual function relations i.e. we will show that a
proof of manifest enumerability is precisely equivalent to a surjection from a
finite prefix of the natural numbers.

For each predicate, we will also prove its closure over sums and products in
both dependent and non-dependent forms, if such a closure exists.
This will culminate in our main result for this section: the formal proof that
decidable Kuratowski finite sets form a topos.
\subsection{Infinite Types}
In section~\ref{infinite-cardinalities}, we will extend our study of finite
types to infinite but countable types.
We will see that the classification of finiteness predicates mirrors that of
their countable counterparts, and we will prove closure under the Kleene star.

\subsection{Practical Uses}
Finally, in section~\ref{practical} we will demonstrate some practical uses of
the finiteness proofs in Cubical Agda.
We will show how to use well-known techniques
\cite{firsovDependentlyTypedProgramming2015, devrieseBrightSideType2011} to
automate proofs of functions with finite (and infinite) domains.
We further show how to automate the synthesis of functions from desired
properties.
\section{Split Enumerability} \label{split-enumerable}
We will start with the simplest definition of finiteness: we say a set is
enumerable if there is a list of its elements which contains every element in
the set.

Before giving the definition of the predicate, we will first define lists (and
membership thereof): we have chosen a \emph{container}-based definition for this
work.
\begin{rm-definition}[Container] \label{container-def}
  A container \cite{abbottContainersConstructingStrictly2005} is a pair
  \(S \triangleright P\) where \(S\) is a type, the elements of which are called
  the \emph{shapes} of the container, and \(P\) is a type family on \(S\), where
  the elements of \(P(s)\) are called the \emph{positions} of a container.
  We ``interpret'' a container into a functor defined like so:
  \begin{equation}
    \llbracket S \triangleright P \rrbracket(A) = \Sigma {(s : S)} , \left( P(s) \rightarrow A \right)
  \end{equation}
  Membership of a container can be defined like so:
  \begin{equation}
    x \in \mathit{xs} = \text{fiber}(\text{snd}(\mathit{xs}), x)
  \end{equation}
  Where \(\text{fiber}\) is from \cite[definition 4.2.4]{hottbook}.
\end{rm-definition}
\begin{rm-definition}[Lists]
  \begin{equation}%
    \mathbf{List} = \llbracket \mathbb{N} \triangleright \mathbf{Fin} \rrbracket
  \end{equation}
\end{rm-definition}
\begin{rm-definition}[\mathbf{Fin}]
  \(\mathbf{Fin}(n)\) is the type of natural numbers smaller than \(n\).
  We define it the standard way, where \(\mathbf{Fin}(0) = \bot\) and
  \(\mathbf{Fin}(n + 1) &= \top + \mathbf{Fin}(n)\).
\end{rm-definition}

Internally, in our formalisation, we actually use the standard
inductive definition of lists more often (it tends to work better in more
complex algorithms, and functions on it seems to satisfy the termination
checker more readily).
However, since both types are equivalent, univalence allows us to transport to
whichever representation is more convenient in a given situation.
For the higher-level proofs we present here, though, the container-based
definition greatly simplifies certain steps, which is why we have chosen it as
our representation.

Finally, we can define formally split enumerability.
\begin{rm-definition}[Split Enumerable Set]
  \begin{equation}
    \mathcal{E}!(A) = \Sigma {(\mathit{xs} : \textbf{List}(A))} , \Pi {(x : A)} , x \in xs
  \end{equation}
  We call the first component of this pair the ``support'' list, and the second
  component the ``cover'' proof.
  This definition is what is referred to as \verb+Listable+ in
\end{rm-definition}
\cite{firsovDependentlyTypedProgramming2015}.
\subsection{Split Surjections}
\begin{rm-definition}[Surjections]
  \begin{multicols}{2}
    \begin{equation}
      A \twoheadrightarrow B = \Sigma (f : A \rightarrow B) , \text{surjective}(f)
    \end{equation} \break
    \begin{equation}
      A \twoheadrightarrow! \; B = \Sigma (f : A \rightarrow B) , \text{split surjective}(f)
    \end{equation}
  \end{multicols}
  Where the definitions of surjective are taken from \cite[definition
  4.6.1]{hottbook}.
\end{rm-definition}
\begin{rm-theorem} \label{split-enum-surj}
  Split enumerability is equivalent to a split surjection from a finite prefix
  of the natural numbers.
  \begin{equation}
    \mathcal{E}!(A) \iff \Sigma {(n : \mathbb{N})} , \left( \mathbf{Fin}(n) \twoheadrightarrow ! \; A \right)
  \end{equation}
\end{rm-theorem}
\begin{proof}
  After sufficient inlining, it emerges that our goal is simply a reassociation.
  \ExecuteMetaData[agda/Cardinality/Finite/SplitEnumerable.tex]{split-surj}
  To be clear: in Agda, the proof could simply be \(\AgdaFunction{reassoc}\); we
  have written out the extra lines for clarity alone.
\end{proof}
\begin{rm-lemma} \label{split-enum-discrete}
  Any split enumerable type has decidable equality (is discrete).
\end{rm-lemma}
\begin{proof}
  We use a corollary that if there is a split-surjection from \(A\) to \(B\),
  and \(A\) is discrete, then \(B\) is also discrete.
\end{proof}
By Hedberg's theorem \cite{hedbergCoherenceTheoremMartinLof1998}, we know that
this implies split enumerable types must all be sets.
\subsection{Closure}
In this section we will prove closure under various operations for split
enumerable sets.
We are working towards a topos proof, which requires us to prove closure under
a variety of operations: for now, we only have enough machinery to demonstrate
the semiring operations, and dependent sums.
In order to show closure under exponentials (function arrows), we will need an
equivalence with \(\mathbf{Fin}\), which will be provided in
section~\ref{manifest-bishop}.
\begin{rm-lemma} \label{split-enum-bool-top-bot}
  \(\mathbf{Bool}\), \(\top\), and \(\bot\) are all split enumerable.
\end{rm-lemma}
\begin{proof}
  Each of these types clearly has a finite number of elements (2, 1, and 0,
  respectively), and furthermore has a straightforward enumeration.
\end{proof}
\begin{rm-theorem} \label{split-enum-sigma}
  Split-enumerability is closed under \(\Sigma\).
  \begin{equation} \label{split-enum-sigma-clos-sig}
    \frac{
      \mathcal{E}!(A) \; \; \; \Pi (x : A) , \mathcal{E}!(U(x))
    }{
      \mathcal{E}!(\Sigma (x : A) , U(x))
    }
  \end{equation}
\end{rm-theorem}
\begin{proof}
  To obtain the support list, we concatenate the support lists of all the proofs
  of split-finiteness for \(U\) over the support list of \(E_A\).
  In Agda:
  \ExecuteMetaData[agda/Cardinality/Finite/SplitEnumerable.tex]{sigma-sup}
  ``do-notation'' is available to us as we're working in the list monad.
\end{proof}
Closure under disjoint union and Cartesian product both follow from \(\Sigma\),
as these types can be defined in terms of \(\Sigma\).
\begin{align}
  A \times B &= \Sigma {(x : A)} , B \\
  A + B &= \Sigma {(x : \mathbf{Bool})} , \text{if} \; x \; \text{then} \; A \; \text{else} \; B
\end{align}
\section{Manifest Bishop Finiteness} \label{manifest-bishop}
Where split enumerability was the enumeration form of a split surjection from
\(\mathbf{Fin}\), manifest Bishop finiteness is the enumeration form of an
\emph{equivalence} with \(\mathbf{Fin}\).

\begin{rm-definition}[Manifest Bishop Finiteness]
  \begin{equation}
    \mathcal{B}(A) = \Sigma {(\mathit{xs} : \textbf{List}(A))} , \Pi {(x : A)} , x \inunique xs
  \end{equation}
\end{rm-definition}
The only difference between this predicate and split enumerability is the list
membership term: we use \(\inunique\) here, where \(x \inunique \mathit{xs}\) is
to be read as ``\(x\) occurs exactly once in \(\mathit{xs}\)''.
\begin{rm-definition}[Unique Membership]
  We say an item \(x\) is ``uniquely in'' some container \(\mathit{xs}\) if its
  membership in that list is a \emph{contraction} \cite[definition
  3.11.1]{hottbook}; i.e. its membership proof exists, and all such proofs are
  equal.
  \begin{equation}
    x \inunique \mathit{xs} = \text{isContr}(x \in \mathit{xs})
  \end{equation}
\end{rm-definition}
By the definition of a contraction, we can always recover the underlying
membership proof, meaning that we can always derive split enumerability from
manifest Bishop finiteness.
\subsection{Equivalence}
\begin{rm-lemma} \label{bishop-equiv}
  A proof of manifest Bishop finiteness is equivalent to an equivalence with a
  finite prefix of the natural numbers.
  \begin{equation}
    \mathcal{B}(A) \simeq \Sigma {(n : \mathbb{N})} , \left( \mathbf{Fin}(n) \simeq A \right)
  \end{equation}
\end{rm-lemma}
\begin{proof}
  There are many equivalent definitions of equivalence in HoTT.
  Here we take the version preferred in the Cubical Agda library: contractible
  maps \cite[definition 4.4.1]{hottbook}.
  Because of the parallels between contractible maps and split surjections,
  the proof proceeds much the same as \ref{split-enum-surj}.
  In other words, the definition of Bishop finiteness is itself a reassociation
  of a contractible map.
\end{proof}
\subsection{Relationship to Split Enumerability}
\begin{rm-theorem} \label{split-enum-to-manifest-bishop}
  Any split enumerable set is manifest Bishop finite.
\end{rm-theorem}
\begin{proof}
  From proposition~\ref{split-enum-discrete} we can derive decidable equality on
  \(A\), and using this we can define a function (called \(\text{uniques}\)) which filters
  out duplicates from lists of \(A\)s.
  This gives us our support list.
  To generate the cover proof it suffices now to prove the following:
  \begin{equation}
    \Pi {(x : A)} , \Pi {(\mathit{xs} : \textbf{List}(A))} , x \in \mathit{xs} \rightarrow x \inunique \text{uniques}(\mathit{xs})
  \end{equation}
\end{proof}
\subsection{Closure}
Proving equal strength of split enumerability and manifest Bishop finiteness
allows us to carry all of the previous proofs of closure over to manifest Bishop
finite sets (and vice-versa).
Missing from our previous proofs was a proof of closure of functions.
We remedy that here.
\begin{rm-theorem}
  Manifest bishop finiteness is closed over dependent functions
  (\(\prod\)-types).
  \begin{equation}
    \frac{
      \mathcal{B}(A) \; \; \; \Pi {(x : A)} , \mathcal{B}\left( U(x) \right)
    }{
      \mathcal{B}\left(\Pi {(x : A)} , U(x)\right)
    }
  \end{equation}
\end{rm-theorem}
\begin{proof}
  This proof is essentially the composition of two transport operations, made
  available to us via univalence.
  
  First, we will simplify things slightly by working only with split
  enumerability.
  As this is equal in strength to manifest Bishop finiteness, any closure proofs
  carry over.

  Secondly, we will replace \(A\) in all places with \(\mathbf{Fin}(n)\).
  Since we have already seen an equivalence between these two types, we are
  permitted to transport along these lines.
  This is the first transport operation.

  The bulk of the proof now is concerned with proving the following:
  \begin{equation}
    \left( \Pi {(x : \mathbf{Fin}(n))} , \mathcal{E}!(A(x)) \right) \rightarrow \mathcal{E}!\left( \Pi {(x : \mathbf{Fin}(n))} , A(x) \right)
  \end{equation}
  Our strategy to accomplish this will be to consider functions from
  \(\mathbf{Fin}(n)\) as \(n\)-tuples over some type family \(T : \mathbb{N}
  \rightarrow \text{Type}\).
  \begin{equation}
    \begin{aligned}
      \mathbf{Tuple}(T, 0)   &= \top \\
      \mathbf{Tuple}(T, n+1) &= T(0) \times \mathbf{Tuple}(T \circ \text{suc}, n)
    \end{aligned}
  \end{equation}
  This type is manifestly Bishop finite, as it is constructed only from products
  and the unit type.
  
  We then prove an isomorphism between this representation and \(\Pi\)-types.
  \begin{equation}
    \mathbf{Tuple}(T, n) \iff \Pi {(x : \mathbf{Fin}(n))} , T(x)
  \end{equation}
  This allows us to transport our proof of finiteness on tuples to one on
  functions from \(\mathbf{Fin}\) (our second transport operation), proving our
  goal.
\end{proof}
\section{Manifest Enumerability} \label{manifest-enumerability}
Both split enumerability and manifest Bishop finiteness are restricted to sets:
because they both imply decidable equality, no non-set types can satisfy them as
predicates.
To find a more general predicate for finiteness, we \emph{truncate} the
membership proof.
\begin{rm-definition}[Manifest Enumerability]
  \begin{equation}
    \mathcal{E}(A) = \Sigma {(\mathit{xs} : \mathbf{List}(A))} , \Pi {(x : A)} , \lVert x \in \mathit{xs} \rVert
  \end{equation}
\end{rm-definition}
\begin{rm-definition}[Propositional Truncation]a \label{prop-trunc}
  The type \(\lVert A \rVert\) on some type \(A\) is a propositionally truncated
  proof of \(A\) \cite[3.7]{hottbook}.
  In other words, it is a proof that some \(A\) exists, but it does not tell you
  \emph{which} \(A\).

  It is defined as a Higher Inductive Type:
  \begin{equation}
    \begin{aligned}
      \lVert A \rVert =& \; \lvert \cdot \rvert : A \rightarrow \lVert A \rVert ; \\
                      |& \; \text{squash} : \Pi {(x, y : \lVert A \rVert)} , x \equiv y  ; 
    \end{aligned}
  \end{equation}

  We will use three eliminators from \(\lVert A \rVert\) in this paper.
  \begin{enumerate}
  \item \label{elim-prop-prop} For any function \(A \rightarrow B\), where
    \(\text{isProp}(B)\), we have a function \(\lVert A \rVert \rightarrow B\).
  \item \label{elim-prop-monad} A special case of \ref{elim-prop-prop} implies
    that \(\lVert \cdot \rVert\) forms a monad: this means that we get the usual
    Monadic operators on \(\lVert \cdot \rVert\) (bind, pure, fmap, etc.).
    \item \label{elim-prop-coh} We can eliminate from \(\lVert A \rVert\) with a
      function \(f : A \rightarrow B\) iff \(f\) ``doesn't care'' about the
      choice of \(A\) (\(\Pi {(x , y : A)} , f(x) \equiv f(y) \)).
      Formally speaking, \(f\) needs to be ``coherently constant''
      \cite{krausGeneralUniversalProperty2015}, and \(B\) needs to be an
      \(n\)-type for some finite \(n\).
  \end{enumerate}
\end{rm-definition}

We will use the circle as an example non-set type which is nonetheless
manifestly enumerable.
\begin{rm-theorem}
  The circle \(S^1\) is manifestly enumerable.
\end{rm-theorem}
\begin{proof}
  As the cover proof is a truncated proposition, we need only consider the point
  constructors, making this poof the same as the proof of split enumerability on
  \(\top\).
\end{proof}
\subsection{Surjections}
This predicates relation to surjectivity is much the same as split
enumerability's relation to \emph{split} surjectivity.
\begin{rm-lemma}
  A proof of manifest enumerability is equivalent to a surjection from a finite
  prefix of the natural numbers.
\end{rm-lemma}
\subsection{Relation to Split Enumerability}
\begin{rm-theorem} \label{manifest-enum-to-split}
  A manifestly enumerable type with decidable equality is split enumerable.
\end{rm-theorem}
\begin{proof}
  The support list stays the same between both enumerability proofs.

  For the cover proof, we first need the following function which searches a
  list for a particular element (given decidable equality on \(A\)).
  \begin{equation}
    \decin \; : \Pi(x : A) , \Pi(\mathit{xs} : \mathbf{List}(A)) , \mathbf{Dec}(x \in \mathit{xs})
  \end{equation}
  Where \(\mathbf{Dec}(A)\) is a decision on some type \(A\).

  We then need to convert a value of type \(\mathbf{Dec}(x \in \mathit{xs})\) to
  \(x \in \mathit{xs}\).
  We use the following to do that:
  \begin{equation}
    \text{recompute} : \mathbf{Dec}(A) \rightarrow \lVert A \rVert \rightarrow A
  \end{equation}
  The function works via case-analysis on \(\mathbf{Dec}\): in the true case, we
  return the proof; in the false case, we can apply the proof of negation under
  the truncation, and then extract \(\lVert \bot \rVert\) to \(\bot\), as
  \(\bot\) is a mere proposition, thus proving the goal via explosion.
\end{proof}
\subsection{Closure}
Since we don't have an equivalence with \(\mathbf{Fin}\), we don't get closure
under \(\Pi\).
\begin{rm-lemma}
  Manifest enumerability is closed under \(\Sigma\).
\end{rm-lemma}
\begin{proof}
  This closure proof is almost the same as theorem~\ref{split-enum-sigma}.
  The manipulation of the support lists can be carried over as-is; but the type
  of the cover proof has changed, so it will need to be updated.
  As it happens, the translation is straightforward: we effectively write the
  ``monadic'' version of the old function.
\end{proof}
\section{Cardinal Finiteness} \label{cardinal}
For manifest enumerability, we removed the need for decidable equality: in these
next two finiteness predicates, we remove the need for a total order on the
underlying type.
\begin{rm-definition}[Cardinal Finiteness]
  A type \(A\) is cardinally finite, \(\mathcal{C}\), if it has a
  propositionally-truncated proof of bishop finiteness.
  \begin{equation}
    \mathcal{C}(A) = \lVert \mathcal{B}(A) \rVert
  \end{equation}%
\end{rm-definition}%
\subsection{Closure}
The closure properties of cardinal finiteness are effectively the non-dependent
versions of manifest Bishop finiteness.
To see why, consider equation~\ref{split-enum-sigma-clos-sig}.
We can ``lift'' the proof (as a binary function) under a propositional
truncation, giving us equation~\ref{sigma-proof-lifted}, but that doesn't give us
the desired closure proof (equation~\ref{sigma-proof-desired}).
\begin{multicols}{2}
  \noindent
  \begin{equation} \label{sigma-proof-lifted}
    \frac{
      \lVert \mathcal{B}(A) \rVert \; \; \; \lVert \Pi (x : A) , \mathcal{B}(U(x)) \rVert
    }{
      \lVert \mathcal{B}(\Sigma(x : A) , U(x)) \rVert
    }
  \end{equation}
  \begin{equation} \label{sigma-proof-desired}
    \frac{
      \lVert \mathcal{B}(A) \rVert \; \; \; \Pi (x : A) , \lVert \mathcal{B}(U(x))  \rVert
    }{
      \lVert \mathcal{B}(\Sigma(x : A) , U(x)) \rVert
    }
  \end{equation}
\end{multicols}%
To remedy the mismatch we would need a function of the type:
\begin{equation}
  \left( \Pi(x : A) , \lVert \mathcal{B}(U(x)) \rVert \right) \rightarrow
  \lVert \Pi(x : A) , \mathcal{B}(U(x)) \rVert
\end{equation}
Unfortunately, this equation in particular is a form of the axiom of choice
\cite[equation 3.8.3]{hottbook}.
This leaves us with closure under only the non-dependent operations.
\begin{rm-lemma}
  Cardinal finiteness is closed under \(\times\), \(+\), and \(\rightarrow\).
\end{rm-lemma}
\begin{proof}
  All of these closure proofs can be lifted directly from their corresponding
  proofs on manifest Bishop finiteness.
\end{proof}
\subsection{Strength}
The following two theorems are proven in
\cite{yorgeyCombinatorialSpeciesLabelled2014}, in much the same way as we have
done here.
Our contribution for this section is simply the formalisation.
\begin{rm-theorem} \label{cardinal-finite-discrete}
  Any cardinal-finite set has decidable equality.
\end{rm-theorem}
\begin{proof}
  See \cite[Proposition 2.4.10]{yorgeyCombinatorialSpeciesLabelled2014}.
\end{proof}
\begin{rm-theorem}
  Given a cardinally finite type, we can derive the type's cardinality, as well
  as a propositionally truncated proof of equivalence with \(\textbf{Fin}\)s of
  the same cardinality.
  \begin{equation}
    \mathcal{C}(A) \rightarrow \Sigma {(n : \mathbb{N})} , \lVert \textbf{Fin}(n) \simeq A \rVert
  \end{equation}
\end{rm-theorem}
\begin{proof}
  \cite[Proposition 2.4.9]{yorgeyCombinatorialSpeciesLabelled2014}.
\end{proof}
\subsection{Relation to Manifest Bishop Finiteness}
Cardinal finiteness tells us that there is an isomorphism between a type and
\(\mathbf{Fin}\); it just doesn't tell us \emph{which} isomorphism.
To take a simple example, \(\mathbf{Bool}\) has 2 possible isomorphisms with the
set \(\mathbf{Fin}(2)\): one where false maps to 0, and true to 1; and another
where false maps to 1 and true to 0.

To convert from Cardinal finiteness to Bishop finiteness, then, requires that we
supply enough information to identity a particular isomorphism.
A total order is sufficient here: it will give us enough to uniquely order the
support list invariant under permutations.
This tells us what we already knew in the introduction: manifest Bishop
finiteness is cardinal finiteness plus an order.

\begin{rm-theorem}
  Any cardinal finite type with a (decidable) total order is manifestly Bishop
  finite.
\end{rm-theorem}
\begin{proof}
  This proof is quite involved, and will rely on several subsequent lemmas, so
  we will give only its outline here.
  \begin{itemize}
    \item First, we will convert to manifest enumerability: knowing that the
      underlying type is discrete (theorem~\ref{cardinal-finite-discrete}) we can
      go from manifest enumerability to split enumerability
      (lemma~\ref{manifest-enum-to-split}), and subsequently to manifest Bishop
      finiteness (lemma~\ref{split-enum-to-manifest-bishop}).
    \item To convert to manifest enumerability, we need to provide a support
      list: this cannot simply be the support list hidden under the truncation,
      since that would violate the hiding promised by the truncation.
      Instead, we sort the list (using insertion sort).
      We must, therefore, prove that insertion sort is invariant under all
      support lists in cardinal finiteness proofs.
    \item We do this by first showing that all support lists in cardinal
      finiteness proofs are permutations of each other, and then that insertion
      sort is invariant under permutations.
    \item Given our particular definition of permutations, cover proofs
      transfers naturally between lists which are permutations of each other.
  \end{itemize} 
\end{proof}
Now we will build up the toolkit we need to perform the above steps.
First, permutations.
\begin{rm-definition}[List Permutations]
  We say that two lists are permutations of each other if there is an
  isomorphism between membership proofs
  \cite{danielssonBagEquivalenceProofRelevant2012}.
  \begin{equation}
    \mathit{xs} \leftrightsquigarrow \mathit{ys} = \Pi {(x : A)} , x \in \mathit{xs} \iff x \in \mathit{ys}
  \end{equation}
\end{rm-definition}
\begin{rm-lemma}
  Insertion sort is invariant under permutations.
  \begin{equation}
    \mathit{xs} \leftrightsquigarrow \mathit{ys} \implies \text{sort}(\mathit{xs}) \equiv \text{sort}(\mathit{ys})
  \end{equation}
\end{rm-lemma}
\begin{proof}
  First, we prove two properties about insertion sort:
  \begin{enumerate}
    \item It returns a sorted list.
    \item It a returns a list that is a permutation of its input.
  \end{enumerate}
  The second of these points allows us to show that \(\text{sort}(\mathit{xs})\)
  is a permutation of \(\text{sort}(\mathit{ys})\).
  \begin{equation}
    \text{sort}(\mathit{xs}) \leftrightsquigarrow \mathit{xs} \leftrightsquigarrow \mathit{ys} \leftrightsquigarrow \text{sort}(\mathit{ys})
  \end{equation}
  Then, we show that any lists which are both sorted and permutations of each
  other are equal.
  Both of these conditions are true for the output of \(\text{sort}\).
\end{proof}
\section{Kuratowski Finiteness} \label{kuratowski}
Finally we arrive at Kuratowski finiteness
\cite{kuratowskiNotionEnsembleFini1920}.
\begin{rm-definition}[Kuratowski-Finite Set]
  The Kuratowski finite set is a free join semilattice (or, equivalently, a free
  commutative idempotent monoid).
  HITs are required to define this type
  \cite{altenkirchDefinableQuotientsType2011}:
  \begin{equation}
    \begin{aligned}
      \mathcal{K}(A) =&
       \; \cdot \dblcolon \cdot : A \times \mathcal{K}(A) \rightarrow \mathcal{K}(A) ; \\
      \vert& \; [] : \mathcal{K}(A) ; \\
      \vert& \; \text{com} : \Pi (x, y: A) , \Pi (\mathit{xs}: \mathcal{K}(A)) , x \dblcolon y \dblcolon \mathit{xs} \equiv y \dblcolon x \dblcolon \mathit{xs} ; \\
      \vert& \; \text{dup} : \Pi (\mathit{x}: A) , \Pi (\mathit{xs}: \mathcal{K}(A)) , x \dblcolon x \dblcolon \mathit{xs} \equiv x \dblcolon \mathit{xs} ; \\
      \vert& \; \text{trunc} : \Pi (\mathit{xs}, \mathit{ys}: \mathcal{K}(A)) , \Pi (p, q : \mathit{xs} \equiv \mathit{ys}) , p \equiv q ;
    \end{aligned}
  \end{equation}
  The \(\text{com}\) and \(\text{dup}\) constructors effectively add
  commutativity and idempotency to the free monoid (the list), which is made by
  the first two constructors.
  The last constructor makes \(\mathcal{K}(A)\) a set.

  To eliminate from \(\mathcal{K}(A)\), we have to provide equations for each of
  the point constructors which obey the equations of the path constructors.
  For \(\text{com}\) and \(\text{dup}\), this means ensuring that the fold is
  commutative and idempotent, whereas \(\text{trunc}\) means we can only
  eliminate into sets.
\end{rm-definition}

Other representations of \(\mathcal{K}\) \cite{fruminFiniteSetsHomotopy2018}
are more explicit constructions of the free join semilattice (i.e. there is a
point constructors for union instead of cons, and then path constructors for
the associativity and identity laws), but we have found this representation
easier to work with.
Nonetheless, the alternative representation is included in our formalisation,
and proven equivalent to the representation here.
\begin{rm-definition}[Membership of \mathcal{K}]
  First, we need to provide equations for the two point constructors.
  \(x \in [] = \bot\), and \(x \in y \dblcolon \mathit{xs} = \lVert (x \equiv y)
  + (x \in \mathit{xs}) \rVert \).
  The \(\text{com}\) and \(\text{dup}\) constructors are handled by proving that
  the truncated form of \(+\) is itself commutative and idempotent.
  The type of propositions is itself a set, satisfying the \(\text{trunc}\)
  constructor.
\end{rm-definition}
\begin{rm-definition}[Kuratowski Finiteness]
  A type is Kuratowski finite iff there exists a Kuratowski Set which contains
  all of its elements.
  \begin{equation}
    \mathcal{K}^{f}(A) = \Sigma {(\mathit{xs} : \mathcal{K}(A))} , \Pi (x : A) , x \in \mathit{xs}
  \end{equation}
\end{rm-definition}

\begin{rm-theorem}
  A proof of Kuratowski finiteness is equivalent to a propositionally truncated
  proof of enumerability.
  \begin{equation}
    \mathcal{K}^f(A) \simeq \lVert \mathcal{E}(A) \rVert
  \end{equation}
\end{rm-theorem}
\begin{proof}
  We prove by way of an isomorphism.
  In the first direction (from \(\mathcal{K}\) to \(\mathcal{E}\)), because we
  are eliminating into a proposition, we need only deal with the point
  constructors.
  For these, we convert the \(\mathcal{K}\) cons to its list counterpart, and
  similarly for the nil constructor.

  The other direction is proven in \cite{fruminFiniteSetsHomotopy2018}, so we
  will not describe it here.
\end{proof}
\subsection{Topos}
At this point, we see that a ``decidable Kuratowski finite set'' is precisely
equivalent to a cardinal finite set.
From this, we can lift over all of the properties of cardinal finite sets.
In particular, we see that decidable Kuratowski finite sets form a \emph{topos}.
We already have most of the components we need: closure under \(\bot\),
\(\top\), \(\mathbf{Bool}\), \(+\), \(\times\), and \(\rightarrow\).
What remains is the subobject classifier.
\todo[inline]{Topos!}
\section{Infinite Cardinalities} \label{infinite-cardinalities}
In the previous sections we saw different flavours of finiteness which were
really just different flavours of relations to \(\mathbf{Fin}\).
In this section we will see that we can construct a similar classification of
relations to \(\mathbb{N}\), in the form of the countably infinite types.
\subsection{Split Countable Types}
Our first foray into the world of countable types will be a straightforward
analogue to the split enumerable types.
We need change only one element: instead of a support \emph{list}, we instead
have a support \emph{stream}, which is its infinite.
\begin{rm-definition}[Stream]
  \begin{equation}
    \mathbf{Stream}(A) = \llbracket \top \triangleright \text{const} (\mathbb{N}) \rrbracket(A) \simeq \mathbb{N} \rightarrow A
  \end{equation}
\end{rm-definition}
\begin{rm-definition}[Split Countability]
  \begin{equation}
    \mathcal{E}!(A) = \Sigma {(\mathit{xs} : \mathbf{Stream}(A))} , \Pi {(x : A)} , x \in \mathit{xs}
  \end{equation}
\end{rm-definition}
\subsubsection{Closure}
We know that countable infinity is not closed under the exponential (function
arrow), so the only closure we need to prove is \(\Sigma\) to cover all of
what's left.
To do this we have to take a slightly different approach to the functions we
defined before.
Figure~\ref{pairings} illustrates the reason why: previously, we used the
``Cartesian'' product pairing for each support list.
This diverges if the first list is infinite, never exploring anything other than
the first element in the second list.
Instead, we use here the cantor pairing function, which performs a breadth-first
search of the pairings of both lists.
\begin{figure}
  \centering
  \begin{subfigure}[b]{.5\linewidth}
    \begin{tikzcd}[sep=tiny,font=\footnotesize]
      (1,e) \ar[rdddd, out=-45, in=135] & (2,e) \ar[rdddd, out=-45, in=135] & (3,e) \ar[rdddd, out=-45, in=135] & (4,e) \ar[rdddd, out=-45, in=135] & (5,e)        \\
      (1,d) \ar[u]     & (2,d) \ar[u]    & (3,d) \ar[u]    & (4,d) \ar[u]    & (5,d) \ar[u] \\
      (1,c) \ar[u]     & (2,c) \ar[u]    & (3,c) \ar[u]    & (4,c) \ar[u]    & (5,c) \ar[u] \\
      (1,b) \ar[u]     & (2,b) \ar[u]    & (3,b) \ar[u]    & (4,b) \ar[u]    & (5,b) \ar[u] \\
      (1,a) \ar[u]     & (2,a) \ar[u]    & (3,a) \ar[u]    & (4,a) \ar[u]    & (5,a) \ar[u]
    \end{tikzcd}
    \caption{Cartesian}
    \label{cartesian}
  \end{subfigure}%
  \begin{subfigure}[b]{.5\linewidth}
    \begin{tikzcd}[sep=tiny,font=\footnotesize]
      (1,e) \ar[dr] & (2,e) \ar[dr]  & (3,e) \ar[dr]    & (4,e) \ar[dr] & (5,e) \\
      (1,d) \ar[dr] & (2,d) \ar[dr]  & (3,d) \ar[dr]    & (4,d) \ar[dr] & (5,d) \ar[u] \\
      (1,c) \ar[dr] & (2,c) \ar[dr]  & (3,c) \ar[dr]    & (4,c) \ar[dr] & (5,c) \ar[uul] \\
      (1,b) \ar[dr] & (2,b) \ar[dr]  & (3,b) \ar[dr]    & (4,b) \ar[dr] & (5,b) \ar[uuull, out=130, in=-50] \\
      (1,a) \ar[u]  & (2,a) \ar[uul, out=130, in=-50] & (3,a) \ar[uuull, out=130, in=-50] & (4,a) \ar[uuuulll, out=130, in=-50] & (5,a) \ar[uuuulll, out=130, in=-50]
    \end{tikzcd}
    \caption{Cantor}
    \label{cantor}
  \end{subfigure}
  \caption{Two possible products for the sets \(\left[ 1 \dots 5 \right]\) and
    \(\left[  a \dots e \right]\)}
  \label{pairings}
\end{figure}
\begin{rm-theorem} \label{split-countability-sigma}
  Split countability is closed under \(\Sigma\).
\end{rm-theorem}
\begin{proof}
  The main task here is to find a productive (in the coinductive sense) way to
  combine the two support streams.
  A trick which greatly simplifies the algorithm is to first produce a stream of
  lists, which we then concatenate to a single stream.
  This intermediate stream, which we will call \(C\), is generated from the two
  support streams \(\mathit{xs}\) and \(\mathit{ys}\) like so:
  \begin{equation}
    C_i = \left\{ \left( \mathit{xs}_j , \mathit{ys}_{i - j} \right) \mid j \in [0,i] \right\}
  \end{equation}
\end{proof}
\begin{rm-lemma}
  Split countability is closed under non-dependent product and sum.
\end{rm-lemma}
\begin{proof}
  Follows from theorem~\ref{split-countability-sigma}.
\end{proof}
\subsubsection{Kleene Star}
While we lose some closures with the inclusion of infinite types, we gain some
others.
In particular, we have the Kleene star.
This means, in effect, that we have closure under lists.
\begin{rm-theorem}
  Split countability is closed under Kleene star.
  \begin{equation}
    \mathcal{E}!(A) \rightarrow \mathcal{E}!(\mathbf{List}(A))
  \end{equation}
\end{rm-theorem}
\begin{proof}
  As with the proof of closure under \(\Sigma\), our main task here is to figure
  out a way to arrange the indices such that 
\end{proof}
\subsection{Manifest Countability}
As we can quotient out the position information with finite types, so can we with
countable types.
\todo[inline]{fill in rest here}
\section{Practical Uses} \label{practical}
\subsection{Omniscience}
In this section we are interested in restricted forms of the limited principle
of omniscience \cite{myhillErrettBishopFoundations1972}.
\begin{rm-definition}[Limited Principle of Omniscience]
  For any type \(A\) and predicate \(P\) on \(A\), the limited principle of
  omniscience is as follows:
  \begin{equation}
    \left( \Pi {(x : A)} , \mathbf{Dec}(P(x)) \right) \rightarrow \mathbf{Dec} \left( \Sigma {(x : A)} , P(x) \right)
  \end{equation}
  In other words, for any decidable predicate the existential quantification of
  that predicate is also decidable.
\end{rm-definition}
The limited principle of omniscience is non-constructive, but individual types
can themselves satisfy omniscience.
In particular, \emph{finite} types are omniscient.

There is also a universal form of omniscience, which we call exhaustibility.
\begin{rm-definition}[Exhaustibility]
  We say a type \(A\) is exhaustible if, for any decidable predicate \(P\) on
  \(A\), the universal quantification of the predicate is decidable.
  \begin{equation}
    \left( \Pi {(x : A)} , \mathbf{Dec}(P(x)) \right) \rightarrow \mathbf{Dec} \left( \Pi {(x : A)} , P(x) \right)
  \end{equation}
\end{rm-definition}

All of the finiteness predicates we have seen justify exhaustibility.
We will only prove it once, then, for the weakest:
\begin{rm-theorem}
  Kuratowski-finite types are exhaustible.
\end{rm-theorem}
\begin{proof}
  \todo[inline]{Proof}
\end{proof}

Omniscience is stronger than exhaustibility, as we can derive the latter from
the former:
\begin{rm-lemma} \label{omniscient-is-exhaustible}
  Any omniscient type is exhaustible.
\end{rm-lemma}
\begin{proof}
  For decidable propositions, we know the following:
  \begin{equation} \label{lem-conversion}
    \Pi (x : A) , P(x) \leftrightarrow \neg \Sigma (x : A) , \neg P(x) 
  \end{equation}
  To derive exhaustibility from omniscience, then, we run the predicate in its
  negated form, and then subsequently negate the result.
  The resulting decision over \(\neg \Sigma(x : A) , \neg P(x)\) can be
  converted into \(\Pi (x : A) , P(x)\).
\end{proof}
We cannot derive, however, that any exhaustible type is omniscient, as we do not
have the inverse of equation~\ref{lem-conversion}:
\begin{equation} \label{choice-conversion}
  \Sigma (x : A) , P(x) \leftrightarrow \neg \Pi (x : A) , \neg P(x)
\end{equation}
Such an equation would allow us to pick a representative element from any type,
which is therefore non-constructive.
In a sense, equation~\ref{lem-conversion} requires a form of LEM on the
proposition (i.e. requires it to be decidable), whereas
equation~\ref{choice-conversion} requires a form of choice.
Those finiteness predicates which are ordered do in fact give us this form of
choice, so the conversion is valid.
As such, all of the ordered finiteness predicates imply omniscience.
Again, we will prove it only for the weakest.
\begin{rm-theorem}
  Manifest enumerable types are omniscient.
\end{rm-theorem}
\begin{proof}
  \todo[inline]{Proof}
\end{proof}
Finally, we do have a form of omniscience for prop-valued predicates, as they do
not care about the chosen representative.
\begin{rm-theorem}
  Kuratowski finite types are omniscient about prop-valued predicates.
\end{rm-theorem}
\begin{proof}
  \todo[inline]{Proof}
\end{proof}
\subsection{Synthesising Pattern-Matching Proofs}
In particular, they can automate large proofs by analysing every possible case.
In \cite{firsovDependentlyTypedProgramming2015}, the \(\AgdaDatatype{Pauli}\)
group is used as an example.
\ExecuteMetaData[agda/Data/Pauli.tex]{def}
As \(\AgdaDatatype{Pauli}\) has 4 constructors, \(n\)-ary functions on
\(\AgdaDatatype{Pauli}\) may require up to \(4^n\) cases, making even simple
proofs prohibitively verbose.

The alternative is to derive the things we need from \(\mathcal{E}!\) somehow.
As \(\AgdaDatatype{Pauli}\) is a simple finite type, the instance can be defined
in a similar way to those in lemma~\ref{split-enum-bool-top-bot}.
From here we can already derive decidable equality, a function which requires 16
cases if implemented manually.

For proof search, the procedure is a well-known one in Agda
\cite{devrieseBrightSideType2011}: we ask for the result of a decision procedure
as an \emph{instance argument}, which will demand computation during
typechecking.
\subsection{Multiple Arguments}
The automation machinery above only deals with single-argument predicates.
This is not a problem, as we know that we can work with multiple arguments by
currying and uncurrying, since all of the finiteness predicates are closed under
\(\times\).
To automate away the curry/uncurry noise we will use instance search, building
on \cite{allaisGenericLevelPolymorphic2019} to develop a small interface to
generic \(n\)-ary functions and properties.
Our generic representation can handle dependent \(\Sigma\) and \(\prod\)
types (rather than their non-dependent counterparts, \(\times\) and
\(\rightarrow\)).
This extension was necessary for our use case: it is mentioned in the paper
as the obvious next step. 
We also implement the curry-uncurry combinators as (verified) isomorphisms.

A full explanation of our implementation is beyond the scope of this work, so we
only present the finished interface, which is used like so:
\ExecuteMetaData[agda/Data/Pauli.tex]{assoc-prf}

\todo[inline]{Synthesise functions}
\todo[inline]{Partial search over infinite types}
\section{Related Work}
\cite{spiwackConstructivelyFinite2010} explored the notion of ``constructive
finite sets'', demonstrating that there is a rich variety of finiteness
predicates in a constructive setting.
Some of those predicates where explored more in
\cite{firsovVariationsNoetherianness2016a},
\cite{parmannInvestigatingStreamlessSets2015}, and
\cite{uustaluFinitenessRationalSequences2017a}.

\cite{yorgeyCombinatorialSpeciesLabelled2014} explored some of the definitions
we have given here (manifest bishop finiteness and cardinal finiteness) in HoTT,
in order to construct species.
Some of this paper is a formalisation of the early parts of that work.

\cite{firsovDependentlyTypedProgramming2015} explores finite sets in Agda and
uses them to automate proof machinery in much the same way as we have here.
The primary difference between their proof automation and ours is that we can
talk about finite functions, as we are working in HoTT.

\cite{fruminFiniteSetsHomotopy2018} explores many of the same notions as we do
here: cardinal finiteness, and Kuratowski finite sets.
Our work makes extensive use of the computational properties of CuTT, however,
and we are able to formalise the finiteness of function objects.
\bibliographystyle{splncs04}
\bibliography{bibliography.bib}
\end{document}