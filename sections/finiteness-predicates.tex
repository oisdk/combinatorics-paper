\section{Finiteness Predicates} \label{finiteness-predicates}
In this section, we will define and briefly describe each of the five predicates
in Figure~\ref{finite-classification}.
As we will see, each of these predicates has subtle differences from the others:
we will outline how some predicates are too informative, and how others aren't
powerful enough, for our needs, before settling on decidable Kuratowski
finiteness as our focus.

As we make our way through each predicate, we will be interested in two aspects:
how can we build proofs of this predicate (i.e. is the product of two finite
types finite?) and what do we \emph{get} once we do (i.e. does this predicate
tell us the number of elements in the finite set?).
\subsection{Split Enumerability}
We will start with a simple notion of finiteness, called \emph{split}
enumerability.

\begin{definition}[Split Enumerable Set] \label{split-enum-def}
  To say that some type \(A\) is split enumerable is to say that there is a list
  \(\mathit{support} : \mathbf{List}(A)\) such that any value \(x : A\) is in
  \(\mathit{support}\).
  \begin{equation} \label{split-enum-def-eqn}
    \mathcal{E}!(A) \coloneqq \Sigma {(\mathit{support} : \textbf{List}(A))} , \Pi {(x : A)} , x \in \mathit{support}
  \end{equation}
  We call the first component of this pair the ``support'' list, and the second
  component the ``cover'' proof.
  An equivalent version of this predicate was called \verb+Listable+ in
  \cite{firsovDependentlyTypedProgramming2015}.
\end{definition}
We used some extra types in the above definition, which we will define here:

\begin{definition}[\(\mathbf{List}\)] \label{List}
  In this paper we define lists as \emph{containers}:
  \begin{equation}
    \mathbf{List} \coloneqq \llbracket \mathbb{N} , \mathbf{Fin} \rrbracket
  \end{equation}
  This may seem quite foreign or complex in comparison to the usual definition
  of lists as an inductive data type:
  \begin{inductivetype}{\mathbf{List}(A)}
    \inductivetypeclause{\left[ \right]}{\mathbf{List}(A)}
    \inductivetypeclause{\wc \dblcolon \wc}{A \rightarrow \mathbf{List}(A) \rightarrow \mathbf{List}(A)}
  \end{inductivetype}
  But it turns out to be quite useful for later proofs.

  Moreover, the inductive type and the container-based type are equivalent, as
  is proven in our formalisation.
\end{definition}
\begin{definition}[Containers] \label{container-def}
  A container \cite{abbottContainersConstructingStrictly2005} is a pair
  \(S , P\) where \(S\) is a type, the elements of which are called
  the \emph{shapes} of the container, and \(P\) is a type family on \(S\), where
  the elements of \(P(s)\) are called the \emph{positions} of a container.
  We ``interpret'' a container into a functor defined like so:
  \begin{equation} \label{container-interp}
    \llbracket S , P \rrbracket(A) \coloneqq \Sigma {(s : S)} , \left( P(s) \rightarrow A \right)
  \end{equation}
  Membership of a container can be defined like so:
  \begin{equation} \label{container-membership}
    x \in \mathit{xs} \coloneqq \text{fib}_{\text{snd}(\mathit{xs})}(x)
  \end{equation}
\end{definition}
\begin{definition}[\(\mathbf{Fin}\)] \label{Fin}
  \(\mathbf{Fin}(n)\) is the type of natural numbers smaller than \(n\). We
  define it the standard way:
  \begin{equation}
    \begin{alignedat}{3}
      & \mathbf{Fin}(0)     && \coloneqq \bot ; \\
      & \mathbf{Fin}(n + 1) && \coloneqq \top \uplus \mathbf{Fin}(n) ;
    \end{alignedat}
  \end{equation} 
\end{definition}
\subsubsection{Instances}
Now that we have a suitable definition of finiteness, we will next prove that
some things are finite.
\begin{lemma}
  \(\bot\), \(\top\), and \(\mathbf{Bool}\) are split enumerable.
\end{lemma}
\begin{proof}
  These three types are represented by fairly simple finite sets:
  \begin{align}
    \bot &= \left\{  \right\} \\
    \top &= \left\{ \text{tt}  \right\} \\
    \mathbf{Bool} &= \left\{  \text{false}, \text{true} \right\}
  \end{align}
  Which leads to quite simple proofs of finiteness.

  The support list for is \(\left[  \right]\), \(\left[ \text{tt} \right]\), and
  \(\left[ \text{false}, \text{true} \right]\), respectively, and the cover
  proof is a function that returns the index of the supplied element.
\end{proof}
\begin{agdalisting}
  The code in our formalisation of these three proofs is quite simple.
  Just as an example, the proof that \(\mathbf{Bool}\) is split enumerable is as
  follows:
  \ExecuteMetaData[agda/Cardinality/Finite/SplitEnumerable.tex]{bool-inst}
\end{agdalisting}

With the most basic simple types out of the way, the obvious next choice is the
(non-dependent) sums and products: \(\uplus\) and \(\times\).
Both of these types can be constructed from the \emph{dependent} sum, however,
so that is the type we will prove finite.
From that we can derive a much wider array of finiteness proofs.
\begin{lemma} \label{split-enum-sigma}
  Split enumerability is closed under \(\Sigma\).
  \begin{equation}
    \frac{
      \mathcal{E}!(A) \; \; \; \Pi(x : A) , \mathcal{E}!(U(x))
    }{
      \mathcal{E}!(\Sigma(x : A) , U(x))
    }
  \end{equation}
\end{lemma}
\begin{proof}
  Let \(A\) be a type which is split enumerable, and \(U\) be a type family over
  \(A\) which is split enumerable at every point.
  Formally, we have the following proofs:
  \begin{align}
    \mathcal{E}!_A &: \mathcal{E}!(A) \\
    \mathcal{E}!_U &: \Pi(x : A) , \mathcal{E}!(U(x))
  \end{align}

  Our task is to construct a proof of type:
  \begin{equation}
    \mathcal{E}!(\Sigma(x : A) , U(x))
  \end{equation}
  This proof itself is composed of two components:
  \begin{align}
    \mathit{support} &: \mathbf{List}(\Sigma(x : A) , U(x)) \\
    \mathit{cover} &: \Pi(x : \Sigma(y : A) , U(y)) , x \in \mathit{support}
  \end{align}

  To construct the support list, we apply the function \(\mathcal{E}!_U\) to
  every element in the support list of \(\mathcal{E}!_A\), extract the support
  lists from the resulting finiteness proofs, and concatenate them.

  \begin{agdalisting}
    This pattern of applying a function to each element in a list and
    concatenating the result is of course well-known in functional programming,
    and is in fact the pattern that makes lists a monad.
    While this insight isn't strictly relevant to our work here, it does mean
    the implementation of this function can use Agda's do notation, resulting
    in the following extremely clean implementation:
    \ExecuteMetaData[agda/Cardinality/Finite/SplitEnumerable.tex]{sup-sigma}
  \end{agdalisting}

  To prove that this support list does in fact cover the entirety of the type
  \(\Sigma \; A \; U\), we note that any element of type \(\Sigma \; A \; U\)
  must have a first component in the support list of \(\mathcal{E}!_A\), and its
  second component must be in the result of applying \(\mathcal{E}!_U\) to that
  first element (since that support list contains every element of type
  \(U(x)\)).
  Therefore, the pair itself must be in our constructed support list.
\end{proof}
\subsubsection{Derivations}
We have a way to construct finiteness proofs, and a semiring-like toolbox to
combine them.
What we're now interested in is what we can \emph{derive} from them.

First, we will look at how this predicate relates to more traditional, classical
notions of finiteness.
in a classical setting we likely wouldn't mention ``lists'' or the like, and
would instead define finiteness based on the existence of some injection or
surjection.
As it turns out, our definition of finiteness here is precisely the same as the
surjection-based one, in quite a deep way!

First, we will need to define our terms: in HoTT, surjections are a little more
complex than what you'd find in either MLTT or classical mathematics.
\begin{definition}[Surjections] \label{surjections}
  We define both surjections and \emph{split} surjections here \cite[definition
  4.6.1]{hottbook}.
  \begin{alignat}{3}
    &\text{surj}(f)             &&\coloneqq \Pi(y : B) , \lVert \text{fib}_f(y) \rVert \\
    &A \twoheadrightarrow B     &&\coloneqq \Sigma (f : A \rightarrow B) , \text{surj}(f) \label{surj-arrow-eqn} \\
    &\text{sp-surj}(f)          &&\coloneqq \Pi(y : B) , \text{fib}_f(y) \label{sp-surj-eqn} \\
    &A \twoheadrightarrow! \; B &&\coloneqq \Sigma (f : A \rightarrow B) , \text{sp-surj}(f) \label{sp-surj-arrow-eqn}
  \end{alignat}
\end{definition}
Over sets, the surjections and split surjections are the same thing, but there
is a difference one we involve non-set types like the circle.

Next, we will show that split enumerability is equivalent to the existence of a
split surjection from a finite prefix of the natural numbers:
\begin{lemma} \label{split-enum-is-split-surj}
  \begin{equation}
    \mathcal{E}!(A) \simeq \Sigma (n : \mathbb{N}) , \left( \mathbf{Fin}(n) \twoheadrightarrow ! \; A \right)
  \end{equation}
\end{lemma}
\begin{proof}
  \begin{align*}
     \mathcal{E}!(A) &
    \simeq \Sigma (\mathit{xs} : \textbf{List}(A)) , \Pi {(x : A)} , x \in \mathit{xs}
    && \text{def.~\ref{split-enum-def} }(\mathcal{E}!)
    \\
    & \simeq \Sigma (\mathit{xs} : \textbf{List}(A)) , \Pi {(x : A)} , \text{fib}_{\text{snd}(\mathit{xs})}(x)
    && \text{eqn.~\ref{container-membership} } (\in)
    \\
    & \simeq \Sigma (\mathit{xs} : \textbf{List}(A)) , \text{sp-surj}(\text{snd}(\mathit{xs}))
    && \text{eqn.~\ref{sp-surj-eqn} (sp-surj)}
    \\
    & \simeq \Sigma (\mathit{xs} : \llbracket \mathbb{N} , \mathbf{Fin} \rrbracket (A)) , \text{sp-surj}(\text{snd}(\mathit{xs}))
    && \text{def.~\ref{List} } (\mathbf{List})
    \\
    & \simeq \Sigma (\mathit{xs} : \Sigma (n : \mathbb{N}) , \Pi (i : \mathbf{Fin}(n)) , A) , \text{sp-surj}(\text{snd}(\mathit{xs}))
    && \text{eqn.~\ref{container-interp} } (\llbracket \wc \rrbracket)
    \\
    & \simeq \Sigma (n : \mathbb{N}) , \Sigma (f : \mathbf{Fin}(n) \rightarrow A) , \text{sp-surj}(f)
    && \text{Reassociation of } \Sigma
    \\
    & \simeq \Sigma (n : \mathbb{N}) , ( \mathbf{Fin}(n) \twoheadrightarrow ! \; A )
    && \text{eqn.~\ref{sp-surj-arrow-eqn} } (\twoheadrightarrow!)
  \end{align*}
\end{proof}
\begin{agdalisting}
In our formalisation, the proof is a single line:
\ExecuteMetaData[agda/Cardinality/Finite/SplitEnumerable.tex]{is-split-inj}
The only step which isn't definitional equality is the reassociation of
\(\Sigma\).
\ExecuteMetaData[agda/Data/Sigma/Properties.tex]{reassoc}
\end{agdalisting}
(The simplicity of this proof, by the way, is why we preferred the
container-based definition of lists over the traditional one.)

Split enumerability implies decidable equality on the underlying type.
To prove this, we will make use of the following lemma, proven in the
formalisation:
\begin{definition}[Injections]
  Injective functions are more straightforward to define constructively than
  surjective ones:
  \begin{alignat}{3}
    & \text{injective}(f) &&\coloneqq \Pi (x, y : A) , f \; x \equiv f \; y \rightarrow x \equiv y \\
    & A \rightarrowtail B &&\coloneqq \Sigma(f : A \rightarrow B) , \text{injective}(f)
  \end{alignat}
\end{definition}
\begin{lemma} \label{split-surj-to-inj}
  A split-surjection from \(A\) to \(B\) implies an injection from \(B\) to
  \(A\).
  \begin{equation}
    (A \twoheadrightarrow! \; B) \rightarrow (B \rightarrowtail A)
  \end{equation}
\end{lemma}
\begin{lemma} \label{inj-discrete}
  For any type \(A\) which injects into a discrete type \(B\), \(A\) is
  discrete.
  \begin{equation}
    \frac{
      A \rightarrowtail B \; \; \; \text{Discrete}(B)
    }{
      \text{Discrete}(A)
    }
  \end{equation}
\end{lemma}

\begin{lemma} \label{discrete-surj}
  \begin{equation}
    \frac{
      A \twoheadrightarrow! \; B \; \; \; \text{Discrete}(A)
    }{
      \text{Discrete}(B) 
    }
  \end{equation}
\end{lemma}
\begin{proof}
  This proof is can be straightforwardly derived from lemmas
  \ref{split-surj-to-inj} and \ref{inj-discrete}.
\end{proof}

\begin{lemma} \label{split-enum-discrete}
  Every split enumerable type is discrete.
\end{lemma}
\begin{proof}
  Let \(A\) be a split enumerable type.
  By lemma~\ref{split-enum-is-split-surj}, there is a surjection from
  \(\mathbf{Fin}(n)\) for some \(n\).
  Also, we know that \(\mathbf{Fin}(n)\) is discrete (proven in our
  formalisation).
  Therefore, by lemma~\ref{discrete-surj}, \(A\) is discrete.
\end{proof}
\subsection{Manifest Bishop Finiteness}
We mentioned in the introduction that occasionally in constructive mathematics
proofs will contain ``too much'' information.
With split enumerability we can see an instance of this.
Consider our proof that the booleans are finite: 


\begin{definition}[Manifest Bishop Finiteness]
  \begin{equation} \label{bish-def}
    \mathcal{B}(A) \coloneqq \Sigma {(\mathit{xs} : \textbf{List}(A))} , \Pi {(x : A)} , x \inunique xs
  \end{equation}
\end{definition}
The only difference between manifest Bishop finiteness and split enumerability
is the membership term: here we require unique membership (\(\inunique\)),
rather than simple membership (\(\in\)).
\begin{definition}[Unique Membership] \label{uniq-memb-def}
  \begin{equation}
    x \inunique \mathit{xs} \coloneqq \text{isContr}(x \in \mathit{xs})
  \end{equation}
\end{definition}

We use the word ``manifest'' here to distinguish from another common
interpretation of Bishop finiteness, which we have called cardinal finiteness in
this paper.
The ``manifest'' refers to the fact that we have a concrete, non-truncated list
of the elements in the proof.

Where split enumerability was the enumeration form of a split surjection from
\(\mathbf{Fin}\), manifest Bishop finiteness is the enumeration form of an
\emph{equivalence} with \(\mathbf{Fin}\).
\begin{lemma} \label{bishop-equiv}
  \begin{equation}
    \mathcal{B}(A) \simeq \Sigma {(n : \mathbb{N})} , \left( \mathbf{Fin}(n) \simeq A \right)
  \end{equation}
\end{lemma}
This proof is effectively the same as that of
lemma~\ref{split-enum-is-split-surj}.
\subsection{Cardinal Finiteness}
Each finiteness predicate so far has contained an \emph{ordering} of the
underlying type.
For our purposes, this is too much information: it means that when constructing
the ``category of finite sets'' later on, instead of each type having one
canonical representative, it will have \(n!\), where \(n\) is the cardinality of
the type\footnotemark.

\footnotetext{
  We actually do get a category (a groupoid, even) from manifest Bishop
  finiteness \cite{yorgeyCombinatorialSpeciesLabelled2014}: it's the groupoid of
  finite sets equipped with a linear order, whose morphisms are order-preserving
  bijections.
  We do not explore this particular construction in any detail.
}

To remedy the problem, we will use propositional truncation
(def.~\ref{prop-trunc}).
\begin{definition}[Cardinal Finiteness]
  \begin{equation}
    \mathcal{C}(A) \coloneqq \lVert \mathcal{B}(A) \rVert \simeq \lVert \Sigma(n : \mathbb{N}) , (\mathbf{Fin}(n) \simeq A) \rVert
  \end{equation}
\end{definition}
At first glance, it might seem that we lose any useful properties we could
derive from \(\mathcal{B}\).
Luckily, this is not the case: by eliminator \ref{elim-prop-coh} of
def.~\ref{prop-trunc}, we need only show that the output is uniquely determined.

The following two lemmas are proven in
\cite{yorgeyCombinatorialSpeciesLabelled2014} (Proposition 2.4.9 and 2.4.10,
respectively), in much the same way as we have done here.
Our contribution for this section is simply the formalisation.
\begin{lemma}
  Given a cardinally finite type, we can derive the type's cardinality, as well
  as a propositionally truncated proof of equivalence with \(\textbf{Fin}\)s of
  the same cardinality.
  \begin{equation}
    \mathcal{C}(A) \rightarrow \Sigma {(n : \mathbb{N})} , \lVert \textbf{Fin}(n) \simeq A \rVert
  \end{equation}
\end{lemma}
\begin{lemma} \label{cardinal-finite-discrete}
  Any cardinal-finite set has decidable equality.
\end{lemma}
\subsection{Manifest Enumerability}
\begin{definition}[Manifest Enumerability]
  \begin{equation}
    \mathcal{E}(A) \coloneqq \Sigma {(\mathit{xs} : \mathbf{List}(A))} , \Pi {(x : A)} , \lVert x \in \mathit{xs} \rVert
  \end{equation}
\end{definition}
As with manifest Bishop finiteness, the only difference with this type and split
enumerability is the membership proof: here we have propositionally truncated
it.
This has two effects.
First, it means that this proof represents a true surjection (rather than a
split surjection) from \(\mathbf{Fin}\).
\begin{lemma}
  \begin{equation}
    \mathcal{E}(A) \simeq \Sigma(n : \mathbb{N}) , (\mathbf{Fin}(n) \twoheadrightarrow A)
  \end{equation}
\end{lemma}
The proof for this lemma is similar in structure to
lemma~\ref{split-enum-is-split-surj} and lemma~\ref{bishop-equiv}.

Secondly, it means the predicate does not imply decidable equality.
More significantly, it allows the predicate to be defined over non-set types,
like the circle.
\begin{definition}[\(S^1\)] \label{circle-def}
  The circle, \(S^1\), can be represented in HoTT as a higher inductive type.
  \begin{equation}
    \begin{alignedat}{3}
      S^1 \coloneqq & \; \text{base} &&: S^1 ; \\
      | & \; \text{loop} &&: \text{base} \equiv \text{base} ; 
    \end{alignedat}
  \end{equation}
  We will use it here as an example of a non-set type, i.e. a type for which not
  all paths are equal.
  This also means that it does not have decidable equality.
\end{definition}
\begin{lemma}
  The circle \(S^1\) is manifestly enumerable.
\end{lemma}
\subsection{Kuratowski Finiteness}
The first thing we must define is a representation of subsets.
\begin{definition}[Kuratowski Finite Subset]
  \(\mathcal{K}(A)\) is the type of Kuratowski-finite subsets of \(A\).
  \begin{inductivetype}{\mathcal{K}(A)}
    \inductivetypeclause{\left[ \right]}{\mathcal{K}(A)}
    \inductivetypeclause{\wc \dblcolon \wc}{A \rightarrow \mathcal{K}(A) \rightarrow \mathcal{K}(A)}
    \inductivetypeclause{\text{com}}{\Pi (x, y: A) , \Pi (\mathit{xs} : \mathcal{K}(A)) , x \dblcolon y \dblcolon \mathit{xs} \equiv y \dblcolon x \dblcolon \mathit{xs}}
    \inductivetypeclause{\text{dup}}{\Pi (x : A) , \Pi (\mathit{xs} : \mathcal{K}(A)) , x \dblcolon x \dblcolon \mathit{xs} \equiv x \dblcolon \mathit{xs}}
    \inductivetypeclause{\text{trunc}}{\Pi (\mathit{xs}, \mathit{ys}: \mathcal{K}(A)) , \Pi (p, q : \mathit{xs} \equiv \mathit{ys}) , p \equiv q}
  \end{inductivetype}
  We define it as a HIT (definition~\ref{HITs}).
  The first two constructors are point constructors, giving ways to create
  values of type \(\mathcal{K}(A)\).
  They are also recognisable as the two constructors for finite lists, a type
  which represents the free monoid.

  The next two constructors add extra paths to the type: equations that usage of
  the type must obey.
  These extra paths turn the free monoid into the free \emph{commutative} (com)
  \emph{idempotent} (dup) monoid.

  The final constructor enforces that the type \(\mathcal{K}(A)\) must be a set.
\end{definition}
The Kuratowski finite subset is a free join semilattice (or, equivalently, a
free commutative idempotent monoid).
More prosaically, \(\mathcal{K}\) is the abstract data type for finite sets, as
defined in the Boom hierarchy \cite{boomFurtherThoughtsAbstracto1981,
  bunkenburgBoomHierarchy1994}.
However, rather than just being a specification, \(\mathcal{K}\) is fully usable
as a data type in its own right, thanks to HITs.

Other definitions of \(\mathcal{K}\) exist (such as the one in
\cite{fruminFiniteSetsHomotopy2018}) which make the fact that \(\mathcal{K}\) is
the free join semilattice more obvious.
We have included such a definition in our formalisation, and proven it
equivalent to the one above.

Next, we need a way to say that an entire type is Kuratowski finite.
For that, we will need to define membership of \(\mathcal{K}\).
\begin{definition}[Membership of \(\mathcal{K}\)]
  Membership is defined by pattern-matching on \(\mathcal{K}\).
  The two point constructors are handled like so:
  \begin{equation}
    \begin{alignedat}{2}
      x \in&& \; []                      &\coloneqq \bot ; \\
      x \in&& \; y \dblcolon \mathit{ys} &\coloneqq \lVert x \equiv y \uplus x \in \mathit{ys} \rVert ;
    \end{alignedat}
  \end{equation}
  The \(\text{com}\) and \(\text{dup}\) constructors are handled by proving that
  the truncated form of \(\uplus\) is itself commutative and idempotent.
  The type of propositions is itself a set, satisfying the \(\text{trunc}\)
  constructor.
\end{definition}
Finally, we have enough background to define Kuratowski finiteness.
\begin{definition}[Kuratowski Finiteness]
  \begin{equation}
    \mathcal{K}^{f}(A) = \Sigma {(\mathit{xs} : \mathcal{K}(A))} , \Pi (x : A) , x \in \mathit{xs}
  \end{equation}
\end{definition}

We also have the following two lemmas, proven in both
\cite{fruminFiniteSetsHomotopy2018} and our formalisation.
\begin{lemma}
  \(\mathcal{K}^f\) is a mere proposition.
\end{lemma}
\begin{lemma}
  This circle \(S^1\) is Kuratowski finite.
\end{lemma}
\section{Relations Between Each Finiteness Definition} \label{relations}
We will now look at the arrows in figure~\ref{finite-classification}.
\subsection{Split Enumerability and Manifest Bishop Finiteness}
While manifest Bishop finiteness might seem stronger than split enumerability,
it turns out this is not the case.
Both predicates imply the other.
\begin{lemma} \label{manifest-bishop-to-split-enum}
  Any manifest Bishop finite type is split enumerable.
\end{lemma}
\begin{proof}
  To construct a proof of split enumerability from one of manifest Bishop
  finiteness, it suffices to convert a proof of \(x \inunique \mathit{xs}\) to
  one of \(x \in \mathit{xs}\), for all \(x\) and \(\mathit{xs}\).
  Since \(\inunique\) is defined as a contraction of \(\in\), such a conversion
  is simply the \(\text{fst}\) function.
\end{proof}

\begin{lemma} \label{split-enum-to-manifest-bishop}
  Any split enumerable set is manifest Bishop finite.
\end{lemma}
This proof takes significantly more work.
The ``unique membership'' condition in \(\mathcal{B}\) means that we are not
permitted duplicates in the support list.
The first step in the proof, then, is to filter those duplicates out from the
support list of the \(\mathcal{E}!\) proof: we can do this using the decidable
equality provided by \(\mathcal{E}!\) (lemma~\ref{split-enum-discrete}).
From there, all we need to show is that the membership proof carries over
appropriately.
\subsection{Split Enumerability and Manifest Enumerability}
\begin{lemma} \label{split-enum-to-manifest-enum}
  Any split enumerable type is manifestly enumerable.
\end{lemma}
This lemma is proven by truncating the membership proof in split enumerability.

\begin{lemma} \label{manifest-enum-to-split-enum}
  A manifestly enumerable type with decidable equality is split enumerable.
\end{lemma}

\subsection{Manifest Bishop Finiteness and Cardinal Finiteness}
\begin{lemma} \label{manifest-bishop-to-cardinal}
  Any manifest Bishop finite type is cardinal finite.
\end{lemma}
\begin{theorem} \label{cardinal-to-manifest-bishop}
  Any cardinal finite type with a total order is Bishop finite.
\end{theorem}
The proof for this particular theorem is quite involved in the formalisation, so
we only give its sketch here.
First, note that we actually convert to manifest enumerability first: this can
be converted to split enumerability with decidable equality, which is provided
by cardinal finiteness.

Next, we define permutations.
\begin{definition}[List Permutations]
  Two lists are permutations of each other if their membership proofs are all
  equivalent\footnotemark \cite{danielssonBagEquivalenceProofRelevant2012}.
  \begin{equation}
    \mathit{xs} \leftrightsquigarrow \mathit{ys} = \Pi {(x : A)} , x \in \mathit{xs} \simeq x \in \mathit{ys}
  \end{equation}
\end{definition}

\footnotetext{
  The definition in \cite{danielssonBagEquivalenceProofRelevant2012} and our
  formalisation is slightly different: we say permutations are lists with
  \emph{isomorphic} membership proofs.
  The distinction, as it happens, does not affect our work here.
}

Next, we define a sort function which relies on the provided total order.
We further prove the following fact about this sort function:
\begin{equation}
  \Pi(\mathit{xs}, \mathit{ys} : \mathbf{List}(A)) , \mathit{xs} \leftrightsquigarrow \mathit{ys} \rightarrow \text{sort}(\mathit{xs}) \equiv \text{sort}(\mathit{ys})
\end{equation}

Next, notice that the support lists of any two proofs of manifest Bishop
finiteness must be permutations of each other.
This will allow us to sort the support list of a proof of cardinal finiteness in
a coherently constant (definition~\ref{prop-trunc},
eliminator~\ref{elim-prop-coh}) way, pulling the support list out from the
truncation.
The cover proof emerges naturally from the definition of the permutation.
\subsection{Cardinal Finiteness and Kuratowski Finiteness}
\begin{lemma} \label{cardinal-kuratowski}
  \begin{equation}
    \mathcal{C}(A) \simeq \mathcal{K}^f(A) \times \text{Discrete}(A)
  \end{equation}
\end{lemma}
This proof is constructed by providing a pair of functions: one from
\(\mathcal{C}(A)\) to \(\mathcal{K}^f(A) \times \text{Discrete}(A)\), and one the
other way.
This pair implies an equivalence, because both source and target are
propositions.
The actual functions themselves are proven in our formalisation, as well as in
\cite{fruminFiniteSetsHomotopy2018}.


%%% Local Variables:
%%% mode: latex
%%% TeX-master: "../paper"
%%% End: