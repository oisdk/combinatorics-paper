\chapter{Topos} \label{topos}
In this section we will examine the categorical interpretation of finite sets.
In particular, we will prove that decidable Kuratowski finite types form a
\(\Pi\)-pretopos.
A lot of the work for this proof has been done already: in
Theorem~\ref{cardinal-kuratowski} we saw that Kuratowski finite types were
equivalent to Cardinally finite types.
We will use the latter definition implementation-wise from now on, as it is
slightly easier to work with: CuTT's transport means we can do this without loss
of generality.

\section{Categories in HoTT}
\section{Closure and the Category of Sets}
\section{The Absence of the Subobject Classifier}
\begin{agdalisting} \label{filter-subobject}
  \ExecuteMetaData[agda/Cardinality/Finite/SplitEnumerable.tex]{subobject}
\end{agdalisting}


\section{Closure}
For the first three closure proofs, we only consider split enumerability:
as it is the strongest of the finiteness predicates, we can derive the other
closure proofs from it.

\section{The Category of Finite Sets}
HoTT and CuTT seem to be especially suitable settings for formalisations of
category theory.
The univalence axiom in particular allows us to treat categorical isomorphisms
as equalities, saving us from the dreaded ``setoid hell''.

We follow \cite[chapter 9]{hottbook} in its treatment of
categories in HoTT, and in its proof that sets do indeed form a category.
We will first briefly go through the construction of the category
\(\mathit{Set}\), as it differs slightly from the usual method in type theory.

First, the type of objects and arrows:
\begin{alignat}{3}
  &\text{Obj}_\mathit{Set}      &&\coloneqq \Sigma(x : \mathbf{Type}) , \text{isSet}(x) \\
  &\text{Hom}_\mathit{Set}(x , y) &&\coloneqq  \text{fst}(x) \rightarrow \text{fst}(y)
\end{alignat}
As the type of objects makes clear, we have already departed slightly from the
simpler \(\text{Obj}_\mathit{Set} \coloneqq \mathbf{Type}\) way of doing things:
of course we have to, as HoTT allows non-set types.
Furthermore, after proving the usual associativity and identity laws for
composition (which are definitionally true in this case), we must further show
\(\text{isSet}(\text{Hom}_\mathit{Set}(x,y))\); even then we only have a
precategory.

To show that \(\mathit{Set}\) is a category, we must show that categorical
isomorphisms are equivalent to equivalences.
In a sense, we must give a univalence rule for the category we are working in.

We have provided formal proofs that \(\mathit{Set}\) does indeed form a
category, and the following:
\begin{theorem}[The Category of Finite Sets]
  Finite sets form a category in HoTT when defined like so:
  \begin{equation}
    \begin{alignedat}{3}
      &\text{Obj}_\mathit{FinSet}      &&\coloneqq \Sigma(x : \mathbf{Type}) , \mathcal{C}(x) \\
      &\text{Hom}_\mathit{FinSet}(x , y) &&\coloneqq  \text{fst}(x) \rightarrow \text{fst}(y)
    \end{alignedat}
  \end{equation}
\end{theorem}
\section{The \(\Pi\)-pretopos of Finite Sets}
For this proof, we follow again the proof that \(\mathit{Set}\) forms a \(\Pi
W\)-pretopos from \cite[chapter 10]{hottbook} and
\cite{rijkeSetsHomotopyType2015}.
The difference here is that clearly we do not have access to \(W\)-types, as
they would permit infinitary structures.

We first must show that \(\mathit{Set}\) has an initial object and finite,
disjoint sums, which are stable under pullback.
We also must show that \(\mathit{Set}\) is a regular category with effective
quotients.
We now have a pretopos: the presence of \(\Pi\) types make it a
\(\Pi\)-pretopos.

We have proven the above statements for both \(\mathit{Set}\) and
\(\mathit{FinSet}\).
As far as we know, this is the first formalisation of either.
\begin{theorem} \label{finite-topos}
  The category of finite sets, \(\mathit{FinSet}\), forms a \(\Pi\)-pretopos.
\end{theorem}


%%% Local Variables:
%%% mode: latex
%%% TeX-master: "../paper"
%%% End: