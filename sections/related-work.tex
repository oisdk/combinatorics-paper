\section{Related Work}
The univalent foundations program is the main basis for this work
\cite{hottbook}.
In particular, our formalisation in section~\ref{topos} relied heavily on
\cite[chapter 10]{hottbook}, and \cite{rijkeSetsHomotopyType2015}, a paper which
contains much of the same material.

Finite sets in a constructive setting has been studied extensively before:
In \cite{coquandConstructivelyFinite2010} four separate predicates for
finiteness were considered (split-enumerable being the only one explored in this
work), and \cite{firsovVariationsNoetherianness2016} explores Noetherianness.
\cite{firsovDependentlyTypedProgramming2015} explored what we have called split
enumerability and manifest Bishop finiteness (although they are stated slightly
differently), and they use these to build a library for proof search.
In \cite{fruminFiniteSetsHomotopy2018} the topic of Kuratowski finite sets in
HoTT is studied extensively: we have focused more on the non-truncated versions
of finiteness (the ``manifest'' predicates), and we have provided the missing
\(\Pi\)-pretopos proof of decidable Kuratowski finite sets.

\cite{iversenUnivalentCategoriesFormalization2018} provided a starting point for
our categorical formalisation: it contains a proof, for instance, that homotopy
sets form a category.


%%% Local Variables:
%%% mode: latex
%%% TeX-master: "../paper"
%%% End: