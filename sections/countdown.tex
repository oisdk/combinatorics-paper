\chapter{Countdown}
Countdown is a well-known functional programming puzzle (with a spin-off TV show
in the UK), first popularised as a puzzle in which to demonstrate functional
algorithms in \cite{huttonCountdownProblem2002}.
The idea is simple: given a list of numbers, contestants must construct an
arithmetic expression (using a small set of functions) using some or all of the
numbers, to reach some target.

Take the following problem as an example:
\begin{gather*}
  \boxed{1} \boxed{3} \boxed{7} \boxed{10} \boxed{25} \boxed{50} \\
  \boxed{765} \tag{Target}
\end{gather*}
It has the following answer:
\begin{equation}
  3 \times ((7 \times (50 - 10)) - 25)
\end{equation}
Importantly, we do not have to use every number given to get to the target.
For our problem, we will use the operators \(\times\), \(+\), and \(-\).

In this section we will develop a program/proof which can decide countdown
problems totally.
In other words, given a list of numbers and a target, our program will prove
whether a solution exists, and if so, it will provide just that solution.
\section{Classifying The Problem}
So what is a ``solution'' to the countdown problem?
Put simply, it is a way to:
\begin{enumerate}
  \item Arrange some or all of the given numbers into a valid expression
  \item Such that the expression, when evaluated, is equal to the target.
\end{enumerate}
The first part of this solution is a transformation (from a list of numbers into
an expression), and the second is a filter (into valid expressions which
evaluate to the target).

In order to find and prove countdown solutions, we will prove that the above
transformation is \emph{finite}: this will allow us to search exhaustively
through the available transformations, proving the existence or absence of a
transformation which is a solution.

\begin{wrapfigure}{L}[\marginparwidth]{9cm}
  \tikzset{>=Triangle[open]}
  \begin{tikzpicture}
    \node (o1)  at (-1, 1.5) { 1} ;
    \node (o3)  at ( 0, 1.5) { 3} ;
    \node (o7)  at ( 1, 1.5) { 7} ;
    \node (o10) at ( 2, 1.5) {10} ;
    \node (o25) at ( 3, 1.5) {25} ;
    \node (o50) at ( 4, 1.5) {50} ;

    \node(t1) at (-1,  0) {\xcancel{1}}  ;
    \node(t3)   at (0 ,  0) {3}  ;
    \node(t7)   at (1 ,  0) {7}  ;
    \node(t10)  at (2 ,  0) {10} ;
    \node(t25)  at (3 ,  0) {25} ;
    \node(t50)  at (4 ,  0) {50} ;

    \draw [-{Rays[n=4]}] (o1.south)  -- (t1.north) ;
    \draw [->] (o3.south)  -- (t3.north) ;
    \draw [->] (o7.south)  -- (t7.north) ;
    \draw [->] (o10.south) -- (t10.north) ;
    \draw [->] (o25.south) -- (t25.north) ;
    \draw [->] (o50.south) -- (t50.north) ;

    \node(b3)   at (0 , -1.5) {3}  ;
    \node(b7)   at (1 , -1.5) {7}  ;
    \node(b50)  at (2 , -1.5) {50} ;
    \node(b10)  at (3 , -1.5) {10} ;
    \node(b25)  at (4 , -1.5) {25} ;
  
    \draw [->, rounded corners] (t3.south) --  (b3.north) ;
    \draw [->, rounded corners] (t7.south) --  (b7.north) ;
    \draw [->, rounded corners] (t10.south) to[out=-90, in=90] (b10.north) ;
    \draw [->, rounded corners] (t25.south) to[out=-90, in=90] (b25.north) ;
    \draw [->, rounded corners] (t50.south) to[out=-90, in=90] (b50.north) ;

    \node[draw] at (0.5 , -2.25) {$\times$} ;
    \node[draw] at (1.5 , -2.25) {$\times$} ;
    \node[draw] at (2.5 , -2.25) {$-$} ;
    \node[draw] at (3.5 , -2.25) {$-$} ;
  
    \node at (0.5 , -3) {$\times$} ;
    \node at (1.5 , -3) {$\times$} ;
    \node at (2.5 , -3) {$-$} ;
    \node at (3.5 , -3) {$-$} ;

    \node(e3) at (0   , -3) {3}  ;
    \node(e7) at (1   , -3) {7}  ;
    \node(e50) at (2   , -3) {50} ;
    \node(e10) at (3   , -3) {10} ;
    \node(e25) at (4   , -3) {25} ;

    \draw[->] (b3)  -- (e3) ;
    \draw[->] (b7)  -- (e7) ;
    \draw[->] (b50) -- (e50) ;
    \draw[->] (b10) -- (e10) ;
    \draw[->] (b25) -- (e25) ;

    \node(r40)  [circle, inner sep=0pt, fill=white, anchor=center] at (2.5, -4) {40} ;
    \node(r280) [circle, inner sep=0pt, fill=white, anchor=center] at (2  , -5) {280} ;
    \node(r255) [circle, inner sep=0pt, fill=white, anchor=center] at (2.5, -6) {255} ;
    \node(r765) [circle, inner sep=0pt, fill=white, anchor=center] at (2  , -7) {765} ;

    \draw[->] (e50)  to[out=-90, in=90] (r40) ;
    \draw[->] (e10)  to[out=-90, in=90] (r40) ;
    \draw[->] (e25)  to[out=-90, in=90] (r255) ;
    \draw[->] (e7)   to[out=-90, in=90] (r280) ;
    \draw[->] (e3)   to[out=-90, in=90] (r765) ;
    \draw[->] (r255) to[out=-90, in=90] (r765) ;
    \draw[->] (r280) to[out=-90, in=90] (r255) ;
    \draw[->] (r40)  to[out=-90, in=90] (r280) ;

    \node at (6, 0.7) {\parbox{2.2cm}{\subcaption{Filter     \hfill \; \label{countdown-filter}}}} ;
    \node at (6,-0.8) {\parbox{2.2cm}{\subcaption{Permutation\hfill \; \label{countdown-permutation}}}} ;
    \node at (6,-2.3) {\parbox{2.2cm}{\subcaption{Operators  \hfill \; \label{countdown-operators}}}} ;
    \node at (6,-5.3) {\parbox{2.2cm}{\subcaption{Parentheses\hfill \; \label{countdown-parens}}}} ;
  \end{tikzpicture}
  \caption{The components of a transformation which makes up a Countdown
    candidate solution}
  \label{countdown-transform}
\end{wrapfigure}

Let's first look at the transformation part.
As shown in figure~\ref{countdown-transform}, we can break the transformation
into four parts.
First, we notice that we don't have to use all of the given numbers in the
solution candidate: keeping or discarding input numbers, then, is the first step
of the transformation (fig.~\ref{countdown-filter}).
Next, we are allowed to rearrange the order of the numbers in our output: the
next step of the transformation, then, applies a permutation to the selected
numbers (fig.~\ref{countdown-permutation}).
Thirdly, we can choose the operators we're going to use: note that for \(n\)
chosen numbers, we can choose \(n - 1\) operators
(fig.~\ref{countdown-operators}).
Finally, we have to apply the binary operators with some parenthesising order
(fig.~\ref{countdown-parens}).

There are finitely many of each of these transformations: finitely many filters
for \(n\) objects, for instance, or finitely many tuples of \(n\) operators.
We are going to construct a type for each of these transformations, prove it
finite, and use that type to search.
The final, completed transformation will then be a dependent product of each
constituent step.


% \section{Filtering and Operators}
% \section{Enumerating Binary Trees}
% \section{Filtering Out Invalid Expressions with the Subobject Classifier}
% \section{Filtering Out Duplicates with Quotients}


%%% Local Variables:
%%% mode: latex
%%% TeX-master: "../paper"
%%% End: