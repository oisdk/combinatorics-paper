\chapter{Search} \label{search}
A common theme in dependently-typed programming is that proofs of interesting
theoretical things often correspond to useful algorithms in some way
related to that thing.
Finiteness is one such case: if we have a proof that a type \(A\) is finite,
we should be able to search through all the elements of that type in a
systematic, automated way.

As it happens, this kind of search is a very common method of proof automation
in dependently-typed languages like Agda.
Proofs of statements like ``the following function is associative''
\begin{agdalisting}
  \ExecuteMetaData[agda/Snippets/Bool.tex]{and-def}
\end{agdalisting}
can be tedious: the associativity proof in particular would take \(2^3 = 8\)
cases.
This is unacceptable!
There are only finitely many cases to examine, after all, and we're
\emph{already} on a computer: why not automate it?
A proof that \(\AgdaDatatype{Bool}\) is finite can get us much of the way to a
library to do just that.
\todo{These examples so far are pretty focused on the bool associativity
  example.
  I'm not sure I can think of a good way to put countdown in instead: will we
  try switch?
  Or will we keep the bool for this short bit?
  }

Similar automation machinery can be leveraged to provide search algorithms for
certain ``logic programming''-esque problems.
Using the machinery we will describe in this section, though, when the program
says it finds a solution to some problem that solution will be accompanied by a
formal \emph{proof} of its correctness.

In this section, we will describe the theoretical underpinning and
implementation of a library for proof search over finite domains, based on the
finiteness predicates we have introduced already.
The library will be able to prove statements like the proof of associativity
above, as well as more complex statements.
As a running example for a ``more complex statement'' we will use the countdown
problem, which we have been using throughout: we will demonstrate how to
construct a prover for the existence of, or absence of, a solution to a given
countdown puzzle.

The API for writing searches over finite domains comes from the language of the
\(\Pi\)-pretopos: with it we will show how to compose QuickCheck-like generators
for proof search, with the addition of some automation machinery that allows us
to prove things like the associativity in a couple of lines:
\begin{agdalisting} \label{bool-assoc-auto-proof}
  \ExecuteMetaData[agda/Snippets/Bool.tex]{bool-assoc-auto-proof}
\end{agdalisting}

We have already, in previous sections, explored the theoretical implications of
Cubical Type Theory on our formalisation.
With this library for proof search, however, we will see two distinct
practical applications which would simply not be possible without
computational univalence.
First and foremost: our proofs of finiteness, constructed with the API we will
describe, have all the power of full equalities.
Put another way any proof over a finite type \(A\) can be lifted to any other
type with the same cardinality.
Secondly our proof search can range over functions: we could, for instance, have
asked the prover to find if \emph{any} function over \(\AgdaDatatype{Bool}\) is
associative, and if so return it to us.
\begin{agdalisting}
  \ExecuteMetaData[agda/Snippets/Bool.tex]{some-assoc}
\end{agdalisting}
The usefulness of which is dubious, but we will see a more interesting
application soon.
\section{Omniscience}
So we now know what is needed of us for proof automation: we need to take our
proofs and make them decidable.
In particular, we need to be able to ``lift'' decidability back over a
function arrow.
For instance, given \(x\), \(y\), and \(z\) we already have
\(\AgdaDatatype{Dec}\;((x\;\AgdaFunction{∧}\;y)\;\AgdaFunction{∧}\;z\;\AgdaFunction{≡}\;x\;\AgdaFunction{∧}\;(y\;\AgdaFunction{∧}\;z))\)
(because equality over booleans is decidable).
In order to turn this into a proof that \AgdaFunction{∧} is associative we need
\(\AgdaDatatype{Dec}\;(\forall \; x \; y \; z \rightarrow (x\;\AgdaFunction{∧}\;y)\;\AgdaFunction{∧}\;z\;\AgdaFunction{≡}\;x\;\AgdaFunction{∧}\;(y\;\AgdaFunction{∧}\;z))\).
The ability to do this is described formally by the notion of
``Exhaustibility''.
\begin{agdalisting}
  \ExecuteMetaData[agda/Relation/Nullary/Omniscience.tex]{exhaustible}
\end{agdalisting}
We say a type \(A\) is exhaustible if, for any decidable predicate \(P\) on
\(A\), the universal quantification of the predicate is decidable.

This property of \AgdaDatatype{Bool} would allow us to automate the proof of
associativity, but it is in fact not strong enough to find individual
representatives of a type which support some property.
For that we need the more well-known related property of
\emph{omniscience}.
\begin{agdalisting}
  \ExecuteMetaData[agda/Relation/Nullary/Omniscience.tex]{omniscient}
\end{agdalisting}
The ``limited principle of omniscience''
\cite{bishopFoundationsConstructiveAnalysis1967} is a classical principle which
says that omniscience holds for all sets.
It doesn't hold constructively, of course: it lies a little bit below LEM in
terms of its non-constructiveness, given that it can be derived from LEM but LEM
cannot be derived from it.

Omniscience implies exhaustibility: we can use the usual rule of \todo{what is
  this called/from again?}
\begin{equation}
  \neg \exists x. P(x) \iff \forall x. \neg P(x)
\end{equation}
to turn omniscience for some predicate \(P\) into exhaustibility for some
predicate \(\neg \neg P\).
Usually we don't have double negation elimination constructively, but since
\(P\) is decidable it's actually present in this case:
\begin{agdalisting}
  \ExecuteMetaData[agda/Relation/Nullary/Decidable/Properties.tex]{dec-double-neg-elim}
\end{agdalisting}
All together, this gives us the following proof:
\begin{agdalisting}
  \ExecuteMetaData[agda/Relation/Nullary/Omniscience.tex]{omniscient-to-exhaustible}
\end{agdalisting}

Our focus here is on those types for which omniscience \emph{does} hold,
which includes the (ordered) finite types.
Perhaps surprisingly, it is not \emph{only} finite types which are exhaustible.
Certain infinite types can be
exhaustible~\cite{escardoInfiniteSetsThat2007}, but an exploration of that is
beyond the scope of this work.

All of the finiteness predicates imply exhaustibility.
To prove that fact we'll just show that the Kuratowski finite types are
exhaustible: since it's the weakest predicate, and can be derived from all the
others.
\begin{lemma}
  Kuratowski finiteness implies exhaustibility.
\end{lemma} \todo{Proof?}
Manifest enumerability is similarly the weakest of the ordered predicates, so we
will prove here that it implies omniscience.
\begin{lemma}
  Manifest enumerability implies omniscience.
\end{lemma} \todo{proof?}

Finally, there is a form of omniscience which works with Kuratowski finiteness:
propositional omniscience.
\begin{agdalisting}
  \ExecuteMetaData[agda/Relation/Nullary/Omniscience.tex]{prop-omniscient}
\end{agdalisting}
By truncating the returned \AgdaDatatype{\ensuremath{\Sigma}} we don't reveal
which \(A\) we've chosen which satisfies the predicate: this means that it can
be pulled out of the Kuratowski finite subset without issue.
\begin{agdalisting}
  \ExecuteMetaData[agda/Cardinality/Finite/Kuratowski.tex]{kuratowski-prop-omniscient}
\end{agdalisting}
\section{Countdown}
The Countdown problem~\cite{huttonCountdownProblem2002} is a well-known puzzle
in functional programming (which was apparently turned into a TV show).
As a running example in this paper, we will produce a verified program which
lists all solutions to a given countdown puzzle: here we will briefly explain
the game and our strategy for solving it.

\begin{wrapfigure}{L}[\marginparwidth]{9cm}
  \tikzset{>=Triangle[open]}
  \begin{tikzpicture}
    \node (o1)  at (-1, 1.5) { 1} ;
    \node (o3)  at ( 0, 1.5) { 3} ;
    \node (o7)  at ( 1, 1.5) { 7} ;
    \node (o10) at ( 2, 1.5) {10} ;
    \node (o25) at ( 3, 1.5) {25} ;
    \node (o50) at ( 4, 1.5) {50} ;

    \node(t1) at (-1,  0) {\xcancel{1}}  ;
    \node(t3)   at (0 ,  0) {3}  ;
    \node(t7)   at (1 ,  0) {7}  ;
    \node(t10)  at (2 ,  0) {10} ;
    \node(t25)  at (3 ,  0) {25} ;
    \node(t50)  at (4 ,  0) {50} ;

    \draw [-{Rays[n=4]}] (o1.south)  -- (t1.north) ;
    \draw [->] (o3.south)  -- (t3.north) ;
    \draw [->] (o7.south)  -- (t7.north) ;
    \draw [->] (o10.south) -- (t10.north) ;
    \draw [->] (o25.south) -- (t25.north) ;
    \draw [->] (o50.south) -- (t50.north) ;

    \node(b3)   at (0 , -1.5) {3}  ;
    \node(b7)   at (1 , -1.5) {7}  ;
    \node(b50)  at (2 , -1.5) {50} ;
    \node(b10)  at (3 , -1.5) {10} ;
    \node(b25)  at (4 , -1.5) {25} ;
  
    \draw [->, rounded corners] (t3.south) --  (b3.north) ;
    \draw [->, rounded corners] (t7.south) --  (b7.north) ;
    \draw [->, rounded corners] (t10.south) to[out=-90, in=90] (b10.north) ;
    \draw [->, rounded corners] (t25.south) to[out=-90, in=90] (b25.north) ;
    \draw [->, rounded corners] (t50.south) to[out=-90, in=90] (b50.north) ;

    \node[draw] at (0.5 , -2.25) {$\times$} ;
    \node[draw] at (1.5 , -2.25) {$\times$} ;
    \node[draw] at (2.5 , -2.25) {$-$} ;
    \node[draw] at (3.5 , -2.25) {$-$} ;
  
    \node at (0.5 , -3) {$\times$} ;
    \node at (1.5 , -3) {$\times$} ;
    \node at (2.5 , -3) {$-$} ;
    \node at (3.5 , -3) {$-$} ;

    \node(e3) at (0   , -3) {3}  ;
    \node(e7) at (1   , -3) {7}  ;
    \node(e50) at (2   , -3) {50} ;
    \node(e10) at (3   , -3) {10} ;
    \node(e25) at (4   , -3) {25} ;

    \draw[->] (b3)  -- (e3) ;
    \draw[->] (b7)  -- (e7) ;
    \draw[->] (b50) -- (e50) ;
    \draw[->] (b10) -- (e10) ;
    \draw[->] (b25) -- (e25) ;

    \node(r40)  [circle, inner sep=0pt, fill=white, anchor=center] at (2.5, -4) {40} ;
    \node(r280) [circle, inner sep=0pt, fill=white, anchor=center] at (2  , -5) {280} ;
    \node(r255) [circle, inner sep=0pt, fill=white, anchor=center] at (2.5, -6) {255} ;
    \node(r765) [circle, inner sep=0pt, fill=white, anchor=center] at (2  , -7) {765} ;

    \draw[->] (e50)  to[out=-90, in=90] (r40) ;
    \draw[->] (e10)  to[out=-90, in=90] (r40) ;
    \draw[->] (e25)  to[out=-90, in=90] (r255) ;
    \draw[->] (e7)   to[out=-90, in=90] (r280) ;
    \draw[->] (e3)   to[out=-90, in=90] (r765) ;
    \draw[->] (r255) to[out=-90, in=90] (r765) ;
    \draw[->] (r280) to[out=-90, in=90] (r255) ;
    \draw[->] (r40)  to[out=-90, in=90] (r280) ;

    \node at (6, 0.7) {\parbox{2.2cm}{\subcaption{Filter     \hfill \; \label{countdown-filter}}}} ;
    \node at (6,-0.8) {\parbox{2.2cm}{\subcaption{Permutation\hfill \; \label{countdown-permutation}}}} ;
    \node at (6,-2.3) {\parbox{2.2cm}{\subcaption{Operators  \hfill \; \label{countdown-operators}}}} ;
    \node at (6,-5.3) {\parbox{2.2cm}{\subcaption{Parentheses\hfill \; \label{countdown-parens}}}} ;
  \end{tikzpicture}
  \caption{The components of a transformation which makes up a Countdown
    candidate solution}
  \label{countdown-transform}
\end{wrapfigure}

The idea behind countdown is simple: given a list of numbers, contestants must
construct an arithmetic expression (using a small set of functions) using some
or all of the numbers, to reach some target.
Here's an example puzzle:
\begin{displayquote}
  Using some or all of the numbers 1, 3, 7, 10, 25, and 50 (using each at most
  once), construct an expression which equals 765.
\end{displayquote}
We'll allow the use of \(+\), \(-\), \(\times\), and \(\div\).
The answer is at the bottom of this page\footnotemark.

\footnotetext{\rotatebox[origin=c]{180}{Answer: \(3 \times (7 \times (50 - 10) - 25)\)}}

Our strategy for finding solutions to a given puzzle is to describe precisely
the type of solutions to a puzzle, and then show that that type is finite.
So what is a ``solution'' to a countdown puzzle?
Broadly, it has two parts:
\begin{description}
  \item[A Transformation] from a list of numbers to an expression.
  \item[A Predicate] showing that the expression is valid and evaluates to the
    target.
\end{description}
The first part is described in Figure~\ref{countdown-transform}.


This transformation has four steps.
First (Fig.~\ref{countdown-selection}) we have to pick which numbers we include
in our solution.
We will need to show there are finitely many ways to filter \(n\) numbers.

Secondly (Fig.~\ref{countdown-permutation}) we have to permute the chosen
numbers.
The representation for a permutation is a little trickier to envision: proving
that it's finite is trickier still.
We will need to rely on some of the more involved lemmas later on for this
problem.

The third step (Fig.~\ref{countdown-operators}) is a vector of length \(n\) of finite objects (in this case operators
chosen from \(+\), \(\times\), \(-\), and \(\div\)).
Although it is complicated slightly by the fact that the \(n\) in this
\(n\)-tuple is dependent on the amount of numbers we let through in the filter
in step one.
(in terms of types, that means we'll need a \(\Sigma\) rather than a
\(\times\), explanations of which are forthcoming).

Finally (Fig.~\ref{countdown-parens}), we have to parenthesise the expression in
a certain way.
This can be encapsulated by a binary tree with a certain number of leaves:
proving that that is finite is tricky again.

Once we have proven that there are finitely many transformations for a list of
numbers, we will then have to filter them down to those transformations which
are valid, and evaluate to the target.
This amounts to proving that the decidable subset of a finite set is also
finite.

Finally, we will also want to optimise our solutions and solver: for this we
will remove equivalent expressions, which can be accomplished with quotients.
We have already introduced and described countdown: in this section, we will
fill in the remaining parts of the solver, glue the pieces together, and show
how the finiteness proofs can assist us to write the solver.
\subsection{Finite Vectors}
We'll start with a simple example: for both the selection
(Fig.~\ref{countdown-selection}) and operators (Fig.~\ref{countdown-operators})
section, all we need to show is that a vector of some finite type is itself
finite.
To describe which elements to keep from an \(n\)-element list, so instance, we
only need a vector of Booleans of length \(n\).
Similarly, to pick \(n\) operators requires us only to provide a vector of \(n\)
operators.
And we can prove in a straightforward way that a vector of finite things is
itself finite.
\begin{agdalisting}
  \ExecuteMetaData[agda/Countdown.tex]{vec-fin}
\end{agdalisting}
We've already shown that there are finitely many booleans, the fact that there
are finitely many operators is similarly simple to prove:
\begin{agdalisting}
  \ExecuteMetaData[agda/Countdown.tex]{op-fin}
\end{agdalisting}
\subsection{Finite Permutations}
A more complex, and interesting, step of the transformation is the first step
(Fig.~\ref{countdown-permutation}), where we need to specify the permutation to
apply to the chosen numbers.

Our first attempt at representing permutations might look something like this:
\begin{agdalisting}
  \ExecuteMetaData[agda/Countdown.tex]{wrong-perm}
\end{agdalisting}
the idea is that \(\AgdaDatatype{Perm}\;n\) represents a permutation of \(n\)
things, as a function from positions to positions.
Unfortunately such a simple answer won't work: there are no restrictions on the
operation of the function, so it could (for instance), send more than one input
position into the same output.

What we actually need is not just a function between positions, but an
\emph{isomorphism} between them.
In types:
\begin{agdalisting}
  \ExecuteMetaData[agda/Countdown.tex]{iso-perm}
\end{agdalisting}
Where an isomorphism is defined as follows:
\begin{agdalisting}
  \ExecuteMetaData[agda/Countdown.tex]{isomorphism}
\end{agdalisting}
\todo{Should this isomorphism definition be put earlier in the intro with the
  equivalences etc?}
While it may look complex, this term is actually composed of individual
components we've already proven finite.
First we have \(\AgdaDatatype{Fin}\;n\rightarrow\AgdaDatatype{Fin}\;n\):
functions between finite types are, as we know, finite
(Theorem~\ref{split-enum-pi}).
We take a pair of them: pairs of finite things are \emph{also} finite
(Lemma~\ref{split-enum-sigma}).
To get the next two components we can filter to the subobject: this requires
these predicates to be decidable. \todo{Need to do filter subobject in topos section}
We will construct a term of the following type:
\begin{agdalisting}
  \ExecuteMetaData[agda/Cardinality/Finite/ManifestBishop.tex]{dec-type}
\end{agdalisting}
So can we construct such a term? Yes!

We basically need to construct decidable equality for functions between
\(\AgdaDatatype{Fin}\;n\)s: of course, this decidable equality is provided by
the fact that such functions are themselves finite.

All in all we can now prove that the isomorphism, and by extension the
permutation, is finite:
\begin{agdalisting}
  \ExecuteMetaData[agda/Cardinality/Finite/ManifestBishop.tex]{iso-finite}
\end{agdalisting}

Unfortunately this implementation is too slow to be useful.
As nice and declarative as it is, fundamentally it builds a list of all possible
pairs of functions between \(\AgdaDatatype{Fin}\;n\) and itself (an operation
which takes in the neighbourhood of \(\mathcal{O}(n^n)\) time), and then tests
each for equality (which is likely worse than \(\mathcal{O}(n^2)\) time).
We will instead use a factoriadic encoding: this is a relatively simple encoding
of permutations which will reduce our time to a blazing fast
\(\mathcal{O}(n!)\).
It is expressed in Agda as follows:
\begin{agdalisting}
  \ExecuteMetaData[agda/Countdown.tex]{perm-def}
\end{agdalisting}
It is a vector of positions, each represented with a \(\AgdaDatatype{Fin}\).
Each position can only refer to the length of the tail of the list at that
point: this prevents two input positions mapping to the same output point, which
was the problem with the naive encoding we had previously.
And it also has a relatively simple proof of finiteness:
\begin{agdalisting}
  \ExecuteMetaData[agda/Countdown.tex]{perm-fin}
\end{agdalisting}
\subsection{Parenthesising}
Our next step is figuring out a way to encode the parenthesisation of the
expression (Fig.~\ref{countdown-parens}). \todo{There's no way
  ``parenthesisation'' is a real word}
At this point of the transformation, we already have our numbers picked out, we
have ordered them a certain way, and we have also selected the operators we're
interested in.
We have, in other words, the following:
\begin{equation}
  3 \times 7 \times 50 - 10 - 25
\end{equation}
Without parentheses, however, (or a religious adherence to BOMDAS) this
expression is still ambiguous.
\begin{align}
  3 \times ((7 \times (50 - 10)) - 25) &= 765 \\
  (((3 \times 7) \times 50) - 10) - 25 &= 1015
\end{align}
The different ways to parenthesise the expression result in different outputs
of evaluation.

So what data type encapsulates the ``different ways to parenthesise'' a given
expression?
That's what we will figure out in this section, and what we will prove finite.

One way to approach the problem is with a binary tree.
A binary tree with \(n\) leaves corresponds in a straightforward way to a
parenthesisation of \(n\) numbers (Fig.~\ref{countdown-parens}).
\todo{Tree diagram? Or link to previous tree?}
This doesn't get us much closer to a finiteness proof, however: for that we will
need to rely on \emph{Dyck} words.
\begin{definition}[Dyck words]
  A Dyck word is a string of balanced parentheses.
  In Agda, we can express it as the following:
  \begin{agdalisting}
    \ExecuteMetaData[agda/Dyck.tex]{dyck-def}
  \end{agdalisting}
  A fully balanced string of \(n\) parentheses has the type
  \(\AgdaDatatype{Dyck}\;\AgdaInductiveConstructor{zero}\;n\).
  Here are some example strings:
  \begin{multicols}{2}
    \begin{agdalisting}
      \ExecuteMetaData[agda/Dyck.tex]{dyck-0-2}
    \end{agdalisting}
    \begin{agdalisting}
      \ExecuteMetaData[agda/Dyck.tex]{dyck-0-3}
    \end{agdalisting}
  \end{multicols}
  The first parameter on the type represents the amount of unbalanced closing
  parens, for instance:
  \begin{agdalisting}
    \ExecuteMetaData[agda/Dyck.tex]{dyck-1-2}
  \end{agdalisting}
\end{definition}

Already Dyck words look easier to prove finite than straight binary trees, but
for that proof to be useful we'll have to relate Dyck words and binary trees
somehow.
As it happens, Dyck words of length \(2n\) are isomorphic to binary trees with
\(n-1\) leaves, but we only need to show this relation in one direction: from
Dyck to binary tree.
To demonstrate the algorithm we'll use a simple tree definition:
\begin{agdalisting}
  \ExecuteMetaData[agda/Dyck.tex]{tree-simpl-def}
\end{agdalisting}
The algorithm itself is quite similar to stack-based parsing algorithms.
\begin{agdalisting}
  \ExecuteMetaData[agda/Dyck.tex]{from-dyck}
\end{agdalisting}
\subsection{Putting It All Together}
At this point we have each of the four components of the transformation defined.
From this we can define what an expression is:
\begin{agdalisting}
  \ExecuteMetaData[agda/Countdown.tex]{expr-def}
\end{agdalisting}
Notice that we don't allow expressions with no numbers.

The proof that this type is finite mirrors its definition closely:
\begin{agdalisting}
  \ExecuteMetaData[agda/Countdown.tex]{expr-finite}
\end{agdalisting}
\subsection{Filtering to Correct Expressions}
We now have a way to construct, formally, every expression we can generate from
a given list of numbers.
This is incomplete in two ways, however.
Firstly, some expressions are invalid: we should not, for instance, be able to
divide two numbers which do not divide evenly.
Secondly, we are only interested in those expressions which actually represent
solutions: those which evaluate to the target, in other words.
We can write a function which tells us if both of these things hold for a given
expression like so:

\begin{minipage}{\linewidth}
  \begin{multicols}{2}
    \begin{agdalisting}
      \ExecuteMetaData[agda/Countdown.tex]{eval}
    \end{agdalisting} \columnbreak
    \begin{agdalisting}
      \ExecuteMetaData[agda/Countdown.tex]{app-op}
    \end{agdalisting}
  \end{multicols}
\end{minipage}

With this all together, we can finally write down the type of all solutions to a
given countdown problem.
\begin{agdalisting}
  \ExecuteMetaData[agda/Countdown.tex]{solution-type}
\end{agdalisting}
And, because the predicate here is decidable and a mere proposition, we can
prove that there are finitely many solutions:
\begin{agdalisting}
  \ExecuteMetaData[agda/Countdown.tex]{finite-solution}
\end{agdalisting}
And we can apply this to a particular problem like so:
\begin{agdalisting}
  \ExecuteMetaData[agda/Countdown.tex]{example-solution}
\end{agdalisting}
\section{Proof Automation}
So we have shown now how powerful the library is for proof search: it can
search for functions, subsets of types, products, and so on, completely
verifiable.
In this section we will present the more user-friendly interface to the library,
designed to be used to automate away tedious proofs in an easy way.
\subsection{How to make the Typechecker do Automation}
For this prover we will not resort to reflection or similar techniques: instead,
we will trick the type checker to do our automation for us.
This is a relatively common technique, although not so much outside of Agda, so
we will briefly explain it here.

To understand the technique we should first notice that some proof automation
\emph{already} happens in Agda, like the following:
\begin{agdalisting}
  \ExecuteMetaData[agda/Snippets/Bool.tex]{obvious}
\end{agdalisting}
The type checker does not require us to manually explain each step of evaluation
of
\(\AgdaInductiveConstructor{true}\;\AgdaFunction{∧}\;\AgdaInductiveConstructor{false}\).
While it's not a particularly impressive example of automation, it does nonetheless
demonstrate a principle we will exploit: closed terms will compute to a normal
form if they're needed to type check.
The type checker will perform \(\beta\)-reduction as much as it can.

So our task is to rewrite proof obligations like the one in
Equation~\ref{bool-assoc-auto-proof} into ones which can reduce completely. 
As it turns out, we have already described the type of proofs which can ``reduce
completely'': \emph{decidable} proofs.
If we have a decision procedure over some proposition \(P\) we can run that
decision during type checking, because the decision procedure itself is a proof
that the decision will terminate.
In code, we capture this idea with the following pair of functions:
\begin{multicols}{2}
  \begin{agdalisting}
    \ExecuteMetaData[agda/Snippets/Bool.tex]{is-true}
  \end{agdalisting} \columnbreak
  \begin{agdalisting}
    \ExecuteMetaData[agda/Snippets/Bool.tex]{from-true}
  \end{agdalisting}
\end{multicols}
The first is a function which derives a type from whether a decision is
successful or not.
This function is important because if we use the output of this type at any
point we will effectively force the unifier to run the decision computation.
The second takes---as an implicit argument---an inhabitant of the type generated
from the first, and uses it to prove that the decision can only be true, and the
extracts the resulting proof from that decision.
All in all, we can use it like this:
\begin{agdalisting}
  \ExecuteMetaData[agda/Snippets/Bool.tex]{extremely-obvious}
\end{agdalisting}
This technique will allow us to automatically compute any decidable predicate.
\subsection{Building an Interface}
We will next look at a syntax and interface to this proof technique.
It starts with the following two functions:

\begin{minipage}{\linewidth}
  \begin{multicols}{2}
    \begin{agdalisting}
      \ExecuteMetaData[agda/Cardinality/Finite/SplitEnumerable/Search.tex]{forall-ask}
    \end{agdalisting} \columnbreak
    \begin{agdalisting}
      \ExecuteMetaData[agda/Cardinality/Finite/SplitEnumerable/Search.tex]{exists-ask}
    \end{agdalisting}
  \end{multicols}
\end{minipage}
Clearly they're just restatements of exhaustibility and omniscience.
However, we can combine these functions with what we've constructed above to
create an automation procedure for each:

\begin{minipage}{\linewidth}
  \begin{multicols}{2}
    \begin{agdalisting}
      \ExecuteMetaData[agda/Cardinality/Finite/SplitEnumerable/Search.tex]{forall-zap}
    \end{agdalisting} \columnbreak
    \begin{agdalisting}
      \ExecuteMetaData[agda/Cardinality/Finite/SplitEnumerable/Search.tex]{exists-zap}
    \end{agdalisting}
  \end{multicols}
\end{minipage}

This automation procedure allows us to state the property succinctly, and have
the type checker go and run the decision procedure to solve it for us.
Here's an example of its use:
\begin{agdalisting}
  \ExecuteMetaData[agda/Snippets/Bool.tex]{pre-inst-proof}
\end{agdalisting} \todo{Include counterexample stuff?}

\subsection{Instances}
One bit of cruft in the above proof is the need to specify the particular
finiteness proof for bools.
While this isn't any great burden in this case, it of course becomes more
difficult in more complex circumstances.

To solve this we can use Agda's instance search.
This changes the definitions of
our automation functions to the following:

\begin{minipage}{\linewidth}
  \begin{multicols}{2}
    \begin{agdalisting}
      \ExecuteMetaData[agda/Cardinality/Finite/SplitEnumerable/Search.tex]{forall-zap-inst}
    \end{agdalisting} \columnbreak
    \begin{agdalisting}
      \ExecuteMetaData[agda/Cardinality/Finite/SplitEnumerable/Search.tex]{exists-zap-inst}
    \end{agdalisting}
  \end{multicols}
\end{minipage}
And this also changes the idempotency proof to the following:
\begin{agdalisting}
  \ExecuteMetaData[agda/Snippets/Bool.tex]{with-inst-proof}
\end{agdalisting}

Again, there's not any great revelation in ease of use here, but more complex
examples really benefit.
Especially when we build the full set of instances: any expression built out of
products and sums will automatically have an instance.
This will allow us, for instance, to perform proof search over tuples, which
gives us some degree of automation for proof search in tuples.
\begin{agdalisting}
  \ExecuteMetaData[agda/Snippets/Bool.tex]{and-comm}
\end{agdalisting}
\subsection{Generic Currying and Uncurrying}
While we have arguably removed the bulk of the boilerplate from the automated
proofs, there is still the case of the ugly noise of currying and uncurrying.
In this section, we take inspiration from
\citet{allaisGenericLevelPolymorphic2019} to develop a small interface to generic
\(n\)-ary functions and properties.
We will describe it briefly here.

The basic idea of currying and uncurrying generically is to allow ourselves to
work with a generic and flexible representation of function arguments which can
be manipulated more easily than a simple function itself.
Our first task, then, is to define that representation of function arguments.
As in \citet{allaisGenericLevelPolymorphic2019}, our representation is a tuple
which is in some sense a ``second order'' indexed type.
By second order here we mean that it is an indexed type indexed by another
indexed type.
The reason for this complexity is that our solution is to be fully
level-polymorphic.
To start, we define a type representing a vector of universe levels:

\begin{minipage}{\linewidth}
  \begin{multicols}{2}
    \begin{agdalisting}
      \ExecuteMetaData[agda/Data/Product/NAry.tex]{levels}
    \end{agdalisting} \columnbreak
    \begin{agdalisting}
      \ExecuteMetaData[agda/Data/Product/NAry.tex]{max-level}
    \end{agdalisting}
  \end{multicols}
\end{minipage}

This will be used to assign our tuple the correct universe level generically.
Next, we define the list of types (this type is indexed by the list of universe
levels of each type):
\begin{agdalisting}
  \ExecuteMetaData[agda/Data/Product/NAry.tex]{types}
\end{agdalisting}
And finally, the tuple, indexed by its list of types:

\begin{minipage}{\linewidth}
  \begin{multicols}{2}
    \begin{agdalisting}
      \ExecuteMetaData[agda/Data/Product/NAry.tex]{tuple-plus}
    \end{agdalisting} \columnbreak
    \begin{agdalisting}
      \ExecuteMetaData[agda/Data/Product/NAry.tex]{tuple}
    \end{agdalisting}
  \end{multicols}
\end{minipage}
The reason for two separate functions here is to avoid the
\(\AgdaDatatype{\(\top\)}\)-terminated tuples we would need if we just had one.
This means that, for instance, to represent a tuple of a \AgdaDatatype{Bool} and
\agdambb{N} we can write
\((\AgdaInductiveConstructor{true}\;\AgdaInductiveConstructor{,}\;\AgdaNumber{2})\)
instead of 
\((\AgdaInductiveConstructor{true}\;\AgdaInductiveConstructor{,}\;\AgdaNumber{2}\;\AgdaInductiveConstructor{,}\;\AgdaInductiveConstructor{tt})\).

Next we turn to how we will represent functions.
In Agda there are three ways to pass function arguments: explicitly,
implicitly, and as an instance.
We will represent these three different versions with a data type:
\begin{agdalisting}
  \ExecuteMetaData[agda/Data/Product/NAry.tex]{arg-form}
\end{agdalisting}

What we provide here differs from that work in the following ways:
\begin{itemize}
  \item Our generic representation can handle dependent \(\sum\) and \(\prod\)
    types (rather than their non-dependent counterparts, \(\times\) and
    \(\rightarrow\)).
    This extension was necessary for our use case: it is mentioned in the paper
    as the obvious next step. 
  \item We implement the curry-uncurry combinators as (verified) isomorphisms.
    Since we are in a cubical setting, this gives us equivalences between the
    types, a feature not available in standard Agda.
  \item We deal with implicit and instance arguments generically.
\end{itemize}
A full explanation of our implementation is beyond the scope of this work, but
we will mention the key parts here.
First, we define a function arrow generic over the application method:
% \ExecuteMetaData[agda/Data/Product/N-ary.tex]{fun-arr}
Then, we prove that it is isomorphic to the normal function arrow:
% \ExecuteMetaData[agda/Data/Product/N-ary.tex]{fun-iso}
This step will allow us to write the curry-uncurry proofs once, and then extend
them to the three different argument forms without difficulty.

We do the same with dependent function types:
% \ExecuteMetaData[agda/Data/Product/N-ary.tex]{pi-iso}

Next, we need to make our machinery for multiple arguments.
Here we will define a generic tuple, indexed by some \(\Nat\).
As observed in \cite{allaisGenericLevelPolymorphic2019}, it's important to
\emph{not} implement this as an inductively defined vector, e.g.
% \ExecuteMetaData[agda/Data/Vec.tex]{def}
As this will not give us the correct \(\eta\)-equality we need for unification.
\todo[noline]{Stick reference to iterated Vec definition}
Once we have a unification-friendly vector type, we can use it to implement our
generic (and level-polymorphic) tuples.
% \ExecuteMetaData[agda/Data/Product/N-ary.tex]{tuple}
We decided against \(\top\)-terminated tuples, as the actual contents of the
tuple type are exposed in the interface.

Finally, we can prove isomorphisms between the curried and uncurried versions of
functions:
% \ExecuteMetaData[agda/Data/Product/N-ary.tex]{curry-n-iso}
And dependent functions:
% \ExecuteMetaData[agda/Data/Product/N-ary.tex]{curry-n-pi-iso}
With these combinators, we can implement the search functions in an
arity-generic way:
% \ExecuteMetaData[agda/Cardinality/Finite/Search.tex]{all-nary}
And automate away our proofs:
\ExecuteMetaData[agda/Data/Pauli.tex]{prf-curried}

%%% Local Variables:
%%% mode: latex
%%% TeX-master: "../paper"
%%% End: