\PassOptionsToPackage{dvipsnames}{xcolor}
\documentclass[draft]{memoir}
\chapterstyle{hangnum}
\hangsecnum

\usepackage{url}

\ifdraftdoc
\newcommand{\todo}[1]{\sidepar{TODO: \\ #1}}
\else
\newcommand{\todo}[1]{}
\fi

\usepackage[mono=false]{libertine}
\usepackage{catchfilebetweentags}
\input{agda}
\AgdaNoSpaceAroundCode{}

\usepackage{tikz-cd}
\tikzcdset{
  arrow style=tikz,
  diagrams={>={Straight Barb[scale=0.8]}},
}
\usepackage{amsmath}
\usepackage{amsthm}
\renewcommand{\qed}{\ensuremath{\hfill\blacksquare}}

% Boxed theorem envs without italics

\usepackage[framemethod=tikz]{mdframed}

\newmdenv[
  linecolor=white,
  tikzsetting={draw=black, line width = 1pt,dotted},
]{dottedbox}
% If a figure is desired:
\newenvironment{agdalisting}{\begin{marginfigure}\begin{dottedbox}\footnotesize}{\end{dottedbox}\end{marginfigure}}
% \newenvironment{agdalisting}{\begin{marginpar}\begin{dottedbox}}{\end{dottedbox}\end{marginpar}}



\newtheoremstyle{roman}
{.6\baselineskip} % Space above
{.6\baselineskip} % Space below
{} % Body font
{} % Indent amount
{} % Theorem head font
{} % Punctuation after theorem head
{0pt} % Space after theorem head
{%
\noindent%
\raisebox{0pt}[0pt][0pt]{%
\llap{%
\parbox[t][][t]{.7\marginparwidth}{%
\raggedleft%
{\bfseries\thmname{#1}\thmnumber{ #2}.}\thmnote{\\(#3)}}\hspace{\marginparsep}}}}

\theoremstyle{roman}
\newtheorem{definition}{Definition}
\newtheorem{theorem}{Theorem}
\newtheorem{lemma}[theorem]{Lemma}


% Haskell bind operator >>=
\newcommand\hbind{%
  \ensuremath{\gg\mkern-5.5mu=}%
}
% Haskell apply operator <*>
\newcommand\hap{%
  \ensuremath{\mathbin{<\mkern-9mu*\mkern-9mu>}}%
}
% Haskell alternative operator <|>
\newcommand\halt{%
  \ensuremath{\mathbin{<\mkern-5mu\vert\mkern-5mu>}}%
}
% Haskell fmap operator <$>$
\newcommand\hfmap{%
  \ensuremath{\mathbin{<\mkern-6mu\$\mkern-6mu>}}%
}
% Haskell mappend operator <>
\newcommand\hcmb{%
  \ensuremath{\mathbin{\diamond}}%
}
% Haskell fish operator <><
\newcommand\hcmbin{%
  \ensuremath{\mathbin{\diamond\mkern-7.4mu\rtimes}}%
}
% Haskell Kleisli composition <=<
\newcommand\hkcomp{%
  \ensuremath{\mathbin{<\mkern-10mu=\mkern-7mu<}}
}
% Haskell forall
\newcommand\hforall{%
  \ensuremath{\forall}%
}
% Unique membership \in!
\newcommand\inunique{%
  \mathrel{\in\mkern-6mu\raisebox{-0.5pt}{!}}
}
% Decide membership \in?
\newcommand\decin{%
  \mathrel{\in\mkern-6mu\raisebox{-0.5pt}{?}}
}
% Dot as wildcard character
\newcommand\wc{{\mkern 2mu\cdot\mkern 2mu}}
% Binary operator !
\newcommand\ind{\mathbin{!}}
\usepackage{subcaption}
\usepackage{locallhs2TeX}
\renewcommand{\Varid}[1]{\mathit{#1}}
\renewcommand{\Conid}[1]{\textcolor{OliveGreen}{\textsf{#1}}}
\usepackage{multicol}
\usepackage{tabularx}
\usepackage{pict2e}
\usepackage{forest}

% Proper double left paren [( (also ``plano-concave lens'')
\newcommand{\lbparen}{%
  \mathopen{%
    \sbox0{$()$}%
    \setlength{\unitlength}{\dimexpr\ht0+\dp0}%
    \raisebox{-\dp0}{%
      \begin{picture}(.32,1)
        \linethickness{\fontdimen8\textfont3}
        \roundcap
        \put(0,0){\raisebox{\depth}{$($}}
        \polyline(0.32,0)(0,0)(0,1)(0.32,1)
      \end{picture}%
    }%
  }%
}

% Proper double right paren [( (also ``plano-concave lens'')
\newcommand{\rbparen}{%
  \mathclose{%
    \sbox0{$()$}%
    \setlength{\unitlength}{\dimexpr\ht0+\dp0}%
    \raisebox{-\dp0}{%
      \begin{picture}(.32,1)
        \linethickness{\fontdimen8\textfont3}
        \roundcap
        \put(-0.08,0){\raisebox{\depth}{$)$}}
        \polyline(0,0)(0.32,0)(0.32,1)(0,1)
      \end{picture}%
    }%
  }%
}

\usepackage[square,numbers]{natbib}

\usepackage{cancel}

\PrerenderUnicode{í}

% \usepackage{hyperref}

% Nice typesetting for the definition of an inductive type.
% For the following Agda-like thing:
%
% data List a where
%   []   : List a
%   (::) : a -> List a -> List a    
%
% write:
% \begin{inductivetype}{\mathbf{List}(A)}
%   \inductivetypeclause{\left[ \right]}{\mathbf{List}(A)}
%   \inductivetypeclause{\wc \dblcolon \wc}{A \rightarrow \mathbf{List}(A) \rightarrow \mathbf{List}(A)}
% \end{inductivetype}

\newif\iffirstclauseofinductivetype
\newenvironment{inductivetype}[1]{%
  \firstclauseofinductivetypetrue
  \begin{equation}
    \begin{alignedat}{3}
      #1
    }{%
    \end{alignedat}
  \end{equation}%
}
\newcommand{\inductivetypeclause}[2]{%
  \iffirstclauseofinductivetype
    \coloneqq
  \else
    \\ \vert
  \fi
  & \; #1 &:&& \; #2 ;
  \firstclauseofinductivetypefalse
}
% \mdfdefinestyle{proof}{
%   leftmargin = 0pt,
%   innerleftmargin = 1em,
%   innertopmargin = 0pt,
%   innerbottommargin = 0pt,
%   innerrightmargin = 0pt,
%   rightmargin = 0pt,
%   % linewidth = 1pt,
%   topline = false,
%   rightline = false,
%   bottomline = false,
% }
% \surroundwithmdframed[style=proof]{proof}

%%% Local Variables:
%%% mode: latex
%%% TeX-master: "paper"
%%% End: