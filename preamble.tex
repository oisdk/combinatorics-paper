\PassOptionsToPackage{dvipsnames}{xcolor}
\documentclass{article}

\usepackage[mono=false]{libertine}
\usepackage{catchfilebetweentags}
\usepackage{mathtools}
\usepackage{mathrsfs}
\usepackage{turnstile}
\usepackage{bbm}
\usepackage[greek, english]{babel}
\usepackage{stmaryrd}
\usepackage{latexsym}
\newcommand\doubleplus{+\kern-1.3ex+\kern0.8ex}
\newcommand\mdoubleplus{\ensuremath{\mathbin{+\mkern-8mu+}}}
% \usepackage[dvipsnames]{xcolor}
\usepackage{relsize}
\newcommand\doublelbrace{%
  {\{\mkern-4mu\lvert}%
}
\newcommand\doublerbrace{%
  {\rvert\mkern-4mu\}}%
}

\makeatletter
\newcommand\incircbin
{%
  \mathpalette\@incircbin
}
\newcommand\@incircbin[2]
{%
  \mathbin%
  {%
    \ooalign{\hidewidth$#1#2$\hidewidth\crcr$#1\bigcirc$}%
  }%
}
\newcommand{\oeq}{\ensuremath{\incircbin{=}}}
\makeatother
\makeatletter
\newcommand\insquarebin
{%
  \mathpalette\@insquarebin
}
\newcommand\@insquarebin[2]
{%
  \mathbin%
  {%
    \ooalign{\hidewidth$#1#2$\hidewidth\crcr$#1\bigbox$}%
  }%
}
\newcommand{\sqtri}{\ensuremath{\insquarebin{\triangle}}}
\makeatother
\newcommand{\circeps}{\ensuremath{\insquarebin{\epsilon}}}
\makeatother
\newcommand{\qeq}{\mathrel{\mathpalette\qeqf\relax}}
\newcommand{\qeqf}[2]{%
  \ooalign{$#1=$\cr
    \hidewidth\raisebox{1ex}{$#1\mkern0mu{\mathsmaller{\mathsmaller{?}}}$}\hidewidth\cr}}
\usepackage{ucs}
\DeclareUnicodeCharacter{8605}{\ensuremath{\leadsto}}
\DeclareUnicodeCharacter{951}{\textgreek{\texteta}}
\DeclareUnicodeCharacter{957}{\textgreek{\textnu}}
\DeclareUnicodeCharacter{961}{\textgreek{\textrho}}
\DeclareUnicodeCharacter{929}{\textgreek{\textRho}}
\DeclareUnicodeCharacter{954}{\textgreek{\textkappa}}
\DeclareUnicodeCharacter{928}{\textgreek{\textPi}}
\DeclareUnicodeCharacter{922}{\textgreek{\textKappa}}
\DeclareUnicodeCharacter{931}{\textgreek{\textSigma}}
\DeclareUnicodeCharacter{916}{\textgreek{\textDelta}}
\DeclareUnicodeCharacter{921}{\textgreek{\textIota}}
\DeclareUnicodeCharacter{8759}{\ensuremath{\dblcolon}}
\DeclareUnicodeCharacter{737}{\ensuremath{^\text{l}}}
\DeclareUnicodeCharacter{691}{\ensuremath{^\text{r}}}
\DeclareUnicodeCharacter{7523}{\ensuremath{_\text{r}}}
\DeclareUnicodeCharacter{8343}{\ensuremath{_\text{l}}}
\DeclareUnicodeCharacter{8718}{\ensuremath{\blacksquare}}
\DeclareUnicodeCharacter{10214}{\ensuremath{\llbracket}}
\DeclareUnicodeCharacter{10215}{\ensuremath{\rrbracket}}
\DeclareUnicodeCharacter{8857}{\ensuremath{\odot}}
\DeclareUnicodeCharacter{8860}{\oeq}
\DeclareUnicodeCharacter{9043}{\ensuremath{\sqtri}}
\DeclareUnicodeCharacter{3870}{\ensuremath{\cong}}
\DeclareUnicodeCharacter{8799}{\ensuremath{\qeq}}
\DeclareUnicodeCharacter{10181}{\ensuremath{\lbag}}
\DeclareUnicodeCharacter{10182}{\ensuremath{\rbag}}
\DeclareUnicodeCharacter{8760}{\ensuremath{-}}
\DeclareUnicodeCharacter{9428}{\ensuremath{\textcircled{ε}}}
\DeclareUnicodeCharacter{8623}{\ensuremath{\lightning}}
\DeclareUnicodeCharacter{10627}{\ensuremath{\doublelbrace}}
\DeclareUnicodeCharacter{10628}{\ensuremath{\doublerbrace}}
\DeclareUnicodeCharacter{10631}{\ensuremath{\llparenthesis}}
\DeclareUnicodeCharacter{10632}{\ensuremath{\rrparenthesis}}
\DeclareUnicodeCharacter{9666}{\ensuremath{\scriptscriptstyle\blacktriangleleft}}
\DeclareUnicodeCharacter{8855}{\ensuremath{\otimes}}
\DeclareUnicodeCharacter{894}{\ensuremath{\fatsemi}}
\DeclareUnicodeCharacter{65373}{\ensuremath{\}}}
\DeclareUnicodeCharacter{65371}{\ensuremath{\{}}
\DeclareUnicodeCharacter{120021}{\ensuremath{\mathcal{F}}}
\DeclareUnicodeCharacter{120017}{\ensuremath{\mathcal{B}}}
\DeclareUnicodeCharacter{8761}{\ensuremath{\coloneq}}
\DeclareUnicodeCharacter{8788}{\ensuremath{\mathrel{\mathop:}=}}
\DeclareUnicodeCharacter{119974}{\ensuremath{\mathcal{K}}}
\DeclareUnicodeCharacter{8492}{\ensuremath{\mathcal{B}}}
\DeclareUnicodeCharacter{8495}{\ensuremath{\mathcal{E}}}
\DeclareUnicodeCharacter{8466}{\ensuremath{\mathcal{L}}}
\DeclareUnicodeCharacter{8779}{\ensuremath{\equiv}}
\DeclareUnicodeCharacter{10626}{\ensuremath{:}}
\DeclareUnicodeCharacter{10629}{\ensuremath{\llparenthesis}}
\DeclareUnicodeCharacter{10630}{\ensuremath{\rrparenthesis}}
\DeclareUnicodeCharacter{9671}{\ensuremath{\Diamond}}
\DeclareUnicodeCharacter{9723}{\ensuremath{\Box}}
\DeclareUnicodeCharacter{9655}{\ensuremath{\triangleright}}
\DeclareUnicodeCharacter{7584}{\ensuremath{^\text{f}}}
\DeclareUnicodeCharacter{8346}{\ensuremath{_\text{p}}}
\DeclareUnicodeCharacter{8621}{\ensuremath{\leftrightsquigarrow}}
\DeclareUnicodeCharacter{12310}{\ensuremath{\lbparen}}
\DeclareUnicodeCharacter{12311}{\ensuremath{\rbparen}}
\DeclareUnicodeCharacter{7470}{\ensuremath{^\text{B}}}
\DeclareUnicodeCharacter{7468}{\ensuremath{^\text{A}}}
\DeclareUnicodeCharacter{7580}{\ensuremath{^\text{c}}}
\usepackage[references]{agda}

\newcommand{\Nat}{\AgdaDatatype{ℕ}}
\newcommand{\Int}{\AgdaDatatype{ℤ}}
\colorlet{AgdaKeyword}{BlueViolet}
\colorlet{AgdaFunction}{OliveGreen}
\colorlet{AgdaDatatype}{OliveGreen}
\colorlet{AgdaPostulate}{OliveGreen}
\colorlet{AgdaPrimitive}{OliveGreen}
\colorlet{AgdaRecord}{OliveGreen}
\colorlet{AgdaRecordtype}{OliveGreen}
\colorlet{AgdaPrimitiveType}{OliveGreen}
\colorlet{AgdaModule}{OliveGreen}
\colorlet{AgdaInductiveConstructor}{Sepia}
\colorlet{AgdaField}{Sepia}
\renewcommand{\AgdaKeywordFontStyle}[1]{\textbf{#1}}
\usepackage{csquotes}


\usepackage{tikz-cd}
\tikzcdset{
  arrow style=tikz,
  diagrams={>={Straight Barb[scale=0.8]}},
}
\usepackage{amsmath}
\usepackage{amsthm}
\renewcommand{\qed}{\ensuremath{\hfill\blacksquare}}

% Boxed theorem envs without italics

\usepackage[framemethod=tikz]{mdframed}

\newtheoremstyle{roman}
{0} % Space above
{0} % Space below
{} % Body font
{} % Indent amount
{\bfseries} % Theorem head font
{.} % Punctuation after theorem head
{.5em} % Space after theorem head
{} % Theorem head spec (can be left empty, meaning `normal')

\theoremstyle{roman}
\newmdtheoremenv[default,nobreak=true]{definition}{Definition}
\newmdtheoremenv[default,nobreak=true]{lemma}{Lemma}
\newmdtheoremenv[default,nobreak=true]{theorem}{Theorem}
\newmdtheoremenv[
  default,
  nobreak=true,
  linecolor=white,
  tikzsetting = {draw=black, line width = 1pt,dotted},
  ]{agdalisting}{Listing}

% Haskell bind operator >>=
\newcommand\hbind{%
  \ensuremath{\gg\mkern-5.5mu=}%
}
% Haskell apply operator <*>
\newcommand\hap{%
  \ensuremath{\mathbin{<\mkern-9mu*\mkern-9mu>}}%
}
% Haskell alternative operator <|>
\newcommand\halt{%
  \ensuremath{\mathbin{<\mkern-5mu\vert\mkern-5mu>}}%
}
% Haskell fmap operator <$>$
\newcommand\hfmap{%
  \ensuremath{\mathbin{<\mkern-6mu\$\mkern-6mu>}}%
}
% Haskell mappend operator <>
\newcommand\hcmb{%
  \ensuremath{\mathbin{\diamond}}%
}
% Haskell fish operator <><
\newcommand\hcmbin{%
  \ensuremath{\mathbin{\diamond\mkern-7.4mu\rtimes}}%
}
% Haskell Kleisli composition <=<
\newcommand\hkcomp{%
  \ensuremath{\mathbin{<\mkern-10mu=\mkern-7mu<}}
}
% Haskell forall
\newcommand\hforall{%
  \ensuremath{\forall}%
}
% Unique membership \in!
\newcommand\inunique{%
  \mathrel{\in\mkern-6mu\raisebox{-0.5pt}{!}}
}
% Decide membership \in?
\newcommand\decin{%
  \mathrel{\in\mkern-6mu\raisebox{-0.5pt}{?}}
}
% Dot as wildcard character
\newcommand\wc{{\mkern 2mu\cdot\mkern 2mu}}
% Binary operator !
\newcommand\ind{\mathbin{!}}
\usepackage{subcaption}
\usepackage{locallhs2TeX}
\renewcommand{\Varid}[1]{\mathit{#1}}
\renewcommand{\Conid}[1]{\textcolor{OliveGreen}{\textsf{#1}}}
\usepackage{multicol}
\usepackage{todonotes}
\usepackage{tabularx}
\usepackage{pict2e}

% Proper double left paren [( (also ``plano-concave lens'')
\newcommand{\lbparen}{%
  \mathopen{%
    \sbox0{$()$}%
    \setlength{\unitlength}{\dimexpr\ht0+\dp0}%
    \raisebox{-\dp0}{%
      \begin{picture}(.32,1)
        \linethickness{\fontdimen8\textfont3}
        \roundcap
        \put(0,0){\raisebox{\depth}{$($}}
        \polyline(0.32,0)(0,0)(0,1)(0.32,1)
      \end{picture}%
    }%
  }%
}

% Proper double right paren [( (also ``plano-concave lens'')
\newcommand{\rbparen}{%
  \mathclose{%
    \sbox0{$()$}%
    \setlength{\unitlength}{\dimexpr\ht0+\dp0}%
    \raisebox{-\dp0}{%
      \begin{picture}(.32,1)
        \linethickness{\fontdimen8\textfont3}
        \roundcap
        \put(-0.08,0){\raisebox{\depth}{$)$}}
        \polyline(0,0)(0.32,0)(0.32,1)(0,1)
      \end{picture}%
    }%
  }%
}

\usepackage[square,numbers]{natbib}

\PrerenderUnicode{í}

\usepackage{hyperref}

% Nice typesetting for the definition of an inductive type.
% For the following Agda-like thing:
%
% data List a where
%   []   : List a
%   (::) : a -> List a -> List a    
%
% write:
% \begin{inductivetype}{\mathbf{List}(A)}
%   \inductivetypeclause{\left[ \right]}{\mathbf{List}(A)}
%   \inductivetypeclause{\wc \dblcolon \wc}{A \rightarrow \mathbf{List}(A) \rightarrow \mathbf{List}(A)}
% \end{inductivetype}

\newif\iffirstclauseofinductivetype
\newenvironment{inductivetype}[1]{%
  \firstclauseofinductivetypetrue
  \begin{equation}
    \begin{alignedat}{3}
      #1
    }{%
    \end{alignedat}
  \end{equation}%
}
\newcommand{\inductivetypeclause}[2]{%
  \iffirstclauseofinductivetype
    \coloneqq
  \else
    \\ \vert
  \fi
  & \; #1 &:&& \; #2 ;
  \firstclauseofinductivetypefalse
}
\newmdenv[
leftmargin = 0pt,
innerleftmargin = 1em,
innertopmargin = 0pt,
innerbottommargin = 0pt,
innerrightmargin = 0pt,
rightmargin = 0pt,
linewidth = 3pt,
topline = false,
rightline = false,
bottomline = false
]{leftbar}

\makeatletter
\renewenvironment{proof}[1][\proofname]{%
  \begin{leftbar}%
  \par
  \pushQED{\qed}%
  \normalfont \topsep6\p@\@plus6\p@\relax
  \list{}{%
    \leftmargin=.5em
    % \rightmargin=\leftmargin
    \settowidth{\itemindent}{\itshape#1}%
    \labelwidth=\itemindent
    % the following line is not needed with amsart, but might be with other classes
    \parsep=0pt \listparindent=\parindent 
  }
\item[\hskip\labelsep
  \itshape
  #1\@addpunct{.}]\ignorespaces
}{%
  \popQED\endlist\@endpefalse
  \end{leftbar}
}
\makeatother

%%% Local Variables:
%%% mode: latex
%%% TeX-master: "paper"
%%% End: